\documentclass{assignment}
\ProjectInfos{高等量子力学}{PHYS5001P}{2022-2023 学年第一学期}{作业一}{截止时间: 2022 年 9 月 12 日 (周一)}{陈稼霖}[https://github.com/Chen-Jialin]{SA21038052}

\begin{document}
\begin{prob}
    \begin{itemize}
        \item[(a)] 考虑两个右矢 $\lvert\alpha\rangle$ 和 $\lvert\beta\rangle$. 假定 $\langle a'\vert\alpha\rangle$, $\langle a''\vert\alpha\rangle$, $\cdots$ 和 $\langle a\vert\beta\rangle$, $\langle a''\vert\beta\rangle$, $\cdots$, 均为已知, 其中 $\lvert a'\rangle$, $\lvert a''\rangle$, $\cdots$ 组成基右矢的完备基. 求在该基下算符 $\lvert\alpha\rangle\langle\beta\rvert$ 的矩阵表示.
        \item[(b)] 现在考虑一个自旋 $\frac{1}{2}$ 系统, 设 $\lvert\alpha\rangle$ 和 $\lvert\beta\rangle$ 分别为 $\lvert s_z=\hbar/2\rangle$ 和 $\lvert s_x=\hbar/2\rangle$ 态. 写出在通常 ($s_z$ 对角) 的基下, 与 $\lvert\alpha\rangle\langle\beta\rvert$ 对应的方阵的显示式.
    \end{itemize}
\end{prob}
\begin{sol}
    \begin{itemize}
        \item[(a)] 
        \item[(b)] 
    \end{itemize}
\end{sol}

\begin{prob}
    假定 $\lvert i\rangle$ 和 $\lvert j\rangle$ 都是某厄米算符 $A$ 的本征右矢. 在什么条件下, $\lvert i\rangle+\lvert j\rangle$ 也是 $A$ 的一个本征右矢. 证明答案的正确性.
\end{prob}
\begin{sol}
    
\end{sol}

\begin{prob}
    考虑被厄米算符 $A$ 的本征右矢 $\{\lvert a'\rangle\}$ 所张的一个右矢空间. 不存在任何简并.
    \begin{itemize}
        \item[(a)] 证明
        \[
            \prod_{a'}(A-a')
        \]
        是零算符.
        \item[(b)] 解释
        \[
            \prod_{a''\neq a'}\frac{(A-a'')}{(a'-a'')}
        \]
        的意义.
        \item[(c)] 令 $A$ 等于自旋 $\frac{1}{2}$ 系统的 $S_z$, 用它解释 (a) 与 (b).
    \end{itemize}
\end{prob}
\begin{sol}
    \begin{itemize}
        \item[(a)] 
        \item[(b)] 
        \item[(c)] 
    \end{itemize}
\end{sol}
\end{document}