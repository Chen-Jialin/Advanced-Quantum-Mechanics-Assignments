\documentclass{assignment}
\ProjectInfos{高等量子力学}{PHYS5001P}{2022-2023 学年第一学期}{作业一}{截止时间: 2022 年 9 月 12 日 (周一)}{陈稼霖}[https://github.com/Chen-Jialin]{SA21038052}

\begin{document}
\begin{prob}[课本习题 1.6]
    \begin{itemize}
        \item[(a)] 考虑两个右矢 $\lvert\alpha\rangle$ 和 $\lvert\beta\rangle$. 假定 $\langle a'\vert\alpha\rangle$, $\langle a''\vert\alpha\rangle$, $\cdots$ 和 $\langle a\vert\beta\rangle$, $\langle a''\vert\beta\rangle$, $\cdots$, 均为已知, 其中 $\lvert a'\rangle$, $\lvert a''\rangle$, $\cdots$ 组成基右矢的完备基. 求在该基下算符 $\lvert\alpha\rangle\langle\beta\rvert$ 的矩阵表示.
        \item[(b)] 现在考虑一个自旋 $\frac{1}{2}$ 系统, 设 $\lvert\alpha\rangle$ 和 $\lvert\beta\rangle$ 分别为 $\lvert s_z=\hbar/2\rangle$ 和 $\lvert s_x=\hbar/2\rangle$ 态. 写出在通常 ($s_z$ 对角) 的基下, 与 $\lvert\alpha\rangle\langle\beta\rvert$ 对应的方阵的显示式.
    \end{itemize}
\end{prob}
\begin{sol}
    \begin{itemize}
        \item[(a)] 将这两个右矢 $\lvert\alpha\rangle$ 和 $\lvert\beta\rangle$ 分别用这组完备基展开:
        \begin{align}
            \lvert\alpha\rangle=&\sum_{m=a',a'',\cdots}\lvert m\rangle\langle m\vert\alpha\rangle,\\
            \lvert\beta\rangle=&\sum_{n=a',a'',\cdots}\lvert n\rangle\langle n\vert\beta\rangle.
        \end{align}
        将上面这两个展开式代入算符 $\lvert\alpha\rangle\langle\beta\rvert$ 中可得其矩阵表示:
        \begin{align}
            \notag\lvert\alpha\rangle\langle\beta\rvert=&\sum_{m=a',a'',\cdots}\lvert m\rangle\langle m\vert\alpha\rangle\sum_{n=a',a'',\cdots}\langle n\rvert\langle\beta\vert n\rangle=\sum_{m=a',a'',\cdots}\sum_{n=a',a'',\cdots}\langle m\vert\alpha\rangle\langle\beta\vert n\rangle\lvert m\rangle\langle n\rvert\\
            =&\begin{bmatrix}
                \langle a'\vert\alpha\rangle\langle\beta\vert a'\rangle&\langle a'\vert\alpha\rangle\langle\beta\vert a''\rangle&\cdots\\
                \langle a''\vert\alpha\rangle\langle\beta\vert a'\rangle&\langle a''\vert\alpha\rangle\langle\beta\vert a''\rangle&\cdots\\
                \vdots&\vdots&\ddots
            \end{bmatrix}=\begin{bmatrix}
                \langle a'\vert\alpha\rangle\langle a'\vert \beta\rangle^*&\langle a'\vert\alpha\rangle\langle a''\vert\beta\rangle^*&\cdots\\
                \langle a''\vert\alpha\rangle\langle a'\vert\beta\rangle^*&\langle a''\vert\alpha\rangle\langle a''\vert\beta\rangle^*&\cdots\\
                \vdots&\vdots&\ddots
            \end{bmatrix}.
        \end{align}
        \item[(b)] 将这两个右矢 $\lvert\alpha\rangle$ 和 $\lvert\beta\rangle$ 分别用 $s_z$ 对角基展开:
        \begin{align}
            \lvert\alpha\rangle=&\lvert s_z=\hbar/2\rangle,\\
            \lvert\beta\rangle=&\lvert s_x=\hbar/2\rangle=\frac{1}{\sqrt{2}}(\lvert s_z=\hbar/2\rangle+\lvert s_z=-\hbar/2\rangle).
        \end{align}
        将上面这两个展开式代入算符 $\lvert\alpha\rangle\langle\beta\rvert$ 中可得其矩阵表示:
        \begin{align}
            \lvert\alpha\rangle\langle\beta\rvert=\frac{1}{\sqrt{2}}\begin{bmatrix}
                1&1\\
                0&0
            \end{bmatrix}.
        \end{align}
    \end{itemize}
\end{sol}

\begin{prob}[课本习题 1.7]
    假定 $\lvert i\rangle$ 和 $\lvert j\rangle$ 都是某厄米算符 $A$ 的本征右矢. 在什么条件下, $\lvert i\rangle+\lvert j\rangle$ 也是 $A$ 的一个本征右矢. 证明答案的正确性.
\end{prob}
\begin{sol}
    当 $\lvert i\rangle$ 和 $\lvert j\rangle$ 简并, 即 $\lvert i\rangle$ 和 $\lvert j\rangle$ 对应的本征值相同时, $\lvert i\rangle+\lvert j\rangle$ 也是 $A$ 的一个本征右矢.

    证明: 假设 $\lvert i\rangle$, $\lvert j\rangle$ 和 $(\lvert i\rangle+\lvert j\rangle)$ 对应的本征值分别为 $a_i$, $a_j$ 和 $a$, 即
    \begin{align}
        A\lvert i\rangle=&a_i\lvert i\rangle,\\
        A\lvert j\rangle=&a_j\lvert j\rangle,\\
        A(\lvert i\rangle+\lvert j\rangle)=&a(\lvert i\rangle+\lvert j\rangle),
    \end{align}
    则
    \begin{gather}
        A(\lvert i\rangle+\lvert j\rangle)=a_i\lvert i\rangle+a_j\lvert j\rangle=a(\lvert i\rangle+\lvert j\rangle),\\
        \Longrightarrow(a_i-a)\lvert i\rangle+(a_j-a)\lvert j\rangle=0.
    \end{gather}
    由于 $\lvert i\rangle$ 和 $\lvert j\rangle$ 均为 $A$ 的本征右矢, 故 $\lvert i\rangle$ 和 $\lvert j\rangle$ 线性无关, 从而
    \begin{gather}
        a_i-a=a_j-a=0,\\
        \Longrightarrow a_i=a_j=0,
    \end{gather}
    即 $\lvert i\rangle$ 和 $\lvert j\rangle$ 简并.
\end{sol}

\begin{prob}[课本习题 1.8]
    考虑被厄米算符 $A$ 的本征右矢 $\{\lvert a'\rangle\}$ 所张的一个右矢空间. 不存在任何简并.
    \begin{itemize}
        \item[(a)] 证明
        \[
            \prod_{a'}(A-a')
        \]
        是零算符.
        \item[(b)] 解释
        \[
            \prod_{a''\neq a'}\frac{(A-a'')}{(a'-a'')}
        \]
        的意义.
        \item[(c)] 令 $A$ 等于自旋 $\frac{1}{2}$ 系统的 $S_z$, 用它解释 (a) 与 (b).
    \end{itemize}
\end{prob}
\begin{sol}
    \begin{itemize}
        \item[(a)] 该右矢空间中任一矢量 $\lvert\alpha\rangle$ 都可用这组本征右矢 $\{\lvert a'\rangle\}$ 展开:
        \begin{align}
            \lvert\alpha\rangle=\sum_{a''}\lvert a''\rangle\langle a''\vert \alpha\rangle.
        \end{align}
        将算符 $\prod_{a'}(A-a')$ 作用于该矢量上有
        \begin{align}
            \notag\prod_{a'}(A-a')\lvert\alpha\rangle=&\prod_{a'}(A-a')\sum_{a''}\lvert a''\rangle\langle a''\vert\alpha\rangle=\sum_{a''}\prod_{a'}(A-a')\lvert a''\rangle\langle a''\vert\alpha\rangle=\sum_{a''}\left[\prod_{a'}(a''-a')\right]\lvert a''\rangle\langle a''\vert\alpha\rangle\\
            =&\sum_{a''}0\lvert a''\rangle\langle a''\vert\alpha\rangle=0.
        \end{align}
        考虑到 $\lvert\alpha\rangle$ 的任意性, $\prod_{a'}(A-a')$ 为零算符.
        \item[(b)] 算符 $\prod_{a''\neq a'}\frac{(A-a'')}{(a'-a'')}$ 作用于 $\lvert a'\rangle$, 则仍为 $\lvert a'\rangle$, 算符作用于 $\{\lvert a'\rangle\}$ 中非 $\lvert a'\rangle$ 的任一右矢, 则得零,
        \begin{align}
            \prod_{a''\neq a'}\frac{(A-a'')}{(a'-a'')}\lvert a'\rangle=&\prod_{a''\neq a'}\frac{(a'-a'')}{(a'-a'')}\lvert a'\rangle=\lvert a'\rangle,\\
            \prod_{a''\neq a'}\frac{(A-a'')}{(a'-a'')}\lvert a_n\rangle=&\prod_{a''\neq a'}\frac{(a_n-a'')}{(a'-a'')}=0,\text{for }\lvert a_n\rangle\in\{\lvert a'\rangle\},
        \end{align}
        换言之, 算符 $\prod_{a''\neq a'}\frac{(A-a'')}{(a'-a'')}$ 为 $\lvert a'\rangle$ 对应的投影算符.
        \item[(c)] 若 $A=S_z$, 则其本征右矢的集合为 $\{\lvert S_z=\frac{\hbar}{2}\rangle,\lvert S_z=-\frac{\hbar}{2}\rangle\}$. 算符
        \begin{align}
            \notag\prod_{a'}(A-a')=&\left(A-\frac{\hbar}{2}\right)\left(A+\frac{\hbar}{2}\right)=\left(\frac{\hbar}{2}\begin{bmatrix}
                1&0\\
                0&-1
            \end{bmatrix}-\frac{\hbar}{2}\begin{bmatrix}
                1&0\\
                0&1
            \end{bmatrix}\right)\left(\frac{\hbar}{2}\begin{bmatrix}
                1&0\\
                0&-1
            \end{bmatrix}+\frac{\hbar}{2}\begin{bmatrix}
                1&0\\
                0&1
            \end{bmatrix}\right)\\
            =&\frac{\hbar}{2}\begin{bmatrix}
                0&0\\
                0&-2
            \end{bmatrix}\frac{\hbar}{2}\begin{bmatrix}
                2&0\\
                0&0
            \end{bmatrix}=\begin{bmatrix}
                0&0\\
                0&0
            \end{bmatrix},
        \end{align}
        为零算符.
        算符
        \begin{align}
            \prod_{a''\neq\frac{\hbar}{2}}\frac{(A-a'')}{\left(\frac{\hbar}{2}-a''\right)}=\frac{1}{\frac{\hbar}{2}+\frac{\hbar}{2}}\left(\frac{\hbar}{2}\begin{bmatrix}
                1&0\\
                0&-1
            \end{bmatrix}+\frac{\hbar}{2}\begin{bmatrix}
                1&0\\
                0&1
            \end{bmatrix}\right)=\begin{bmatrix}
                1&0\\
                0&0
            \end{bmatrix}=\lvert S_z=\frac{\hbar}{2}\rangle\langle S_z=\frac{\hbar}{2}\rvert
        \end{align}
        为 $\lvert S_z=\frac{\hbar}{2}\rangle$ 对应的投影算符,
        \begin{align}
            \prod_{a''\neq-\frac{\hbar}{2}}\frac{(A-a'')}{\left(-\frac{\hbar}{2}-a''\right)}=\frac{1}{-\frac{\hbar}{2}-\frac{\hbar}{2}}\left(-\frac{\hbar}{2}\begin{bmatrix}
                1&0\\
                0&-1
            \end{bmatrix}+\frac{\hbar}{2}\begin{bmatrix}
                1&0\\
                0&1
            \end{bmatrix}\right)=\begin{bmatrix}
                0&0\\
                0&1
            \end{bmatrix}=\lvert S_z=-\frac{\hbar}{2}\rangle\langle S_z=-\frac{\hbar}{2}\rvert
        \end{align}
        为 $\lvert S_z=-\frac{\hbar}{2}\rangle$ 对应的投影算符.
    \end{itemize}
\end{sol}

\begin{prob}[课本习题 1.4]
    利用左矢-右矢代数规则证明或计算下列各式:
    \begin{itemize}
        \item[(a)] $\tr(XY)=\tr(YX)$, 其中 $X$ 和 $Y$ 都是算符.
        \item[(b)] $(XY)^{\dagger}=Y^{\dagger}X^{\dagger}$, 其中 $X$ 和 $Y$ 都是算符.
        \item[(c)] 在左矢-右矢形式下 $\exp[if(A)]=$? 其中 $A$ 是厄米算符, 其本征值是已知的.
        \item[(d)] $\sum_{a'}\psi_{a'}^*(\bm{x}')\psi_{a'}(\bm{x}'')$, 其中 $\psi_{a'}(\bm{x}')=\langle\bm{x}'\vert a'\rangle$.
    \end{itemize}
\end{prob}
\begin{sol}
    \begin{itemize}
        \item[(a)] 
        \begin{align}
            \tr(XY)=&\sum_{a'}\langle a'\rvert XY\lvert a'\rangle=\sum_{a',a''}\langle a'\rvert X\lvert a''\rangle\langle a''\rvert Y\lvert a'\rangle=\sum_{a',a''}\langle a''\rvert Y\lvert a'\rangle\langle a'\rvert X\lvert a''\rangle=\sum_{a''}\langle a''\rvert YX\lvert a''\rangle=\tr(YX).
        \end{align}
        \item[(b)] 一方面,
        \begin{align}
            [XY\lvert\alpha\rangle]^{\dagger}=[(XY)\lvert\alpha\rangle]^{\dagger}=\langle\alpha\rvert(XY)^{\dagger},
        \end{align}
        另一方面,
        \begin{align}
            [XY\lvert\alpha\rangle]^{\dagger}=[X(Y\lvert\alpha\rangle)]^{\dagger}=(Y\lvert\alpha\rangle)^{\dagger}X^{\dagger}=\langle\alpha\rvert Y^{\dagger}X^{\dagger},
        \end{align}
        考虑到 $\langle\alpha\rvert$ 的任意性, 有
        \begin{align}
            (XY)^{\dagger}=Y^{\dagger}X^{\dagger}.
        \end{align}
        \item[(c)] 假设 $A$ 的本征值和对应的本征矢为 $\{a'\}$ 和 $\{\lvert a'\rangle\}$. $f(A)$ 是关于算符 $A$ 的函数, 将其关于 $A$ 做泰勒展开并作用于本征态 $\lvert a'\rangle$, 有
        \begin{align}
            f(A)\lvert a'\rangle=\sum_{n=0}^{\infty}\frac{f^{(n)}(0)A^n}{n!}\lvert a'\rangle=\sum_{n=0}^{\infty}\frac{f^{(n)}(0){a'}^n}{n!}\lvert a'\rangle=f(a')\lvert a'\rangle,
        \end{align}
        故
        \begin{align}
            \notag\exp[if(A)]=&\sum_{a'}\exp[if(A)]\lvert a'\rangle\langle a'\rvert=\sum_{a'}\sum_n\frac{i^n[f(A)]^n}{n!}\lvert a'\rangle\langle a'\rvert=\sum_{a'}\sum_n\frac{i^n[f(a')]^n}{n!}\lvert a'\rangle\langle a'\rvert\\
            =&\sum_{a'}\exp[if(a')]\lvert a'\rangle\langle a'\rvert.
        \end{align}
        \item[(d)] 
        \begin{align}
            \sum_{a'}\psi_{a'}^*(\bm{x}')\psi_{a'}(\bm{x}'')=\sum_{a'}\langle a'\vert\bm{x}'\rangle\langle\bm{x}''\vert a'\rangle=\sum_{a'}\langle\bm{x}''\vert a'\rangle\langle a'\vert\bm{x}'\rangle=\langle\bm{x}''\vert\bm{x}'\rangle=\delta(\bm{x}'-\bm{x}'').
        \end{align}
    \end{itemize}
\end{sol}

\begin{prob}[课本习题 1.10]
    一个双胎系统其哈密顿算符由下式给出
    \[
        H=a(\lvert 1\rangle\langle 1\rvert-\lvert 2\rangle\langle 2\rvert+\lvert 1\rangle\langle 2\rvert+\lvert 2\rangle\langle 1\rvert),
    \]
    其中 $a$ 是一个数, 其量纲为能量. 求能量的本征值和相应的能量本征右矢 (作为 $\lvert 1\rangle$ 和 $\lvert 2\rangle$ 的线性组合).
\end{prob}
\begin{sol}
    在以 $\{\lvert 1\rangle,\lvert 2\rangle\}$ 为基的表象下, 该哈密顿量的矩阵表示为
    \begin{align}
        H=a\begin{bmatrix}
            1&1\\
            1&-1
        \end{bmatrix},
    \end{align}
    其特征方程为
    \begin{align}
        \det(H-EI)=\begin{vmatrix}
            a-E&a\\
            a&-a-E
        \end{vmatrix}=E^2-2a^2=0,
    \end{align}
    解得能量的本征值为
    \begin{align}
        E_1=\sqrt{2}a,\quad E_2=-\sqrt{2}a.
    \end{align}
    将这两个能量的本征值分别代入
    \begin{align}
        H\lvert\psi\rangle=E_{1/2}\lvert\psi\rangle,
    \end{align}
    解得相应的能量本征右矢分别为
    \begin{align}
        \lvert\psi_1\rangle=\frac{1}{\sqrt{4-2\sqrt{2}}}\begin{bmatrix}
            1\\
            \sqrt{2}-1
        \end{bmatrix},\quad\lvert\psi_2\rangle=\frac{1}{\sqrt{4+2\sqrt{2}}}\begin{bmatrix}
            -1\\
            \sqrt{2}+1
        \end{bmatrix}.
    \end{align}
\end{sol}

\begin{prob}[课本习题 1.13]
    一束自旋 $\frac{1}{2}$ 的原子通过如下一系列斯特恩-盖拉赫类的测量
    \begin{itemize}
        \item[(a)] 第一次测量存留 $s_z=\hbar/2$ 的原子而舍弃 $s_z=-\hbar/2$ 的原子.
        \item[(b)] 第二次测量存留 $s_n=\hbar/2$ 的原子而舍弃 $s_n=-\hbar/2$ 的原子, 其中 $s_n$ 是算符 $\bm{S}\cdot\dot{\bm{n}}$ 的本征值, 而 $\hat{\bm{n}}$ 在 $xz$ 平面上与 $z$ 轴夹角为 $\beta$.
        \item[(c)] 第三次测量存留 $s_z=-\hbar/2$ 的原子而舍弃 $s_z=\hbar/2$ 的原子. 当第一次测量存活下的 $s_z=\hbar/2$ 束流归一到 $1$ 时, 找到 $s_z=-\hbar/2$ 束流的强度是什么? 如果我们想使最后找到 $s_z=-\hbar/2$ 束流的强度取最大值, 我们必须怎样设置第二次测量仪器取向?
    \end{itemize}
\end{prob}
\begin{sol}
    第一次测量存留 $s_z=\hbar/2$ 的原子而舍弃 $s_z=-\hbar/2$ 的原子后, 原子的状态为 $\lvert s_z;+\rangle$, 此时找到 $s_z=-\hbar/2$ 束流的强度为 $0$.

    第二次测量存留 $s_n=\hbar/2$ 的原子而舍弃 $s_n=-\hbar/2$ 的原子后, 得到原子的状态为
    \begin{align}
        \lvert s_n;+\rangle=\cos\frac{\beta}{2}\lvert s_z;+\rangle+\sin\frac{\beta}{2}\lvert s_z;-\rangle,
    \end{align}
    束流的强度为
    \begin{align}
        I_2=\abs{\langle s_n;+\vert s_z;+\rangle}^2=\cos^2\frac{\beta}{2}.
    \end{align}
    第三次测量存留 $s_z=-\hbar/2$ 的原子而舍弃 $s_z=\hbar/2$ 的原子, 得到的束流的强度为
    \begin{align}
        I_3=I_2\abs{\langle s_z;-\vert s_n;+\rangle}^2=\cos^2\frac{\beta}{2}\sin^2\frac{\beta}{2}=\cos^2\frac{\beta}{2}\left(1-\cos^2\frac{\beta}{2}\right).
    \end{align}
    若想使最后找到 $s_z=-\hbar/2$ 束流的强度取最大值, 我们必须设置第二次测量仪器取向 $\hat{\bm{n}}$ 在 $xz$ 平面上与 $z$ 轴夹角 $\beta=\frac{\pi}{2}$, 此时束流强度为 $\frac{1}{4}$.
\end{sol}

\begin{prob}[课本习题 1.23]
    考虑一个三维右矢空间, 如果某一组正交的右矢集合, 比如 $\lvert 1\rangle$, $\lvert 2\rangle$, $\lvert 3\rangle$, 用作基右矢, 算符 $A$ 和 $B$ 由
    \[
        A=\begin{pmatrix}
            a&0&0\\
            0&-a&0\\
            0&0&-a
        \end{pmatrix},\quad B=\begin{pmatrix}
            b&0&0\\
            0&0&-ib\\
            0&ib&0
        \end{pmatrix}
    \]
    表示, 其中 $a$ 和 $b$ 都是实数.
    \begin{itemize}
        \item[(a)] 显然, $A$ 展示了一个简并的谱, $B$ 也展示了简并的谱吗?
        \item[(b)] 证明 $A$ 和 $B$ 对易.
        \item[(c)] 找到一组新的正交归一右矢集合, 它们是 $A$ 和 $B$ 的共同本征右矢, 具体确定在这三个本征右矢的每一个本征右矢上 $A$ 和 $B$ 的本征值. 你确定的本征值能完全地表征每个本征右矢吗?
    \end{itemize}
\end{prob}
\begin{sol}
    \begin{itemize}
        \item[(a)] $B$ 的特征方程为
        \begin{align}
            \det(B-\lambda I)=\begin{vmatrix}
                b-\lambda&0&0\\
                0&-\lambda&-ib\\
                0&ib&-\lambda
            \end{vmatrix}=(b-\lambda)[\lambda^2-b^2]=0,
        \end{align}
        解得本征值为
        \begin{align}
            \lambda_1=b,\quad\lambda_2=b,\quad\lambda_3=-b,
        \end{align}
        故 $B$ 也展示了简并的谱.
        \item[(b)] 
        \begin{align}
            \notag[A,B]=&AB-BA=\begin{bmatrix}
                a&0&0\\
                0&-a&0\\
                0&0&-a
            \end{bmatrix}\begin{bmatrix}
                b&0&0\\
                0&0&-ib\\
                0&ib&0
            \end{bmatrix}-\begin{bmatrix}
                b&0&0\\
                0&0&-ib\\
                0&ib&0
            \end{bmatrix}\begin{bmatrix}
                a&0&0\\
                0&-a&0\\
                0&0&-a
            \end{bmatrix}\\
            =&\begin{bmatrix}
                ab&0&0\\
                0&0&-iab\\
                0&iab&0
            \end{bmatrix}-\begin{bmatrix}
                ab&0&0\\
                0&0&-iab\\
                0&iab&0
            \end{bmatrix}=0,
        \end{align}
        故 $A$ 和 $B$ 对易.
        \item[(c)] $A$ 的本征值为 $\lambda_1=a$, $\lambda_2=-a$,$\lambda_3=-a$,其中第一个本征值对应的归一化本征右矢为 $\begin{bmatrix}
            1\\
            0\\
            0
        \end{bmatrix}$, 后两个简并本征值对应的正交归一右矢为 $\begin{bmatrix}
            0\\
            \cos\theta_1\\
            e^{i\varphi_1}\sin\theta_1
        \end{bmatrix}$ 和 $\begin{bmatrix}
            0\\
            -\sin\theta_1\\
            e^{-i\varphi_1}\cos\theta_1
        \end{bmatrix}$, 其中 $\theta_1$, $\varphi_1$ 待定.

        $B$ 的本征值为 $\lambda_1=b$, $\lambda_2=b$, $\lambda_3=-b$, 其中最后一个本征值对应的归一化本征右矢为 $\frac{1}{\sqrt{2}}\begin{bmatrix}
            0\\
            1\\
            -i
        \end{bmatrix}$, 前两个本征值对应的归一化本征右矢为 $\begin{bmatrix}
            \cos\theta_2\\
            e^{i\varphi_2}\frac{1}{\sqrt{2}}\sin\theta_2\\
            e^{i\varphi_2}\frac{i}{\sqrt{2}}\sin\theta_2
        \end{bmatrix}$ 和 $\begin{bmatrix}
            -\sin\theta_2\\
            e^{-i\varphi_2}\frac{1}{\sqrt{2}}\cos\theta_2\\
            e^{-i\varphi_2}\frac{i}{\sqrt{2}}\cos\theta_2
        \end{bmatrix}$, 其中 $\theta_2$, $\varphi_2$ 待定.

        要找到 $A$ 和 $B$ 共有的一组新的正交归一右矢集合, 则取 $\theta_1=\frac{\pi}{4}$, $\varphi_1=\frac{\pi}{2}$, $\theta_2=0$, $\varphi_2=\frac{\pi}{2}$, 从而有
        \begin{table}[H]
            \centering
            \begin{tabular}{|c|c|c|}
            \hline
            $A$ 的本征值 & $B$ 的本征值 & 正交归一本征右矢 \\ \hline
            $a$ & $b$ & $\lvert 1\rangle$ \\ \hline
            $-a$ & $-b$ & $\frac{1}{\sqrt{2}}(\lvert 2\rangle-i\lvert 3\rangle)$ \\ \hline
            $-a$ & $b$ & $-\frac{1}{\sqrt{2}}(\lvert 2\rangle+i\lvert 3\rangle)$ \\ \hline
            \end{tabular}
        \end{table}
        由上表知, 结合 $A$ 和 $B$ 的本征值可以完全地表征每个本征右矢.
    \end{itemize}
\end{sol}
\end{document}