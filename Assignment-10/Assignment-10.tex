\documentclass{assignment}
\ProjectInfos{高等量子力学}{PHYS5001P}{2022-2023 学年第一学期}{第十次作业}{截止时间: 2022 年 11 月 28 日 (周一)}{陈稼霖}[https://github.com/Chen-Jialin]{SA21038052}

\begin{document}
\begin{prob}[课本习题 5.11]
    一个双态系统的哈密顿量矩阵能被写为
    \[
        \mathcal{H}=\begin{pmatrix}
            E_1^0&\lambda\Delta\\
            \lambda\Delta&E_2^0
        \end{pmatrix}
    \]
    很清楚, 无微扰问题 ($\lambda=0$) 的能量本征函数由
    \[
        \phi_1^{(0)}=\begin{pmatrix}
            1\\
            0
        \end{pmatrix},\quad\phi_z^{(0)}=\begin{pmatrix}
            0\\
            1
        \end{pmatrix}
    \]
    给出.
    \begin{itemize}
        \item[(a)] \textit{精确}求解这个问题以找到能量本征函数 $\psi_1$ 和 $\psi_2$, 及能量本征值 $E_1$ 和 $E_2$.
        \item[(b)] 假定 $\lambda\abs{\Delta}\ll\abs{E_1^0-E_2^0}$, 使用时间无关微扰论, 求解同样的问题到一级的能量本征函数和到二级的能量本征值. 并与 (a) 中得到的精确结果比较.
        \item[(c)] 假定两个无微扰的能量是 ``几乎简并'' 的, 即
        \[
            \abs{E_1^0-E_2^0}\ll\lambda\abs{\Delta}.
        \]
        证明 (a) 中得到的精确结果非常像通过令 $E_1^0$ 严格等于 $E_2^0$, 而把\textit{简并}微扰论应用于这个问题时所期待的结果.
    \end{itemize}
\end{prob}
\begin{sol}
    \begin{itemize}
        \item[(a)] 通过
        \begin{align}
            \abs{\mathcal{H}-EI}=\begin{vmatrix}
                E_1^0-E&\lambda\Delta\\
                \lambda\Delta&E_2^0-E
            \end{vmatrix}=E^2-(E_1^0+E_2^0)E+E_1^0E_2^0-\lambda^2\Delta^2=0
        \end{align}
        解得该问题的能量本征值为
        \begin{align}
            E_{1,2}=\frac{(E_1^0+E_2^0)\pm\sqrt{(E_1^0-E_2^0)^2+4\lambda^2\Delta^2}}{2}.
        \end{align}
        将这两个能量本征值分别代入本征方程
        \begin{align}
            \mathcal{H}\psi_{1,2}=E_{1,2}\psi_{1,2}
        \end{align}
        中可得对应的归一化能量本征函数分别为
        \begin{align}
            \psi_1=\frac{1}{\sqrt{(E_1-E_2^0)^2+\lambda^2\Delta^2}}\begin{bmatrix}
                E_1-E_2^0\\
                \lambda\Delta
            \end{bmatrix},\quad\psi_2=\frac{1}{\sqrt{(E_2-E_1^0)^2+\lambda^2\Delta^2}}\begin{bmatrix}
                \lambda\Delta\\
                E_2-E_1^0
            \end{bmatrix}.
        \end{align}
        \item[(b)] \uline{使用时间无关微扰论,} 该问题的精确到一级的能量本征函数为
        \begin{align}
            \psi_1=\phi_1^0+\lambda\frac{\Delta}{E_1^0-E_2^0}\phi_2^0,\quad\psi_2=\phi_2^0+\lambda\frac{\Delta}{E_2^0-E_1^0}\phi_1^0.
        \end{align}
        精确到二级的能量本征值为
        \begin{align}
            E_1=E_1^0+\frac{\lambda^2\abs{\Delta}^2}{E_1^0-E_2^0},\quad E_2=E_2^0+\frac{\lambda^2\abs{\Delta}^2}{E_2^0-E_1^0}.
        \end{align}

        \uline{由 (a) 中的精确结果,} 关于 $\lambda\Delta$ 展开并取到一阶小量的能量本征函数为
        \begin{align}
            \psi_1=\begin{bmatrix}
                1\\
                \frac{\lambda\Delta}{E_1^0-E_2^0}
            \end{bmatrix},\quad\psi_2=\begin{bmatrix}
                \frac{\lambda\Delta}{E_2^0-E_1^0}\\
                1
            \end{bmatrix}.
        \end{align}
        关于 $\lambda\Delta$ 展开并取到二阶小量的能量本征值为
        \begin{align}
            E_1=\max\{E_1^0,E_2^0\}+\frac{\lambda\abs{\Delta}^2}{\abs{E_1^0-E_2^0}},\quad E_2=\min\{E_1^0,E_2^0\}-\frac{\lambda\abs{\Delta}}{\abs{E_1^0-E_2^0}}.
        \end{align}
        可见, 由时间无关微扰论得到的结果与精确结果一致.
        \item[(c)] \uline{先令 $E_1^0$ 严格等于 $E_2^0$ 并应用简并微扰论:} 微扰矩阵为
        \begin{align}
            V=\begin{bmatrix}
                0&\lambda\Delta\\
                \lambda\Delta&0
            \end{bmatrix},
        \end{align}
        通过久期方程
        \begin{align}
            \abs{V-EI}=\begin{vmatrix}
                -\Delta^{(1)}&\lambda\Delta\\
                \lambda\Delta&-\Delta^{(1)}
            \end{vmatrix}=(\Delta^{(1)})^2-(\lambda\Delta)^2=0
        \end{align}
        解得该微扰矩阵的本征能量 (一阶能量修正) 为
        \begin{align}
            \Delta_{1,2}^{(1)}=\pm\lambda\Delta.
        \end{align}
        将这两个本征值代入微扰矩阵的本征方程
        \begin{align}
            V\psi_{1,2}^{(0)}=\Delta_{1,2}^{(1)}\psi_{1,2}^{(0)}
        \end{align}
        解得对应的零阶本征矢分别为
        \begin{align}
            \psi_{1,2}^{(0)}=\frac{1}{\sqrt{2}}\begin{bmatrix}
                1\\
                \pm1
            \end{bmatrix}.
        \end{align}
        原简并子空间内的一阶态矢修正为
        \begin{align}
            P_0\psi_1^{(1)}=&\frac{\lambda\psi_2^{(0)}}{\Delta_1^{(1)}-\Delta_2^{(1)}}\sum_{k\notin D}\cdots=0,\\
            P_0\psi_2^{(1)}=&\frac{\lambda\psi_1^{(0)}}{\Delta_2^{(1)}-\Delta_1^{(1)}}\sum_{k\notin D}\cdots=0,
        \end{align}
        原简并子空间外的一阶态矢修正为
        \begin{align}
            P_1\psi_{1,2}^{(1)}=0
        \end{align}
        故一阶态矢修正为
        \begin{align}
            \psi_{1,2}^{(1)}=P_0\psi_{1,2}^{(1)}+P_1\psi_{1,2}^{(1)}=0.
        \end{align}
        二阶能量修正为
        \begin{align}
            \Delta_{1,2}^{(2)}=\sum_{k\notin D}\cdots=0.
        \end{align}
        综上, 精确到一级的能量本征函数为
        \begin{align}
            \psi_{1,2}=\frac{1}{\sqrt{2}}\begin{bmatrix}
                1\\
                \pm 1
            \end{bmatrix},
        \end{align}
        精确到二级的能量本征值为
        \begin{align}
            E_{1,2}=E_1^0\pm\lambda\Delta.
        \end{align}

        \uline{再精确求解:} 当两个无微扰的能量几乎简并, 即 $\abs{E_1^0-E_2^0}\ll\lambda\abs{\Delta}$ 时, 能量本征函数为
        \begin{align}
            \psi_{1,2}=\frac{1}{\sqrt{2}}\begin{bmatrix}
                \pm 1\\
                1
            \end{bmatrix},
        \end{align}
        能量本征值为
        \begin{align}
            E_{1,2}=\frac{E_1^0+E_2^0}{2}\pm\lambda\Delta.
        \end{align}
        可见, 通过简并微扰论得到的结果与精确结果是一致的.
    \end{itemize}
\end{sol}

\begin{prob}[课本习题 5.13]
    在下述情况下计算氢的 \ce{2S_{1/2}} 和 \ce{2P_{1/2}} 能级的斯塔克效应, 在那里场 $\varepsilon$ 足够的弱以至 $e\varepsilon a_0$ 小于精细结构, 但要计入兰姆位移 $\delta$ ($\delta=$ \SI{1057}{MHz}) (即在计算中忽略 \ce{2P_{3/2}} 能级). 证明: 当 $e\varepsilon a_0\ll\delta$ 时, 能移对 $\varepsilon$ 是二次的; 而当 $e\varepsilon a_0\gg\delta$ 时, 能移对 $\varepsilon$ 是线性的. (所需径向积分是 $\langle 2s\rvert r\lvert 2p\rangle=3\sqrt{3}a_0$.) 简要地讨论这个问题的时间反演结果 (如果有的话). 这个问题取自 Gottfried 1966, 习题 7-3.
\end{prob}
\begin{sol}
    无微扰时, 考虑兰姆位移 $\delta$, 能量本征态 $\lvert 2S_{1/2}\rangle$ 和 $\lvert 2P_{1/2}\rangle$ 的能量本征值分别为 $E_2+\delta$ 和 $E_2$, 故系统的无微扰哈密顿量可表为
    \begin{align}
        H_0=\begin{bmatrix}
            E_2+\delta&0\\
            0&E_2
        \end{bmatrix}.
    \end{align}
    外加电场 $\varepsilon$ 带来的微扰为
    \begin{align}
        V=-e\varepsilon z,
    \end{align}
    其中由于 $\lvert 2S_{1/2}\rangle$ 和 $\lvert 2P_{1/2}\rangle$ 分别具有偶/奇宇称, 故 $\langle 2S_{1/2}\rvert V\lvert 2S_{1/2}\rangle=\langle 2P_{1/2}\rvert V\lvert 2P_{1/2}\rangle=0$, 由于径向积分 $\langle 2s\rvert r\lvert 2p\rangle=\langle 2s\rvert\sqrt{x^2+y^2+z^2}\lvert 2p\rangle=\langle 2s\rvert\sqrt{3}\abs{z}\lvert 2p\rangle=3\sqrt{3}a_0$, 故 $\langle 2s\rvert V\lvert 2p\rangle=-3e\varepsilon a_0$, 从而微扰可表为
    \begin{align}
        V=\begin{bmatrix}
            0&-3e\varepsilon a_0\\
            -3e\varepsilon a_0&0
        \end{bmatrix}.
    \end{align}
    微扰下哈密顿量为
    \begin{align}
        H=H_0+V=\begin{bmatrix}
            E_2+\delta&-3e\varepsilon a_0\\
            -3e\varepsilon a_0&E_2
        \end{bmatrix}.
    \end{align}

    当 $e\varepsilon a_0\ll\delta$ 时, 可认为 $\lvert 2S_{1/2}\rangle$ 和 $\lvert 2P_{1/2}\rangle$ 不简并, 利用\uline{非简并定态微扰论}, 精确到二级的能量本征值为
    \begin{align}
        E_{2S_{1/2}}=E_2+\delta+\frac{(3e\varepsilon a_0)^2}{\delta},\quad E_{2P_{1/2}}=E_2-\frac{(3e\varepsilon a_0)^2}{\delta},
    \end{align}
    故此时的能移对 $\varepsilon$ 是二次的.

    当 $e\varepsilon a_0\gg\delta$ 时, 可认为 $\lvert 2S_{1/2}\rangle$ 和 $\lvert 2P_{1/2}\rangle$ 近简并, 利用\uline{简并定态微扰论}, 通过久期方程
    \begin{align}
        \abs{V-\Delta^{(1)}I}=\begin{vmatrix}
            \Delta^{(1)}&-3e\varepsilon a_0\\
            -3e\varepsilon a_0&\Delta^{(0)}
        \end{vmatrix}=(\Delta^{(1)})^2-(3e\varepsilon a_0)^2=0
    \end{align}
    解得一阶能量修正为
    \begin{align}
        \Delta_{2S_{1/2}}^{(1)}=+3e\varepsilon a_0,\quad\Delta_{2P_{1/2}}^{(1)}=-3e\varepsilon a_0,
    \end{align}
    故此时的能移对 $\varepsilon$ 是线性的.

    在时间反演变换下, $\lvert 2S_{1/2}\rangle\rightarrow\lvert 2S_{1/2}\rangle$ (不变), $\lvert 2P_{1/2}\rangle\rightarrow\lvert 2P_{1/2}\rangle$ 不变, 位置坐标 $\bm{x}\rightarrow\bm{x}$ (不变), 电场 $\varepsilon\rightarrow\varepsilon$ (不变), 从而微扰算符 $V$ 仅改变符号, 用与上文相同的推导方法最终得到的能移是相同的.
\end{sol}

\begin{prob}[补充习题]
    对类氢原子, 求解 $\langle nlm\rvert\frac{1}{r}\frac{\partial V_c}{\partial r}\lvert nlm\rangle$.
\end{prob}
\begin{sol}
    对类氢原子,
    \begin{align}
        V_c=-\frac{Ze^2}{r},
    \end{align}
    故
    \begin{align}
        \langle nlm\rvert\frac{1}{r}\frac{\partial V_c}{\partial r}\lvert nlm\rangle=\langle nlm\rvert\frac{Ze^2}{r^3}\lvert nlm\rangle.
    \end{align}
    注意到对任意算符 $A$, 均有
    \begin{align}
        \langle nlm\rvert[H_0,A]\lvert nlm\rangle=\langle nlm\rvert H_0A\lvert nlm\rangle-\langle nlm\rvert AH_0\rvert nlm\rangle=E_{nl}^{(0)}\langle nlm\rvert A\lvert nlm\rangle-E_{nl}^{(0)}\langle nlm\rvert A\lvert nlm\rangle=0.
    \end{align}
    取径向动量算符 $A=p_r=-i\hbar\left(\frac{\partial}{\partial r}+\frac{1}{r}\right)$, 则有
    \begin{align}
        \notag&\langle nlm\rvert\left[-\frac{\hbar^2}{2m}\left(\frac{\mathrm{d}^2}{\mathrm{d}r^2}+\frac{2}{r}\frac{\mathrm{d}}{\mathrm{d}r}\right)+\frac{\bm{L}^2}{2mr^2}+V_c(r),p_r\right]\lvert nlm\rangle\\
        =&\langle nlm\rvert\left[-\frac{\hbar^2}{2m}\left(\frac{\mathrm{d}^2}{\mathrm{d}r^2}+\frac{2}{r}\frac{\mathrm{d}}{\mathrm{d}r}\right)+\frac{l(l+1)\hbar^2}{2mr^2}+V_c(r),p_r\right]\lvert nlm\rangle=0,
    \end{align}
    其中 $H_0$ 的径向部分 $-\frac{\hbar^2}{2m}\left(\frac{\mathrm{d}^2}{\mathrm{d}r^2}+\frac{2}{r}\frac{\mathrm{d}}{\mathrm{d}r}\right)$ 是与径向动量算符 $p_r$ 对易的, 而 $H_0$ 的角向部分 $\frac{l(l+1)\hbar^2}{2mr^2}$ 和带有 $\frac{\partial}{\partial r}$ 的 $V_c$ 与径向角动量算符 $p_r$ 不对易:
    \begin{gather}
        \langle nlm\rvert\left[\frac{l(l+1)\hbar^2}{2mr^2}-\frac{Ze^2}{r},p_r\right]\lvert nlm\rangle=-i\hbar\langle nlm\rvert\left[\frac{l(l+1)\hbar^2}{mr^3}-\frac{Ze^2}{r^2}\right]\lvert nlm\rangle\\
        \Longrightarrow\langle nlm\rvert\frac{Ze^2}{r^3}\lvert nlm\rangle=\frac{m}{l(l+1)\hbar^2}\langle nlm\rvert\frac{(Ze^2)^2}{r^2}\lvert nlm\rangle.
    \end{gather}
    由课本式 (5.3.9) 和式 (3.7.53),
    \begin{align}
        \notag\langle nlm\rvert\frac{1}{r}\frac{\partial V_c}{\partial r}\lvert nlm\rangle=&\langle nlm\rvert\frac{Ze^2}{r^3}\lvert nlm\rangle=\frac{m}{l(l+1)\hbar^2}\frac{4n}{l+1/2}(E_n^{(0)})^2=-\frac{2m^2c^2Z^2\alpha^2}{nl(l+1)(l+1/2)\hbar^2}E_n^{(0)}\\
        =&\frac{m^3c^4Z^4\alpha^4}{n^3l(l+1)(l+1/2)\hbar^2}.
    \end{align}
\end{sol}

\begin{prob}[课本习题 5.17]
    \begin{itemize}
        \item[(a)] 一个刚性转子处于一个垂直于其转轴的磁场中, 假定它的哈密顿量具有如下形式 (Merzbacher 1970, 习题 17-1)
        \[
            A\bm{L}^2+BL_z+CL_y
        \]
        如果磁场的二次项被忽略的话. 假设 $B\gg C$, 将微扰论用到最低非零级以获得近似的能量本征值.
        \item[(b)] 考虑一个单电子 (如, 碱金属) 原子的矩阵元
        \begin{gather*}
            \langle n'l'm_l'm_s'\rvert(3z^2-r^2)\lvert nlm_lm_s\rangle,\\
            \langle n'l'm_l'm_s'\rvert xy\lvert nlm_lm_s\rangle.
        \end{gather*}
        对 $\Delta l$, $\Delta m_l$ 和 $\Delta m_s$ 写出选择定则. 证明你的答案是合理的.
    \end{itemize}
\end{prob}
\begin{sol}
    \begin{itemize}
        \item[(a)] 由于 $B\gg C$, 以该哈密顿量的前两项为无微扰哈密顿量
        \begin{align}
            H_0=A\bm{L}^2+BL_z,
        \end{align}
        以末项为微扰
        \begin{align}
            V=CL_y=C\frac{L_+-L_-}{2i}.
        \end{align}
        无微扰时的能量本征态为 $\lvert lm\rangle$, 对应的能量本征值为 $E_{lm}^{(0)}=A\hbar^2l(l+1)+B\hbar m$.
        利用非简并定态微扰论, 一阶能量本征值修正为
        \begin{align}
            \Delta_{lm}^{(1)}=\langle lm\rvert V\lvert lm\rangle=\frac{C}{2i}\langle lm\rvert L_+-L_-\lvert lm\rangle=0,
        \end{align}
        二阶能量本征值修正为
        \begin{align}
            \notag\Delta_{lm}^{(2)}=&\sum_{l'm'}\frac{\abs{\langle lm\rvert V\lvert l'm'\rangle}^2}{E_{lm}^{(0)}-E_{l'm'}^{(0)}}\\
            \notag=&\frac{C^2}{4}\sum_{l'm'}\frac{\abs{\langle lm\rvert L_+-L_-\lvert l'm'\rangle}^2}{E_{lm}^{(0)}-E_{l'm'}^{(0)}}\\
            \notag=&\frac{C^2\hbar^2}{4}\sum_{l'm'}\frac{\abs{\sqrt{m'+1}\delta_{ll'}\delta_{m,m'+1}-\sqrt{m'}\delta_{ll'}\delta_{m,m'-1}}^2}{A\hbar^2l(l+1)+B\hbar m-A\hbar^2l'(l'+1)-B\hbar m'}\\
            \notag=&\frac{C^2\hbar m}{2B}.
        \end{align}
        故精确到最低非零级的近似能量本征值为
        \begin{align}
            E_{lm}^{(2)}=E_{lm}^{(0)}+\Delta_{lm}^{(1)}+\Delta_{lm}^{(2)}=A\hbar^2l(l+1)+B\hbar m+\frac{C^2\hbar m}{2B}.
        \end{align}
        \item[(b)] 这些矩阵元与电子自旋无关, 故不会引起自旋翻转, $\Delta m_s=0$.

        由课本习题 (3.32),
        \begin{align}
            \notag\langle n'l'm_l'm_s'\rvert(3z^2-r^2)\lvert nlm_lm_s\rangle=&\sqrt{\frac{16\pi}{5}}\langle n'l'm_l'm_s'\rvert r^2Y_2^0(\bm{r})\lvert nlm_lm_s\rangle\\
            =&\sqrt{\frac{16\pi}{5}}\langle l2;m_l0\vert l2;l'm_l'\rangle\frac{\langle n'l'\rVert Y^{(2)}(\bm{r})\lVert nl\rangle}{\sqrt{2l+1}}\delta_{m_sm_s'}.
        \end{align}
        考虑到 Clebsch-Gordan 系数 $\langle l2;m_l0\vert l2;l'm_l'\rangle$ 仅当 $m_l=m_l'$, $\abs{l-2}\leq l'\leq l+2$ 时不为零, 并且由于 $Y^{(2)}(\bm{r})$ 具有偶宇称, 故 $\langle n'l'\rvert Y^{(2)}(\bm{r})\lvert nl\rangle$ 仅当 $\Delta l=l-l'$ 为偶数时不为零.
        综上, 矩阵元 $\langle n'l'm_l'm_s'\rvert(3z^2-r^2)\lvert nlm_lm_s\rangle$ 对应的微扰引发的跃迁遵循选择定则: \uline{$\Delta l$ 为偶数且 $\abs{l-2}\leq l'\leq l+2$, $\Delta m_l=0$, $\Delta m_l=0$, $\Delta m_s=0$}.

        由课本习题 (3.32) 并利用 Wigner-Eckart 定理,
        \begin{align}
            \notag\langle n'l'm_l'm_s'\rvert xy\lvert n'l'm_l'm_s'\rangle=&i\sqrt{\frac{2\pi}{15}}\langle n'l'm_l'm_s'\rvert r^2[Y_2^{-2}(\bm{r})+Y_2^{+2}(\bm{r})]\lvert nlm_lm_s\rangle\\
            =&i\sqrt{\frac{2\pi}{15}}\left[\langle l2;m_l,-2\vert l2;l'm_l'\rangle+\langle l2;m_l2\vert l2;l'm_l'\rangle\right]\frac{\langle n'l'\rvert Y^{(2)}\lvert nl\rangle}{\sqrt{2l+1}}\delta_{m_sm_s'}.
        \end{align}
        考虑到 Clebsch-Gordan 系数 $\langle l2;m_l,-2\vert l2;l'm_l'\rangle$ 仅当 $m_l-2m_l'$, $\abs{l-2}\leq l'\leq l+2$ 时不为零, $\langle l2;m_l2\vert l2;l'm_l'\rangle$ 仅当 $m_l+2=m_l'$, $\abs{l-2}\leq l'\leq l+2$ 时不为零, 并且由于 $Y^{(2)}(\bm{r})$ 具有偶宇称, 故 $\langle n'l'\rvert Y^{(2)}(\bm{r})\lvert nl\rangle$ 仅当 $\Delta l=l-l'$ 为偶数时不为零.
        综上, 矩阵元 $\langle n'l'm_l'm_s'\rvert xy\lvert nlm_lm_s\rangle$ 对应的微扰引发的跃迁遵循选择定则: \uline{$\Delta l$ 为偶数且 $\abs{l-2}\leq l'\leq l+2$, $\Delta m_l=\pm 2$, $\Delta m_s=0$}.
    \end{itemize}
\end{sol}

\begin{prob}[课本习题 5.19]
    (Merzbacher 1970, 448 页, 习题 11.) 对氦 (He) 的波函数, 使用
    \[
        \psi(\bm{x}_1,\bm{x}_2)=(Z_{\text{有效}}^3/\pi a_0^3)\exp\left[-\frac{Z_{\text{有效}}(r_1+r_2)}{a_0}\right]
    \]
    其中 $Z_{\text{有效}}=2-5/16$, 它是使用变分法得到的. 抗磁磁化率的测量值是 \SI{1.88e-6}{cm^3/mole}.

    如果系统处在一个由矢势 $\bm{A}=(1/2)\bm{B}\times\bm{r}$ 描写的一个均匀磁场中, 使用对于在磁场中原子电子的哈密顿量, 确定一个零角动量态的到 $B^2$ 量级的能量变化.

    用 $E=-(1/2)\chi B^2$ 定义原子的抗磁磁化率 $\chi$, 计算基态氦原子的 $\chi$, 并将其与测量值比较.
\end{prob}
\begin{sol}
    磁场带来的 $B^2$ 量级的微扰为
    \begin{align}
        V=\frac{e^2}{8m_ec^2}B^2[(x_1^2+y_1^2)+(x_2^2+y_2^2)].
    \end{align}
    该微扰带来基态的能量变化为
    \begin{align}
        \notag\Delta=&\int\psi^*(\bm{x}_1,\bm{x}_2)V\psi(\bm{x}_1,\bm{x}_2)\,\mathrm{d}^3\bm{x}_1\mathrm{d}^3\bm{x}_2\\
        \notag=&\frac{e^2}{8m_ec^2}B^2\frac{Z_{\text{有效}}^3}{\pi a_0^3}\left\{\frac{2}{3}4\pi\int_0^{\infty}r_1^2\exp\left(-\frac{2Z_{\text{有效}}r_1}{a_0}\right)r_1^2\,\mathrm{d}r_1+\frac{2}{3}4\pi\int_0^{\infty}r_2^2\exp\left(-\frac{2Z_{\text{有效}}r_2}{a_0}\right)r_2^2\,\mathrm{d}r_2\right\}\\
        =&\frac{e^2B^2a_0^2}{2Z_{\text{有效}}^2m_ec^2}.
    \end{align}
    其中利用了 $\langle x^2\rangle=\langle y^2\rangle=\langle z^2\rangle=\frac{1}{3}\langle r^2\rangle$, 以及积分 $\int_0^{\infty}e^{-\alpha r}r^n\,\mathrm{d}r=\frac{n!}{\alpha^{n+1}}$. 基态氦原子的抗磁磁化率为
    \begin{align}
        \chi=-\frac{2\Delta}{B^2}=-\frac{e^2a_0^2}{Z_{\text{有效}}^2m_ec^2}=-2.76\times 10^{-36}\text{m}^3/\text{atom}=1.67\times 10^{-6}\text{cm}^3/\text{mol}.
    \end{align}
    (注意此处的 $e$ 为高斯单位制的元电荷. 最前面的系数 $2$ 是因为氦原子中含有两个电子, 他们在外加磁场下均感应出磁矩从而均产生 $\Delta$ 的能量变化.) 该理论计算结果与测量值吻合.
\end{sol}

\begin{prob}[课本习题 5.20]
    使用试探波函数
    \[
        \langle\bm{x}\vert\tilde{0}\rangle=e^{-\beta\abs{x}}
    \]
    其中 $\beta$ 作为变分参数, 估算一维简谐振子的基态能量. 可以使用
    \[
        \int_0^{\infty}e^{-\alpha r}r^n\,\mathrm{d}r=\frac{n!}{\alpha^{n+1}}.
    \]
\end{prob}
\begin{sol}
    一维简谐振子的哈密顿量为
    \begin{align}
        H=-\frac{p^2}{2m}+\frac{m\omega^2x^2}{2}=-\frac{\hbar^2}{2m}\frac{\mathrm{d}^2}{\mathrm{d}x^2}+\frac{m\omega^2x^2}{2}.
    \end{align}
    试探波函数对应的能量为
    \begin{align}
        \bar{H}=\frac{\langle\tilde{0}\rvert H\lvert\tilde{0}\rangle}{\langle\tilde{0}\vert\tilde{0}\rangle}=\frac{\int_{-\infty}^{+\infty}e^{-\beta\abs{x}}\left(-\frac{\hbar^2}{2m}\frac{\mathrm{d}^2}{\mathrm{d}x^2}+\frac{m\omega^2x^2}{2}\right)e^{-\beta\abs{x}}\,\mathrm{d}x}{\int_{-\infty}^{+\infty}e^{-2\beta\abs{x}}\,\mathrm{d}x},
    \end{align}
    其中
    \begin{align}
        \notag-\frac{\hbar^2}{2m}\int_{-\infty}^{+\infty}e^{-\beta\abs{x}}\frac{\mathrm{d}^2}{\mathrm{d}x^2}e^{-\beta\abs{x}}\,\mathrm{d}x=&-\frac{\hbar^2}{2m}\lim_{\epsilon\rightarrow 0}\left[2\int_{\epsilon}^{\infty}e^{-\beta x}\frac{\mathrm{d}^2}{\mathrm{d}x^2}e^{-\beta x}+\int_{-\epsilon}^{\epsilon}e^{-\beta\abs{x}}\frac{\mathrm{d}^2}{\mathrm{d}x^2}e^{-\beta\abs{x}}\,\mathrm{d}x\right]\\
        \notag=&-\frac{\hbar^2}{2m}\left[2\beta^2\int_{\epsilon}^{\infty}e^{-2\beta x}\,\mathrm{d}x+e^{-\beta\cdot 0}\int_{-\epsilon}^{\epsilon}\mathrm{d}\left(\frac{\mathrm{d}e^{-\beta\abs{x}}}{\mathrm{d}x}\right)\right]\\
        \notag=&-\frac{\hbar^2}{2m}\left[\beta+\left(\frac{\mathrm{d}e^{-\beta\abs{x}}}{\mathrm{d}x}\right)_{-\epsilon}^{\epsilon}\right]\\
        =&\frac{\hbar^2\beta}{2m},
    \end{align}
    \begin{align}
        \frac{m\omega^2}{2}\int_{-\infty}^{+\infty}x^2e^{-2\beta\abs{x}}\,\mathrm{d}x=&m\omega^2\int_0^{+\infty}x^2e^{-2\beta x}\,\mathrm{d}x=\frac{m\omega^2}{4\beta^3},
    \end{align}
    \begin{align}
        \int_{-\infty}^{+\infty}e^{-2\beta\abs{x}}=\frac{1}{\beta},
    \end{align}
    故
    \begin{align}
        \bar{H}=\frac{\hbar^2\beta^2}{2m}+\frac{m\omega^2}{4\beta^2}.
    \end{align}
    要使估算的基态能量最小, 则需参数 $\beta$ 满足
    \begin{gather}
        \frac{\partial\bar{H}}{\partial\beta}=\frac{\hbar^2\beta}{m}-\frac{m\omega^2}{2\beta^3}=0,\\
        \Longrightarrow\beta^2=\frac{m\omega}{\sqrt{2}\hbar},
    \end{gather}
    此时估算的基态能量为
    \begin{align}
        \bar{H}=\frac{\hbar\omega}{\sqrt{2}}.
    \end{align}
\end{sol}
\end{document}