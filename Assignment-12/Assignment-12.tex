\documentclass{assignment}
\ProjectInfos{高等量子力学}{PHYS5001P}{2022-2023 学年第一学期}{第十二次作业}{截止时间: 2022 年 12 月 12 日 (周一)}{陈稼霖}[https://github.com/Chen-Jialin]{SA21038052}

\begin{document}
\begin{prob}[课本习题 5.41]
    求基态氢原子具有特定动量 $\bm{p}'$ 的概率 $\abs{\phi(\bm{p}')}^2\,\mathrm{d}^3p'$. (这是一个很好的三维傅里叶变换的练习. 为进行角积分, 选择 $z$ 轴沿 $\bm{p}$ 方向.)
\end{prob}
\begin{sol}
    在位置表象中, 氢原子电子基态波函数为
    \begin{align}
        \psi(\bm{x})=\frac{1}{\sqrt{\pi}}\left(\frac{1}{a_0}\right)^{3/2}e^{-r/a_0}.
    \end{align}
    利用傅里叶变换将其变换至动量表象中得
    \begin{align}
        \notag\phi(\bm{p}')=&\frac{1}{(2\pi\hbar)^{3/2}}\int\mathrm{d}^3x\,e^{-i\bm{p}\cdot\bm{x}/\hbar}\psi(\bm{x})\\
        \notag=&\frac{1}{(2\pi\hbar a_0)^{3/2}}\frac{1}{\sqrt{\pi}}\int\mathrm{d}^3x\,e^{-i\bm{p}\cdot\bm{x}/\hbar}e^{-r/a_0}\\
        \notag=&\frac{1}{(2\pi\hbar a_0)^{3/2}}2\sqrt{\pi}\int_0^{\infty}e^{-r/a_0}r^2\,\mathrm{d}r\int_0^{\pi}e^{-ipr\cos\theta/\hbar}\sin\theta\,\mathrm{d}\theta\\
        \notag=&\frac{1}{(2\pi\hbar a_0)^{3/2}}2\sqrt{\pi}\int_0^{\infty}e^{-r/a_0}r^2\frac{e^{ipr/\hbar}-e^{-ipr/\hbar}}{ipr/\hbar}\,\mathrm{d}r\\
        \notag=&\frac{1}{(2\pi\hbar a_0)^{3/2}}2\sqrt{\pi}\int_0^{\infty}e^{-r/a_0}r^2\,\mathrm{d}r\int_{-1}^1e^{-ipr\cos\theta/\hbar}\,\mathrm{d}(\cos\theta)\\
        \notag=&\frac{1}{(2\pi\hbar a_0)^{3/2}}2\sqrt{\pi}\frac{1}{ip/\hbar}\int_0^{\infty}[e^{-(1/a_0-ip/\hbar)r}-e^{-(1/a_0+ip/\hbar)r}]r\,\mathrm{d}r\\
        \notag&(\text{利用 }\int_0^{\infty}e^{-x}x\,\mathrm{d}x=1)\\
        \notag=&\frac{1}{(2\pi\hbar a_0)^{3/2}}2\sqrt{\pi}\frac{1}{ip/\hbar}\left[\frac{1}{(1/a_0-ip/\hbar)^2}-\frac{1}{(1/a_0+ip/\hbar)^2}\right]\\
        =&\frac{1}{(2\pi\hbar a_0)^{3/2}}2\sqrt{\pi}\frac{4\hbar^4a_0^3}{(\hbar^2+a_0^2p^2)^2}.
    \end{align}
    故基态氢原子具有特定动量的概率为
    \begin{align}
        \abs{\phi(\bm{p}')}^2\,\mathrm{d}^3p'=\frac{8\hbar^5a_0^3}{\pi^2(\hbar^2+a_0^2p^2)^4}\,\mathrm{d}^3p'.
    \end{align}
\end{sol}

\begin{prob}[课本习题 5.42]
    求氢原子的 $\tau(2p\rightarrow 1s)$ 的表达式. 证明它等于 \SI{1.6e-9}{s}.
\end{prob}
\begin{pf}
    氢原子自 2p 态衰变至 1s 态的衰变时间为
    \begin{align}
        \tau(2p\rightarrow 1s)=\frac{1}{w_{2p\rightarrow 1s}}.
    \end{align}
    该过程中电子由 2p 态跃迁至 1s 态的同时放射出 1 个的光子, 该光子的能量等于 2p 态与 1s 态的能量差 ($\hbar\omega=E_{2p}-E_{1s}$), 这一自发辐射过程是 1 个光子将电子从 1s 态激发至 2p 态的受激吸收过程的逆过程, 两者的跃迁速率的表达式是类似的, 由课本式 (5.8.8),
    \begin{align}
        w_{2p\rightarrow 1s}=\frac{2\pi}{\hbar}\frac{e^2}{m_e^2c^2}\abs{A_0}^2\abs{\langle 1s\rvert e^{i(\omega/c)(\hat{\bm{n}}\cdot\bm{x})}\hat{\bm{\varepsilon}}\cdot\bm{p}\lvert 2p\rangle}^2\rho(\hbar\omega=E_{2p}-E_{1s}).
    \end{align}
    下面就来分别求解上式中的各部分: $\abs{A_0}^2$, $\abs{\langle 1s\rvert e^{i(\omega/c)(\hat{\bm{n}}\cdot\bm{x})}\hat{\bm{\varepsilon}}\cdot\bm{p}\lvert 2p\rangle}^2$, $\rho(\hbar\omega=E_{2p}-E_{1s})$.
    \begin{itemize}
        \item[(a)] \uline{$\abs{A_0}^2$}: 当采用边界条件, 边长为 $L$ 的正方体箱体内给定偏振方向 $\hat{\bm{\varepsilon}}$ 和频率 $\omega$ 的单色光场的矢势可表为
        \begin{align}
            \bm{A}=2A_0\hat{\bm{\varepsilon}}\cos\left(\bm{k}\cdot\bm{x}-\omega t\right),
        \end{align}
        其中 $\bm{k}=\frac{\omega}{c}\hat{\bm{k}}=\frac{2\pi}{L}(n_x,n_y,n_z)$, $n_x,n_y,n_z\in\mathbb{Z}$.
        该箱体内电场和磁场可分别表为
        \begin{align}
            \bm{E}=&-\frac{1}{c}\frac{\partial\bm{A}}{\partial t}=2A_0k\hat{\bm{\varepsilon}}\sin(\bm{k}\cdot\bm{x}-\omega t),\\
            \bm{B}=&\nabla\times\bm{A}=2A_0(\hat{\bm{\varepsilon}}\times\bm{k})\sin(\bm{k}\cdot\bm{x}-\omega t).
        \end{align}
        该箱体内光场的能量密度 (注意高斯单位制) 为
        \begin{align}
            u=\frac{\bm{E}^2+\bm{B}^2}{8\pi}=\frac{k^2A_0^2}{\pi}\sin^2(\bm{k}\cdot\bm{x}-\omega t)=\frac{k^2A_0^2}{2\pi}[1-\cos 2(\bm{k}\cdot\bm{x}-\omega t)].
        \end{align}
        当仅有 1 个光子时, 该箱体内光场的总能量为
        \begin{align}
            \hbar\omega=\int_0^L\int_0^L\int_0^Lu\,\mathrm{d}x\mathrm{d}y\,\mathrm{d}z=\frac{k^2A_0^2}{2\pi}L^3.
        \end{align}
        故
        \begin{align}
            A_0^2=\frac{2\pi\hbar c^2}{\omega L^3}.
        \end{align}
        \item[(2)] \uline{$\abs{\langle 1s\rvert e^{i(\omega/c)(\hat{\bm{n}}\cdot\bm{x})}\hat{\bm{\varepsilon}}\cdot\bm{p}\lvert 2p\rangle}^2$}: 由于跃迁放出的光子能量 $\hbar\omega=E_{2p}-E_{1s}=$\SI{13.6}{eV}, 对应的光子波长为 $\lambda=$\SI{91.4}{nm}, 远大于原子的尺度 (\AA 量级), 故可进行偶极近似, 即 $e^{i(\omega/c)\hat{\bm{n}}\cdot\bm{x}}=1+i\frac{\omega}{c}\hat{\bm{n}}\cdot\bm{x}+\cdots\approx 1$, 从而
        \begin{align}
            \langle 1s\rvert e^{i(\omega/c)(\hat{\bm{n}}\cdot\bm{x})}\hat{\bm{\varepsilon}}\cdot\bm{p}\lvert 2p\rangle=\langle 1s\rvert\hat{\bm{\varepsilon}}\cdot\bm{p}\lvert 2p\rangle,
        \end{align}
        不妨取 $\hat{\bm{\varepsilon}}$ 为 $z$ 轴方向, 则
        \begin{align}
            \notag\langle 1s\rvert\hat{\bm{\varepsilon}}\cdot\bm{p}\lvert 2p\rangle=&\langle 1s\rvert p_z\lvert 2p\rangle=\frac{m}{i\hbar}\langle 1s\rvert[z,H_0]\lvert 2p\rangle=-im\omega\langle 1s\rvert z\lvert 2p\rangle\\
            =&-im\omega\sqrt{\frac{4\pi}{3}}\langle 1s\rvert rY_1^0\lvert 2p\rangle,
        \end{align}
        由于 $\langle 1s\rvert rY_1^0\lvert 2p,m\rangle$ 仅在 $0=0+m$, 即 $m=0$ 时不等于 $0$, 故
        \begin{align}
            \langle 1s\rvert\hat{\bm{\varepsilon}}\cdot\bm{p}\lvert 2p\rangle=-im\omega\frac{1}{\sqrt{3}}\langle 1s\rvert z\lvert 2p,m=0\rangle,
        \end{align}
        其中
        \begin{align}
            \langle 1s\rvert z\lvert 2p,m=0\rangle=&\int\left(\frac{1}{a_0}\right)^{3/2}2e^{-r/a_0}\frac{1}{\sqrt{4\pi}}\times z\times\left(\frac{1}{2a_0}\right)^{3/2}\frac{r}{\sqrt{3}a_0}e^{-r/2a_0}\sqrt{\frac{3}{4\pi}}\cos\theta\,\mathrm{d}^3x\\
            \notag=&\frac{1}{2^{5/2}\pi a_0^4}2\pi\int_0^{\infty}e^{-3r/2a_0}r^4\,\mathrm{d}r\int_0^{\pi}\cos^2\theta\sin\theta\,\mathrm{d}\theta\\
            \notag=&\frac{1}{2^{5/2}\pi a_0^4}2\pi\int_0^{\infty}e^{-3r/2a_0}r^4\,\mathrm{d}r\int_{-1}^1\cos^2\theta\,\mathrm{d}(\cos\theta)\\
            \notag=&\frac{1}{2^{5/2}\pi a_0^4}2\pi\frac{2}{3}\int_0^{\infty}e^{-3r/2a_0}r^4\,\mathrm{d}r\\
            \notag&(\text{利用 }\int_0^{\infty}e^{-x}x^4\,\mathrm{d}x=4!)\\
            \notag=&\frac{1}{2^{5/2}\pi a_0^4}2\pi\frac{2}{3}\left(\frac{2a_0}{3}\right)^54!\\
            =&\frac{2^{15/2}a_0}{3^5}.
        \end{align}
        故
        \begin{align}
            \abs{\langle 1s\rvert e^{i(\omega/c)(\hat{\bm{n}}\cdot\bm{x})}\hat{\bm{\varepsilon}}\cdot\bm{p}\lvert 2p\rangle}^2=\frac{m^2\omega^2}{3}\frac{2^{15}a_0^2}{3^{10}}.
        \end{align}
        \item[(3)] \uline{$\rho(\hbar\omega=E_{2p}-E_{1s})$}: 光子的态密度满足
        \begin{align}
            \notag\rho(\mathcal{E})\,\mathrm{d}\mathcal{E}=&2\frac{4\pi k^2\,\mathrm{d}k}{(2\pi/L)^3}\qquad(\text{前方系数 2 来源于每个给定波矢的光子均有 2 个正交的偏振方向})\\
            \notag&(\text{光子的能量 }\mathcal{E}=\hbar\omega=\hbar kc)\\
            =&\frac{L^3\omega^2\,\mathrm{d}\mathcal{E}}{\pi^2\hbar c^3},
        \end{align}
        \begin{align}
            \Longrightarrow\rho(\hbar\omega)=\frac{L^3\omega^2}{\pi^2\hbar c^3}.
        \end{align}
    \end{itemize}
    综合起来, 最后得到
    \begin{align}
        w_{2p\rightarrow 1s}=\frac{2\pi}{\hbar}\frac{e^2}{m_e^2c^2}\frac{2\pi\hbar c^2}{\omega L^3}\frac{m^2\omega^2}{3}\frac{2^{15}a_0^2}{3^{10}}\frac{L^3\omega^2}{\pi^2\hbar c^3}=\frac{2^{17}e^2a_0^2\omega^3}{3^{11}\hbar c^3},
    \end{align}
    代入 $\hbar\omega=E_{2p}-E_{1s}=\frac{e^2}{2a_0}\left(1-\frac{1}{2^2}\right)=\frac{3}{4}\times\SI{13.6}{eV}$, 有
    \begin{align}
        w_{2p\rightarrow 1s}=\frac{2^{17}e^2a_0^2}{3^{11}\hbar c^3}\left[\frac{1}{\hbar}\frac{e^2}{2a_0}\left(1-\frac{1}{2^2}\right)\right]^3=\frac{2^8e^8}{3^8\hbar^4c^3a_0}=\SI{619}{MHz},
    \end{align}
    从而衰变时间为
    \begin{align}
        \tau=\frac{1}{w_{2p\rightarrow 1s}}=\SI{1.62E-9}{s}.
    \end{align}
\end{pf}

\begin{prob}[课本习题 7.2]
    \begin{itemize}
        \item[(a)] $N$ 个自旋为 $1/2$ 的全同粒子受到一个一维简谐振子势的作用. 忽略粒子之间所有的相互作用, 基态的能量是什么? 费米能量是什么?
        \item[(b)] 如果忽略了粒子间的相互作用并假定 $N$ 非常大, 基态能量和费米能量又是什么?
    \end{itemize}
\end{prob}
\begin{sol}
    \begin{itemize}
        \item[(a)] 自旋为 $1/2$ 的粒子为费米子, 它们遵循泡利不相容原理, 即任意两个该种粒子不可占据相同的态. 忽略粒子之间的相互作用, 该系统基态的情形是, 这 $N$ 个粒子从一维简谐振子的基态能级开始向高能级依次填充, 且每个能级可以填充两个自旋相反的粒子. 假设一维简谐振子势为 $\frac{m\omega^2x^2}{2}$, 则由这 $N$ 个粒子组成的系统的基态的能量为
        \begin{align}
            E_0=\left\{\begin{array}{ll}
                \sum_{n=0}^{(N-3)/2}2\hbar\omega\left(n+\frac{1}{2}\right)+\hbar\omega\left(\frac{N-1}{2}+\frac{1}{2}\right)=\hbar\omega\frac{N^2+1}{4},&N\text{ 为奇数},\\
                \sum_{n=0}^{N/2-1}2\hbar\omega\left(n+\frac{1}{2}\right)=\hbar\omega\frac{N^2}{4},&N\text{ 为偶数}.
            \end{array}\right.
        \end{align}
        费米能量即基态填充的最高能级的能量, 该系统的费米能量为
        \begin{align}
            E_f=\left\{\begin{array}{ll}
                \hbar\omega\frac{N}{2},&N\text{ 为奇数},\\
                \hbar\omega\frac{N-1}{2},&N\text{ 为偶数}.
            \end{array}\right.
        \end{align}
        \item[(b)] 当 $N$ 非常大时, 基态能量可近似为
        \begin{align}
            E_0=\hbar\omega\frac{N^2}{4}.
        \end{align}
        费米能量可近似为
        \begin{align}
            E_f=\hbar\omega\frac{N}{2}.
        \end{align}
    \end{itemize}
\end{sol}

\begin{prob}[课本习题 7.3]
    两个无轨道角动量的 (即两个粒子都在 $s$ 态) 自旋为 $1$ 的非全同粒子显然能够形成 $j=0$, $j=1$ 和 $j=2$ 的态. 然而, 假定这两个粒子是全同的, 得到什么限制?
\end{prob}
\begin{pf}
    
\end{pf}

\begin{prob}[课本习题 7.4]
    讨论如果氦原子的电子是无自旋的玻色子, 氦原子的能级会发生什么变化, 尽你所能做定量的讨论.
\end{prob}
\begin{sol}
    
\end{sol}
\end{document}