\documentclass{assignment}
\ProjectInfos{高等量子力学}{PHYS5001P}{2022-2023 学年第一学期}{第十三次作业}{截止时间: 2022 年 12 月 19 日 (周一)}{陈稼霖}[https://github.com/Chen-Jialin]{SA21038052}

\begin{document}
\begin{prob}[课本习题 7.7]
    证明一个算符 $a$ 和它的共轭算符遵从反对易关系 $\{a,a^{\dagger}\}=aa^{\dagger}+a^{\dagger}a=1$, 则算符 $N=a^{\dagger}a$ 具有本征值为 $0$ 和 $1$ 的本征态.
\end{prob}
\begin{pf}
    假设 $\lvert\alpha\rangle$ 为算符 $N=a^{\dagger}a$ 的一个本征态, 对应的本征值为 $\alpha$:
    \begin{align}
        N\lvert\alpha\rangle=\alpha\lvert\alpha\rangle.
    \end{align}
    首先证明算符 $N$ 的本征值非负: 一方面,
    \begin{align}
        \langle\alpha\rvert N\lvert\alpha\rangle=\langle\alpha\rvert\alpha\lvert\alpha\alpha\rangle=\alpha,
    \end{align}
    另一方面,
    \begin{align}
        \langle\alpha\rvert N\lvert\alpha\rangle=\langle\alpha\rvert a^{\dagger}a=[a\lvert\alpha\rangle]^{\dagger}a\lvert\alpha\rangle=\abs{a\lvert\alpha\rangle}^2,
    \end{align}
    故
    \begin{align}
        \alpha=\abs{a\lvert\alpha\rangle}^2\geq 0.
    \end{align}

    其次考虑 $a^{\dagger}\lvert\alpha\rangle$, 将算符 $N$ 作用于其上并利用反对易关系 $\{a^{\dagger},a\}=a^{\dagger}a+aa^{\dagger}$ 有
    \begin{align}
        N\lvert\alpha\rangle=a^{\dagger}aa^{\dagger}\lvert\alpha\rangle=a^{\dagger}(1-a^{\dagger}a)\lvert\alpha\rangle=a^{\dagger}(1-\alpha)\lvert\alpha\rangle,
    \end{align}
    故 $a^{\dagger}\lvert\alpha\rangle$ 亦为算符 $N$ 的本征态, 对应的本征值为 $1-\alpha$. 由于算符 $N$ 的本征值非负, 故 $1-\alpha\geq 0\Longrightarrow\alpha\leq 1$.

    因此, 算符 $N=a^{\dagger}a$ 具有本征值为 $0$ 和 $1$ 的本征态.

    (上述推导存疑: 如何仿照波色子的案例利用本征值非负的条件来截断本征态 / 值之间的递推关系从而将本征值限定为整数, 暂时没想到.)
\end{pf}

\begin{prob}
    若 $N$ 个全同粒子体系的哈密顿算符为:
    \[
        \hat{H}=\sum_{k=1}^NT(x_k)+\frac{1}{2}\sum_{k\neq l=1}^NV(x_k,x_l),
    \]
    写出对应的二次量子化形式并指出产生湮灭算符的基本对易或反对易关系.
\end{prob}
\begin{sol}
    以坐标表象的基矢为基, 哈密顿算符的二次量子化形式为
    \begin{align}
        \hat{H}=\int\mathrm{d}x'\int\mathrm{d}x''\,\langle x'\rvert T\lvert x''\rangle a_{x'}^{\dagger}a_{x''}+\int\mathrm{d}x'\int\mathrm{d}x''\int\mathrm{d}x'''\int\mathrm{d}x''''\,\langle x_i,x_j\rvert V\lvert x_k,x_l\rangle a_{x'}^{\dagger}a_{x''}^{\dagger}a_{x''''}a_{x'''},
    \end{align}
    (注意 $a_{x''''}$ 和 $a_{x'''}$ 的顺序不可调换!)\\
    其中
    \begin{align}
        \langle x'\rvert T\lvert x''\rangle=&T(x')\delta(x'-x''),\\
        \langle x',x''\rvert V\lvert x''',x''''\rangle=&V(x'-x'')\delta(x'-x''')\delta(x''-x'''')
    \end{align}
    故哈密顿算符的二次量子化形式为
    \begin{align}
        \hat{H}=\int\mathrm{d}x'\,T(x')a_{x'}^{\dagger}a_{x'}+\int\mathrm{d}x'\int\mathrm{d}x''\,V(x'-x'')a_{x'}^{\dagger}a_{x''}^{\dagger}a_{x''}a_{x'},
    \end{align}

    产生湮灭算符的基本对易或反对易关系: 对费米子,
    \begin{align}
        [a_{x'}^{\dagger},a_{x''}^{\dagger}]=[a_{x'},a_{x''}]=0,\quad[a_{x'},a_{x''}^{\dagger}]=\delta(x'-x'');
    \end{align}
    对玻色子
    \begin{align}
        \{a_{x'}^{\dagger},a_{x''}^{\dagger}\}=\{a_{x'},a_{x''}\}=0,\quad\{a_{x'},a_{x''}^{\dagger}\}=\delta(x'-x'').
    \end{align}
\end{sol}

\begin{prob}
    $N$ 个自旋为 $1/2$ 的无相互作用的全同粒子受外势 $V_0\delta(x)$ 作用, 求以平面波为基时的哈密顿算符的二次量子化形式, 指出相应产生与湮灭算符的基本关系, 并利用微扰法给出基态能量.
\end{prob}
\begin{sol}
    以平面波为基, 哈密顿算符的二次量子化形式为
    \begin{align}
        H=\sum_{k_1\lambda_1,k_2\lambda_2}\langle k_1\lambda_1\rvert \frac{p^2}{2m}\lvert k_2\lambda_2\rangle a_{k_1\lambda_1}^{\dagger}a_{k_2\lambda_2}+\sum_{k_1\lambda_1,k_2\lambda_2}\langle k_1\lambda_1\rvert V_0\delta(x)\lvert k_2\lambda_2\rangle a_{k_1\lambda_1}^{\dagger}a_{k_2\lambda_2},
    \end{align}
    (注意 $a_{k_4\lambda_4}$ 和 $a_{k_3\lambda_3}$ 的顺序不可调换!)\\
    其中 $k_i$ 代表平面波矢, $\lambda_i$ 代表自旋,
    \begin{align}
        \langle k_1\lambda_1\rvert\frac{p^2}{2m}\lvert k_2\lambda_2\rangle=&\frac{1}{V}\int\mathrm{d}^3x\,e^{-ik_1\cdot x}\eta_{\lambda_1}^{\dagger}\frac{-\hbar^2\nabla^2}{2m}e^{ik_2\cdot x}\eta_{\lambda_2}=\frac{\hbar^2k_2^2}{2mV}\delta_{\lambda_1\lambda_2}\int\mathrm{d}^3xe^{i(k_2-k_1)\cdot x}=\frac{\hbar^2k_2^2}{2m}\delta_{\lambda_1\lambda_1}\delta_{k_1k_2},\\
        \langle k_1\lambda_1\rvert V_0\delta(x)\lvert k_2\lambda_2\rangle=&\frac{1}{V}\int\mathrm{d}^3x\,e^{-ik_1\cdot x}\eta_{\lambda_1}^{\dagger}V_0\delta(x)e^{ik_2\cdot x}\eta_{\lambda_2}=V_0\delta_{\lambda_1\lambda_2},
    \end{align}
    其中 $V=L^3$ 为用于平面波箱归一化的箱体积, 故哈密顿算符的二次量子化形式为
    \begin{align}
        H=\sum_{k\lambda}\frac{\hbar^2k^2}{2m}a_{k\lambda}^{\dagger}a_{k\lambda}+\sum_{k_1k_2\lambda}V_0a_{k_1\lambda}^{\dagger}a_{k_2\lambda}.
    \end{align}

    自旋为 $1/2$ 的粒子为费米子, 其产生与湮灭算符的基本关系为
    \begin{align}
        \{a_{k_1\lambda_1}^{\dagger},a_{k_2\lambda_2}^{\dagger}\}=\{a_{k_1\lambda_1},a_{k_2\lambda_2}\}=0,\quad\{a_{k_1\lambda_1},a_{k_2\lambda_2}^{\dagger}\}=\delta_{k_1k_2}\delta_{\lambda_1\lambda_2}.
    \end{align}

    以 $\sum_{k\lambda}\frac{\hbar^2k^2}{2m}a_{k\lambda}^{\dagger}a_{k\lambda}$ 为无微扰哈密顿量, $\sum_{k_1k_2\lambda}V_0a_{k_1\lambda}^{\dagger}a_{k_2\lambda}$ 为微扰. 无微扰下, 系统基态即为自由费米子气体的基态, $N$ 个粒子从平面波的基态能级开始向高能级依次填充, 并且每个能级可以填充两个自旋相反的粒子, 自由费米子气体的态密度满足
    \begin{gather}
        D(E)\,\mathrm{d}E=2\times\frac{1}{(2\pi /L)^3}4\pi k^2\,\mathrm{d}k=\frac{2^{1/2}Vm^{3/2}E^{1/2}}{\pi^2\hbar^3}\mathrm{d}E\\
        \Longrightarrow D(E)=\frac{2^{1/2}Vm^{3/2}E^{1/2}}{\pi^2\hbar^3}.
    \end{gather}
    故此时填充至费米能量
    \begin{gather}
        \int_0^{E_f}D(E)\,\mathrm{d}E=\int_0^{E_f}\frac{2^{1/2}Vm^{3/2}E^{1/2}}{\pi^2\hbar^3}\mathrm{d}E=\frac{2^{3/2}Vm^{3/2}E_f^{3/2}}{3\pi^2\hbar^3}=N,\\
        \Longrightarrow E_f=\frac{3^{2/3}\pi^{4/3}N^{2/3}\hbar^2}{2mL^2}.
    \end{gather}
    对应费米波矢
    \begin{align}
        k_f=\frac{3^{1/3}\pi^{2/3}N^{1/3}}{L}.
    \end{align}
    故无微扰下的基态能量为
    \begin{align}
        \notag E_0^{(0)}=&\int_0^{E_f}\mathrm{d}E\,D(E)\times E=\int_0^{E_f}\mathrm{d}E\,\frac{2^{1/2}Vm^{3/2}E^{1/2}}{\pi^2\hbar^3}\times E=\frac{2^{1/2}Vm^{3/2}}{\pi^2\hbar^3}\frac{2}{5}E_f^{5/2}\\
        =&\frac{2\times 3^{5/2}\pi^{4/3}\hbar^2N^{5/3}}{5mV^{2/3}}.
    \end{align}
    利用微扰理论, 一阶微扰为
    \begin{align}
        \Delta_0^{(1)}=\langle\text{base state}\rvert V_0a_{k_1\lambda}^{\dagger}a_{k_2\lambda}\lvert\text{base state}\rangle=\int_0^{E_f}\mathrm{d}E\,D(E)\times V_0=NV_0,
    \end{align}
    其中基态 $\lvert\text{base state}\rangle$ 为波矢小于费米波矢 ($k<k_f$) 的每个能级均被有两个自旋相反的粒子填充, 波矢大于费米波矢 ($k>k_f$) 的每个能级的粒子数均为零. 综上, 由微扰理论近似到一阶的基态能量为
    \begin{align}
        E_0^{(1)}=E_0^{(0)}+\Delta_0^{(1)}=\frac{2\times 3^{5/2}\pi^{4/3}\hbar^2N^{5/3}}{5mV^{2/3}}+NV_0.
    \end{align}
    (\textit{注意: 以上的推导中, 我们实际上假定该系统的态是连续分布的, 故可以使用态密度的概念, 这适用于大 $N$ (即 $k_f\gg\frac{2\pi}{L}$) 的情形. 对于较小的 $N$ (即 $k_f\sim\frac{2\pi}{L}$) 的情形, 粒子仅占据较少个数的分离的态, 故应当将 $\int\mathrm{d}E\,D(E)$ 换成 $\sum_{k}$ 更为准确, 其余推导过程仍类似, 对于这一情形, 三维的推导过程较为复杂, 难以给出解析形式, 这里我们仅给出一维的答案 (近似到一阶):
    \begin{align}
        \notag E_0^{(1)}=&\sum_{n=1}^{\lceil N/4\rceil}4\times\frac{\left(\hbar\frac{2\pi n}{L}\right)^2}{2m}+\frac{\left(\hbar\frac{2\pi\lceil N/4\rceil}{L}\right)^2}{2m}\times[(N+2)\mod 4]+NV_0\\
        =&\frac{8\pi^2\hbar^2}{mL^2}\frac{\lceil N/4\rceil(\lceil N/4\rceil+1)(2\lceil N/4\rceil+1)}{6}+\frac{2\pi^2\hbar^2}{mL^2}\lceil N/4\rceil\times[(N+2)\mod 4]+NV_0,
    \end{align}
    其中 $\lceil\cdot\rceil$ 代表高斯符号, 即 $\lceil x\rceil$ 为不大于 $x$ 的最大整数. 具体的推导过程略去.})
\end{sol}

\begin{prob}[课本习题 8.10]
    证明狄拉克方程的连续性方程 (8.2.11).
\end{prob}
\begin{pf}
    自由粒子的狄拉克方程为
    \begin{align}
        i\hbar\frac{\partial\Psi}{\partial t}=(c\bm{\alpha}\cdot\bm{p}+\beta mc^2)\Psi.
    \end{align}
    自由粒子的狄拉克方程的连续性方程为
    \begin{align}
        \frac{\partial\rho}{\partial t}+\nabla\cdot\bm{j}=0,
    \end{align}
    其中概率密度 $\rho=\Psi^{\dagger}\Psi$, 概率密度流 $\bm{j}=c\Psi^{\dagger}\bm{\alpha}\Psi$.
    \begin{align}
        \notag\frac{\partial\rho}{\partial t}=&\frac{\partial\Psi^{\dagger}}{\partial t}\Psi+\Psi^{\dagger}\frac{\partial\Psi}{\partial t}\\
        \notag=&\frac{1}{i\hbar}\left\{-[(c\bm{\alpha}\cdot\bm{p}+\beta mc^2)\Psi]^{\dagger}\Psi+\Psi^{\dagger}[c\bm{\alpha}\cdot\bm{p}+\beta mc^2]\Psi\right\}\\
        =&c\left\{-\nabla\Psi^{\dagger}\cdot\bm{\alpha}\Psi-\Psi^{\dagger}\bm{\alpha}\cdot\nabla\Psi\right\},\\
        \notag\nabla\cdot\bm{j}=&c(\nabla\Psi^{\dagger}\cdot\bm{\alpha}\Psi+\Psi^{\dagger}\bm{\alpha}\cdot\nabla\Psi).
    \end{align}
    故
    \begin{align}
        \frac{\partial\rho}{\partial t}+\nabla\cdot\bm{j}=0.
    \end{align}
    至此证明了狄拉克方程的连续性方程.
\end{pf}

\begin{prob}[课本习题 8.11]
    求自由粒子狄拉克方程 (8.2.20) 的本征值.
\end{prob}
\begin{sol}
    自由粒子狄拉克方程 (8.2.20):
    \begin{align}
        \begin{bmatrix}
            m&0&p&0\\
            0&m&0&-p\\
            p&0&-m&0\\
            0&-p&0&-m
        \end{bmatrix}\begin{bmatrix}
            u_1\\
            u_2\\
            u_3\\
            u_4
        \end{bmatrix}=E\begin{bmatrix}
            u_1\\
            u_2\\
            u_3\\
            u_4
        \end{bmatrix}.
    \end{align}
    利用
    \begin{align}
        \abs{H-EI}=\begin{vmatrix}
            m-E&0&p&0\\
            0&m-E&0&-p\\
            p&0&-m-E&0\\
            0&-p&0&-m-E
        \end{vmatrix}=[(m-E)(m+E)+p^2]^2=0
    \end{align}
    解得哈密顿量的本征值为
    \begin{align}
        E_{1,2}=\sqrt{m^2+p^2},\quad E_{3,4}=-\sqrt{m^2+p^2}.
    \end{align}
\end{sol}

\begin{prob}
    对核子数为 $Z$ 的类氢原子, 估算其基态波函数大小分量模平方的比值.
\end{prob}
\begin{sol}
    类氢原子的相对论性哈密顿量为
    \begin{align}
        H=(\bm{p}^2c^2+m^2c^4)^{1/2}-\frac{Ze^2}{r},
    \end{align}
    从而其狄拉克方程为
    \begin{align}
        i\hbar\frac{\partial\Psi}{\partial t}=\left[c\bm{\alpha}\cdot\bm{p}+\beta mc^2-\frac{Ze^2}{r}\right]\Psi.
    \end{align}
    展开精确至 $(v/c)^2$, 有
    \begin{gather}
        \begin{bmatrix}
            E-mc^2+\frac{Ze^2}{r}&-c\bm{\sigma}\cdot\bm{p}\\
            -c\bm{\sigma}\cdot\bm{p}&E+mc^2-\frac{Ze^2}{r}
        \end{bmatrix}\begin{bmatrix}
            \chi\\
            \Phi
        \end{bmatrix}=\begin{bmatrix}
            0\\
            0
        \end{bmatrix}\\
        \Longrightarrow\Phi=\left(E-mc^2+\frac{Ze^2}{r}\right)\chi=c\bm{\sigma}\cdot\bm{p}\Phi,\quad\left(E+mc^2+\frac{Ze^2}{r}\right)\Phi=c\bm{\sigma}\cdot\bm{p}\chi,
    \end{gather}
    从而对于基态, 波函数大小分量模平方的比值可估算为
    \begin{align}
        \frac{\abs{\Psi}}{\abs{\chi}}=&\frac{\abs{c\bm{\sigma}\cdot\bm{p}}}{\abs{E+mc^2+\frac{Ze^2}{r}}}\approx\frac{c\langle p^2\rangle^{1/2}}{2mc^2}\\
        \notag\approx&\frac{c}{2mc^2}\left[4\left(\frac{a_0}{Z}\right)^3\int_0^{\infty}\mathrm{d}r\,e^{-Zr/a_0}\left(-\hbar^2\frac{1}{r^2}\frac{\partial}{\partial r}r^2\frac{\partial}{\partial r}e^{-Zr/a_0}\right)\right]^{1/2}\\
        =&\frac{\hbar Z}{2mca_0}=\frac{Z}{2\alpha}=\frac{Z}{274},
    \end{align}
    其中 $a_0$ 为玻尔半径, $\alpha=\frac{1}{137}$ 为精细结构常数.
\end{sol}

\begin{prob}
    若 $4$ 个厄米矩阵 $M_i$ ($i=1,2,3,4$) 满足关系: $M_iM_j+M_jM_i=2\delta_{ij}$, 证明:
    \begin{itemize}
        \item[1)] $M_i$ 的本征值为 $\pm 1$;
        \item[2)] $M_i$ 的迹为 $0$;
        \item[3)] $M_i$ 必为偶数位矩阵.
    \end{itemize}
\end{prob}
\begin{pf}
    \begin{itemize}
        \item[1)] 由于 $M_i$ ($i=1,2,3,4$) 为厄米矩阵, 故其本征值必为实数. 假设 $\psi_i$ 为 $M_i$ 的本征矢, 对应的本征态为 $m_i$, 则
        \begin{gather}
            M_i\psi_i=m_i\psi_i,\\
            \Longrightarrow M_i^2\psi_i=m_i^2\psi_i.
        \end{gather}
        又由于 $M_iM_j+M_jM_i=2\delta_{ij}$,
        \begin{gather}
            (M_iM_i+M_iM_i)\psi_i=2\psi_i\\
            \Longrightarrow M_i^2\psi_i=\psi_i,
        \end{gather}
        故
        \begin{gather}
            m_i^2=1,\\
            \Longrightarrow m_i=\pm 1.
        \end{gather}
        \item[2)] 利用 $M_iM_j+M_jM_i=2\delta_{ij}$,
        \begin{gather}
            M_iM_j=-M_jM_i\\
            \Longrightarrow M_i=-M_jM_iM_j^{-1}\\
            \Longrightarrow\tr(M_i)=-\tr(M_jM_iM_j^{-1})=-\tr(M_j^{-1}M_jM_i)=-\tr(M_i)\\
            \Longrightarrow\tr(M_i)=0.
        \end{gather}
        \item[3)] 利用反证法即可证明. 根据线性代数的知识, 我们知道矩阵的本征值之和即为其迹. 假设 $M_i$ 为奇数位矩阵, 则 $M_i$ 具有奇数个本征值, 这些本征值为 $\pm 1$, 它们的和必 $\neq 0$, 这与上一小题得到的结论相悖, 故假设错误, 故 $M_i$ 必为偶数位矩阵.
    \end{itemize}
\end{pf}
\end{document}