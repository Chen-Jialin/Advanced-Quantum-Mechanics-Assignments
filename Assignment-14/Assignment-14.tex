\documentclass{assignment}
\ProjectInfos{高等量子力学}{PHYS5001P}{2022-2023 学年第一学期}{第十四次作业}{截止时间: 2023 年 1 月 2 日 (周一)}{陈稼霖}[https://github.com/Chen-Jialin]{SA21038052}

\begin{document}
\begin{prob}[课本习题 6.1]
    李普曼-施温格形式也能用于仅当 $0<\abs{x}<a$ 时 $V(x)\neq 0$ 的一个有限力程势的一维透射-反射问题.
    \begin{itemize}
        \item[(a)] 假定有一个来自左边的入射波: $\langle x\vert\phi\rangle=e^{ikx}/\sqrt{2\pi}$. 如果想要只在 $x>a$ 的区域有一个透射波, 在 $x<-a$ 的区域有一个反射波和原始的波, 必须如何处理奇异的 $1/(E-H_0)$ 算符? $E\rightarrow E+i\varepsilon$ 的做法是否仍然正确? 求一个恰当的格林函数表达式, 并且写出 $\langle x\vert\psi^{(+)}\rangle$ 的一个积分方程.
        \item[(b)] 考虑一个吸引的 $\delta$ 函数的特例
        \[
            V=-\left(\frac{\gamma\hbar^2}{2m}\right)\delta(x)\quad(\gamma>0).
        \]
        求解这个积分方程以得到透射和反射振幅. 检查该结果是否与 Gottfried 1966, 52 页的相符合.
        \item[(c)] 有着 $\gamma>0$ 的一维 $\delta$ 函数势, 对 $\gamma$ 取任意值, 都允许一个 (且仅一个) 束缚态. 当 $k$ 被看做是一个复变量时, 证明你算出的透射和反射振幅在预期的位置具有束缚态的极点.
    \end{itemize}
\end{prob}
\begin{sol}
    \begin{itemize}
        \item[(a)] 如果想要只在 $x>a$ 的区域有一个透射波, 在 $x<-a$ 的区域有一个反射波和原始的波, 则应当在 $x>a$ 的区域 $1/(E-H_0)\rightarrow 1/(E-H_0+i\varepsilon)$, 即 $E\rightarrow E+i\varepsilon$ 的做法仍然正确, 这将导致 $\langle x\vert\psi^{(+)}\rangle$ 无论在 $x>a$ 还是 $x<a$ 的区域都有一个原始波 + 散射波的形式, 在接下来的计算中会看到, 这最终会表现为在 $x>a$ 的区域有一个透射波, 在 $x<-a$ 的区域有一个反射波和原始的波.

        此时,
        \begin{align}
            \lvert\psi^{(+)}\rangle=\lvert\phi\rangle+\frac{1}{E-H_0+i\varepsilon}V\lvert\psi^{(+)}\rangle.
        \end{align}
        格林函数:
        \begin{align}
            \notag G_{(+)}(x,x')=&\frac{\hbar^2}{2m}\langle x\rvert\frac{1}{E-H_0+i\varepsilon}\lvert x'\rangle\\
            \notag=&\frac{\hbar^2}{2m}\sum_{k'}\sum_{k''}\langle x\vert k'\rangle\langle k'\rvert\frac{1}{E-H_0+i\varepsilon}\lvert k''\rangle\langle k''\vert x'\rangle\\
            \notag=&\frac{1}{L}\sum_{k'}\frac{e^{ik'(x-x')}}{k^2-k'^2+i\varepsilon}\\
            \notag=&\frac{1}{L}\int\frac{\mathrm{d}k'}{2\pi/L}\,\frac{e^{ik'(x-x')}}{k^2-k'^2+i\varepsilon}\\
            \notag&(\text{重定义 }\varepsilon>0)\\
            =&\frac{1}{2\pi}\int\mathrm{d}k'\,\frac{e^{ik'(x-x')}}{-(k'-k-i\varepsilon)(k'+k+i\varepsilon)}.
        \end{align}
        对 $x>x'$, 取上半复平面的积分半圆,
        \begin{align}
            G_{(+)}=\frac{1}{2\pi}2\pi i\res\left[\frac{e^{ik'(x-x')}}{-(k'-k-i\varepsilon)(k'+k+i\varepsilon)},k+i\varepsilon\right]=\frac{1}{2\pi}2\pi i\frac{e^{ik(x-x')}}{-2k}=-\frac{ie^{ik(x-x')}}{2k}.
        \end{align}
        对 $x<x'$, 去下半复平面的积分半圆,
        \begin{align}
            G_{(+)}=\frac{1}{2\pi}2\pi i\res\left[\frac{e^{ik'(x-x')}}{-(k'-k-i\varepsilon)(k'+k+i\varepsilon)},-k-i\varepsilon\right]=-\frac{1}{2\pi}2\pi i\frac{e^{-ik(x-x')}}{2k}=-\frac{ie^{-ik(x-x')}}{2k}.
        \end{align}
        综上, 格林函数表达式:
        \begin{align}
            G_{(+)}(x,x')=\left\{\begin{array}{ll}
                -\frac{ie^{ik(x-x')}}{2k},&x>x',\\
                -\frac{ie^{-ik(x-x')}}{2k},&x<x'.
            \end{array}\right.
        \end{align}
        散射波函数:
        \begin{align}
            \notag\langle x\vert\psi^{(+)}\rangle=&\langle x\vert\phi\rangle+\frac{2m}{\hbar^2}\int\mathrm{d}x'\,\langle x\rvert\frac{1}{E-H_0+i\varepsilon}\lvert x'\rangle\langle x'\rvert V\lvert\psi^{(+)}\rangle\\
            \notag=&\frac{e^{ikx}}{\sqrt{2\pi}}+\frac{2m}{\hbar^2}\int\mathrm{d}x'\,G_{(+)}(x,x')V(x')\langle x'\vert\psi^{(+)}\rangle\\
            =&\left\{\begin{array}{ll}
                \frac{e^{ikx}}{\sqrt{2\pi}}-\frac{2m}{\hbar^2}\frac{i}{2k}\int\mathrm{d}x'\,e^{ik(x-x')}V(x')\langle x'\vert\psi^{(+)}\rangle,&x>a,\\
                \frac{e^{ikx}}{\sqrt{2\pi}}-\frac{2m}{\hbar^2}\frac{i}{2k}\int\mathrm{d}x'\,e^{-ik(x-x')}V(x')\langle x'\vert\psi^{(+)}\rangle,&x<-a
            \end{array}\right.
        \end{align}
        \item[(b)] 对 $x>a$,
        \begin{align}
            \langle x\vert\psi^{(+)}\rangle=\frac{e^{ikx}}{\sqrt{2\pi}}-\frac{2m}{\hbar^2}\frac{i}{2k}\int\mathrm{d}x'\,e^{ik(x-x')}\left(-\frac{\gamma\hbar^2}{2m}\right)\delta(x')\langle x'\vert\psi^{(+)}\rangle=\frac{e^{ikx}}{\sqrt{2\pi}}+\frac{i\gamma}{2k}e^{ikx}\psi^{(+)}(0),
        \end{align}
        上式中取 $x=0$ 得
        \begin{gather}
            \psi^{(+)}(0)=\frac{1}{\sqrt{2\pi}}+\frac{i\gamma}{2k}\psi^{(+)}(0),\\
            \Longrightarrow\psi^{(+)}(0)=\frac{1}{\sqrt{2\pi}}\frac{2k}{2k-i\gamma},
        \end{gather}
        从而
        \begin{align}
            \langle x\vert\psi^{(+)}\rangle=\frac{e^{ikx}}{\sqrt{2\pi}}\left[1+\frac{i\gamma}{2k-i\gamma}\right]=\frac{e^{ikx}}{\sqrt{2\pi}}\frac{2k}{2k-i\gamma},\quad\text{for }x>a.
        \end{align}
        故透射系数为
        \begin{align}
            T=\frac{2k}{2k-i\gamma}.
        \end{align}
        对 $x<-a$,
        \begin{align}
            \langle x\vert\psi^{(+)}\rangle=\frac{e^{ikx}}{\sqrt{2\pi}}-\frac{2m}{\hbar^2}\frac{i}{2k}\int\mathrm{d}x'\,e^{-ik(x-x')}\left(-\frac{\gamma\hbar^2}{2m}\right)\delta(x')\langle x'\vert\psi^{(+)}\rangle=\frac{e^{ikx}}{\sqrt{2\pi}}+\frac{i\gamma}{2k-i\gamma}e^{-ikx}.
        \end{align}
        故反射系数为
        \begin{align}
            R=\frac{i\gamma}{2k-i\gamma}.
        \end{align}
        \item[(c)] 由课本习题 (2.24) 的结论, 该一维 $\delta$ 函数势的束缚态的波函数为
        \begin{align}
            \psi(x)=\left\{\begin{array}{ll}
                \sqrt{\frac{\gamma}{2}}e^{-ikx},&x<0,\\
                \sqrt{\frac{\gamma}{2}}e^{ikx},&x>0,
            \end{array}\right.
        \end{align}
        其中 $k=i\frac{\gamma}{2}$. 而上述透射-反射问题的透射和反射振幅均在 $k=i\frac{\gamma}{2}$ 处达到极值, 这与束缚态的 $k$ 一致.
    \end{itemize}
\end{sol}

\begin{prob}[课本习题 6.4]
    考虑一个势
    \[
        V=0\quad\text{对 }r>R\qquad V=V_0=\text{常数}\quad\text{对 }r<R.
    \]
    其中的 $V_0$ 可能是正的或负的. 采用分波法, 证明对 $\abs{V_0}\ll E=\hbar^2k^2/2m$ 和 $kR\ll 1$, 微分截面是各向同性的, 并且总截面由
    \[
        \sigma_{\text{总}}=\left(\frac{16\pi}{9}\right)\frac{m^2V_0^2R^6}{\hbar^4}
    \]
    给出. 假定能量稍稍增加. 证明角分布能被写成
    \[
        \frac{\mathrm{d}\sigma}{\mathrm{d}\Omega}=A+B\cos\theta.
    \]
    求 $B/A$ 的一个近似表达式.
\end{prob}
\begin{sol}
    在 $r>R$ 处, $A_l(r)$ 的具有这样的形式:
    \begin{align}
        A_l(r)=e^{i\delta_1}[\cos\delta_lj_l(kr)-\sin\delta_ln_l(kr)],
    \end{align}
    其中 $n_l$ 为 $l$ 阶球 Neumann 函数. 其在 $r=R$ 处的对数导数为
    \begin{align}
        \beta_l=kR\left[\frac{j_l'(kR)\cos\delta_l-n_l'(kR)\sin\delta_l}{j_l(kR)\cos\delta_l-n_l(kR)\sin\delta_l}\right].
    \end{align}
    相移 $\delta_l$ 满足
    \begin{align}
        \tan\delta_l=\frac{kRj_l'(kR)-\beta_lj_l(kR)}{kRn_l'(kR)-\beta_ln_l(kR)}.
    \end{align}
    利用适用于球 Bessel 和球 Neumann 函数的递推式 $f_l'(x)=\frac{l}{x}f_l(x)-f_{l+1}(x)$, 处理掉上式中的微分项得
    \begin{align}
        \label{A14-2-phase-shift}
        \tan\delta_l=\frac{kR\left[\frac{l}{kR}j_l(kR)-j_{l+1}(kR)\right]-\beta_lj_l(kR)}{kR\left[\frac{l}{lR}n_l(kR)-n_{l+1}(kR)\right]-\beta_ln_l(kR)}=\frac{\left[lj_l(kR)-kRj_{l+1}(kR)\right]-\beta_lj_l(kR)}{\left[ln_l(kR)-kRn_{l+1}(kR)\right]-\beta_ln_l(kR)}.
    \end{align}

    在 $r<R$ 处, 求解
    \begin{align}
        \frac{\mathrm{d}^2u_l}{\mathrm{d}r^2}+\left(k^2-\frac{2m}{\hbar^2}V_0-\frac{l(l+1)}{r^2}\right)u_l=0,
    \end{align}
    考虑到边界条件 $u_l|_{r=0}=0$ 得
    \begin{gather}
        u_l=rA_l(r)=\kappa rj_l(\kappa r),\quad\text{for }0\leq r<R,\\
        \Longrightarrow A_l(r)=\kappa j_l(\kappa r),\quad\text{for }0\leq r<R,
    \end{gather}
    其中 $j_l$ 为 $l$ 阶球 Bessel 函数, $\kappa$ 满足 $\frac{\hbar^2\kappa^2}{2m}=\frac{\hbar k^2}{2m}-V_0$. 在 $r=R$ 处的对数微商为
    \begin{align}
        \beta_l=\left(\frac{r}{A}\frac{\mathrm{d}A_l}{\mathrm{d}r}\right)_{r=R}=\frac{\kappa Rj_l'(\kappa R)}{j_l(\kappa R)}.
    \end{align}
    利用适用于球 Bessel 和球 Neumann 函数的递推式 $f_l'(x)=\frac{l}{x}f_l(x)-f_{l+1}(x)$, 处理掉上式中的微分项得
    \begin{align}
        \beta_l=\frac{lj_l(\kappa R)-\kappa Rj_{l+1}(\kappa R)}{j_l(\kappa R)}=l-\frac{\kappa Rj_l(\kappa R)}{j_l(\kappa R)}.
    \end{align}
    将上式代入式 \eqref{A14-2-phase-shift} 中得
    \begin{align}
        \notag\tan\delta_l=&\frac{\left[lj_l(kR)-kRj_{l+1}(kR)\right]-\left[l-\frac{\kappa Rj_l(\kappa R)}{j_l(\kappa R)}\right]j_l(kR)}{\left[ln_l(kR)-kRn_{l+1}(kR)\right]-\left[l-\frac{\kappa Rj_l(\kappa R)}{j_l(\kappa R)}\right]n_l(kR)}\\
        =&\frac{-kRj_{l+1}(kR)+\kappa R\frac{j_{l+1}(\kappa R)}{j_l(\kappa R)}j_l(kR)}{-kRn_{l+1}(kR)+\kappa R\frac{j_{l+1}(\kappa R)}{j_l(\kappa R)}n_l(kR)}.
    \end{align}

    对 $\abs{V_0}\ll E=\hbar^2k^2/2m$, $\kappa\approx k$, 又对 $kR\ll 1$, 故 $j_l(kR)\rightarrow\frac{(kR)^l}{(2l+1)!!}$, $n_l(kR)\rightarrow-\frac{(2l-1)!!}{(kR)^{l+1}}$, 从而
    \begin{align}
        \notag\tan\delta_l\approx&\frac{-\frac{(kR)^{l+2}}{(2l+3)!!}+\frac{(\kappa R)^2(kR)^l}{(2l+3)!!}}{\frac{(2l+1)!!}{(kR)^{l+1}}-\frac{(2l+1)!!(2l-1)!!}{(2l+3)!!}\frac{(\kappa R)^2}{(kR)^{l+1}}}\\
        \notag=&(kR)^{2l+1}\frac{(\kappa R)^2-(kR)^2}{(2l+3)!!(2l+1)!!-(2l+1)!!(2l-1)!!}\\
        \notag\approx&(kR)^{2l+1}\frac{(\kappa R)^2-(kR)^2}{(2l+3)!!(2l+1)!!}\\
        \notag=&\frac{(kR)^{2l+3}}{(2l+3)!!(2l+1)!!}\left[\left(\frac{\kappa}{k}\right)^2-1\right]\\
        \notag=&\frac{(kR)^{2l+3}}{(2l+3)!!(2l+1)!!}\left[\frac{E-V_0}{E}-1\right]\\
        =&-\frac{(kR)^{2l+3}}{(2l+3)!!(2l+1)!!}\frac{V_0}{E},
    \end{align}
    \begin{align}
        \sin\delta_0\approx\tan\delta_0\approx-\frac{(kR)^3}{3}\frac{V_0}{E}=-\frac{2mV_0kR^3}{3\hbar^2}.
    \end{align}
    故总截面为
    \begin{align}
        \sigma_{\text{总}}=4\pi\frac{\sin^2\delta_0}{k^2}=\frac{16\pi}{9}\frac{m^2V_0^2R^6}{\hbar^4}.
    \end{align}
    假定能量稍稍增加,
    \begin{align}
        \sin\delta_1\approx\tan\delta_1\approx-\frac{(kR)^5}{45}\frac{V_0}{E}.
    \end{align}
    角分布为
    \begin{align}
        \notag\frac{\mathrm{d}\sigma}{\mathrm{d}\Omega}=&\abs{f(\theta)}^2=\abs{\frac{1}{k}\sum_{l=0}(2l+1)e^{i\delta_l}\sin\delta_lP_l(\cos\theta)}^2\\
        \notag\approx&\frac{1}{k^2}\abs{e^{i\delta_0}\sin\delta_0+3e^{i\delta_1}\sin\delta_l\cos\theta}^2\\
        \notag\approx&\frac{1}{k^2}\left[\sin^2\delta_0+6\cos(\delta_1-\delta_0)\sin\delta_0\sin\delta_1\cos\theta\right]\\
        \approx&\frac{1}{k^2}\left[\sin^2\delta_0+6\sin\delta_0\sin\delta_1\cos\theta\right],
    \end{align}
    故
    \begin{align}
        \frac{B}{A}=\frac{6\sin\delta_1}{\sin\delta_0}=\frac{2}{5}(kR)^2.
    \end{align}
\end{sol}
\end{document}