\documentclass{assignment}
\ProjectInfos{高等量子力学}{PHYS5001P}{2022-2023 学年第一学期}{第二次作业}{截止时间: 2022 年 9 月 26 日 (周一)}{陈稼霖}[https://github.com/Chen-Jialin]{SA21038052}

\begin{document}
\begin{prob}[课本习题 1.29]
    \begin{itemize}
        \item[(a)] Gottfried (1966) 在他的书的 247 页上说: 对所有能表示成其宗量的幂级数的函数 $F$ 和 $G$, 从基本对易关系都可以 "容易地推导" 出
        \[
            [x_i,G(\bm{p})]=i\hbar\frac{\partial G}{\partial p_i},\quad[p_i,F(\bm{x})]=-i\hbar\frac{\partial F}{\partial x_i}
        \]
        证明这个说法.
        \item[(b)] 求 $[x^2,p^2]$ 的值, 把你的结果与经典的泊松括号 $[x^2,p^2]_{\text{经典}}$ 相比较.
    \end{itemize}
\end{prob}
\begin{sol}
    \begin{itemize}
        \item[(a)] 通过泰勒展开将函数 $F$ 和 $G$ 分别表为其宗量的幂级数和
        \begin{align}
            F(\bm{x})=&\sum_{n=0}^{\infty}\sum_{m=0}^{\infty}\sum_{l=0}^{\infty}\begin{pmatrix}
                n+m+l\\
                n
            \end{pmatrix}\begin{pmatrix}
                m+l\\
                m
            \end{pmatrix}\frac{1}{(n+m+l)!}\frac{\partial^{n+m+l}F}{\partial x_i^n\partial x_j^m\partial x_k^l}x_i^lx_j^mx_k^n\equiv\sum_{nml}f_{nml}x_i^nx_j^mx_k^l,\\
            G(\bm{p})=&\sum_{n=0}^{\infty}\sum_{m=0}^{\infty}\sum_{l=0}^{\infty}\begin{pmatrix}
                n+m+l\\
                n
            \end{pmatrix}\begin{pmatrix}
                m+l\\
                m
            \end{pmatrix}\frac{1}{(n+m+l)!}\frac{\partial^{n+m+l}G}{\partial p_i^n\partial p_j^m\partial p_k^l}p_i^np_j^mp_k^l\equiv\sum_{nml}g_{nml}p_i^np_j^mp_k^l
        \end{align}
        利用基本对易关系
        \begin{align}
            [x_i,p_j]=i\hbar\delta_{ij},
        \end{align}
        有
        \begin{align}
            \notag[x_i,p_i^n]=&p_i[x_i,p_i^{n-1}]+[x_i,p_i]p_i^{n-1}=p_i[x_i,p_i^{n-1}]+i\hbar p_i^{n-1}\\
            \notag=&p_i^2[x_i,p_i^{n-2}]+p_i[x_i,p_i]p_i^{n-2}+i\hbar p_i^{n-1}=p_i^2[x_i,p_i^{n-2}]+2i\hbar p_i^{n-1}\\
            \notag\cdots&\\
            =&ni\hbar p_i^{n-1},\\
            \notag[p_i,x_i^n]=&x_i[p_i,x_i^{n-1}]+[p_i,x_i]x_i^{n-1}=x_i[p_i,x_i^{n-1}]-i\hbar x_i^{n-1}\\
            \notag=&x_i^2[p_i,x_i^{n-2}]+x_i[p_i,x_i]x_i^{n-2}-i\hbar x_i^{n-1}=x_i^2[p_i,x_i^{n-2}]-2i\hbar x_i^{n-1}\\
            \notag\cdots&\\
            =&-ni\hbar x_i^{n-1},
        \end{align}
        从而
        \begin{align}
            [x_i,G(\bm{p})]=&[x_i,\sum_{nml}g_{nml}p_i^np_j^mp_k^l]=\sum_{nml}g_{nml}[x_i,p_i^n]p_j^mp_k^l=i\hbar\sum_{nml}g_{nml}np_i^{n-1}p_j^mp_k^l=i\hbar\frac{\partial G}{\partial p},\\
            [p_i,F(\bm{x})]=&[p_i,\sum_{nml}f_{nml}x_i^nx_j^mx_k^l]=\sum_{nml}g_{nml}[p_i,x_i^n]x_j^mx_k^l=-i\hbar\sum_{nml}f_{nml}nx_i^{n-1}x_j^mx_k^l=-i\hbar\frac{\partial F}{\partial x}.
        \end{align}
        \item[(b)] 
        \begin{align}
            [x^2,p^2]=x[x,p^2]+[x,p^2]x=x\{p[x,p]+[x,p],p\}+\{p[x,p]+[x,p]p\}x=2i\hbar(xp+px)
        \end{align}
        经典的泊松括号:
        \begin{align}
            [x^2,p^2]_{\text{经典}}=\frac{\partial x^2}{\partial x}\frac{\partial p^2}{\partial p}-\frac{\partial x^2}{\partial p}\frac{\partial p^2}{\partial x}=4xp.
        \end{align}
        只需将经典的泊松括号厄米化, 即可得到与量子力学中一致的形式:
        \begin{align}
            [x^2,p^2]_{\text{经典}}=4xp\overset{\text{厄米化}}{\longrightarrow}2xp+2px=\frac{1}{i\hbar}[x^2,p^2].
        \end{align}
    \end{itemize}
\end{sol}

\begin{prob}[课本习题 1.31]
    在正文中我们讨论了 $\mathscr{T}(\mathrm{d}\bm{x}')$ 在位置和动量本征右矢上以及在一个更一般的态右矢 $\lvert\alpha\rangle$ 的效应. 我们还可以研究期待值 $\langle\bm{x}\rangle$ 和 $\langle\bm{p}\rangle$ 在无穷小平移下的行为. 利用 (1.6.25) 式和 (1.6.45) 式并令 $\lvert\alpha\rangle\rightarrow\mathscr{T}(\mathrm{d}\bm{x}')\lvert\alpha\rangle$, 证明在无穷小平移下 $\langle\bm{x}\rangle\rightarrow\langle\bm{x}\rangle+\mathrm{d}\bm{x}'$, $\langle\bm{p}\rangle\rightarrow\langle\bm{p}\rangle$.
\end{prob}
\begin{pf}
    平移前,
    \begin{align}
        \langle\bm{x}\rangle=&\langle\alpha\rvert\bm{x}\lvert\alpha\rangle,\\
        \langle\bm{p}\rangle=&\langle\alpha\rvert\bm{p}\lvert\alpha\rangle.
    \end{align}
    平移后, 利用 (1.6.25) 式 $[\bm{x},\mathscr{T}(\mathrm{d}\bm{x}')]=\mathrm{d}\bm{x}'$, 有
    \begin{align}
        \notag\langle\alpha\rvert\mathscr{T}^{\dagger}(\mathrm{d}\bm{x}')\bm{x}\mathscr{T}(\mathrm{d}\bm{x}')\lvert\alpha\rangle=&\langle\alpha\rvert\mathscr{T}^{\dagger}(\mathrm{d}\bm{x}')[\mathscr{T}(\mathrm{d}\bm{x}')\bm{x}+\mathrm{d}\bm{x}']\lvert\alpha\rangle\\
        \notag=&\langle\alpha\rvert\mathscr{T}^{\dagger}(\mathrm{d}\bm{x}')\mathscr{T}(\mathrm{d}\bm{x}')\bm{x}\lvert\alpha\rangle+\langle\alpha\rvert\mathscr{T}^{\dagger}(\mathrm{d}\bm{x}')\mathrm{d}\bm{x}'\lvert\alpha\rangle\\
        \notag=&\langle\alpha\rvert\bm{x}\lvert\alpha\rangle+\langle\alpha\rvert(1+i\bm{K}\cdot\mathrm{d}\bm{x}')\mathrm{d}\bm{x}'\lvert\alpha\rangle\\
        \notag&(\text{略去高阶小量})\\
        \notag=&\langle\bm{x}\rangle+\langle\alpha\rvert\mathrm{d}\bm{x}'\lvert\alpha\rangle\\
        =&\langle\bm{x}\rangle+\mathrm{d}\bm{x}',
    \end{align}
    由于平移算符 $\mathscr{T}(\mathrm{d}\bm{x}')$ 可表为动量算符 $\bm{p}$ 的幂级数之和, 利用 (1.6.45) 式 $[p_i,p_j]=0$, 有 $[\mathscr{T}(\mathrm{d}\bm{x}'),\bm{p}]=0$, 从而
    \begin{align}
        \langle\alpha\rvert\mathscr{T}^{\dagger}(\mathrm{d}\bm{x}')\bm{p}\mathscr{T}(\mathrm{d}\bm{x}')\lvert\rangle=\langle\alpha\rvert\mathscr{T}^{\dagger}(\mathrm{d}\bm{x}')\mathscr{T}(\mathrm{d}\bm{x}')\bm{p}\lvert\alpha\rangle=\langle\alpha\rvert\bm{p}\lvert\alpha\rangle=\langle\bm{p}\rangle.
    \end{align}
\end{pf}
\end{document}