\documentclass{assignment}
\ProjectInfos{高等量子力学}{PHYS5001P}{2022-2023 学年第一学期}{第二次作业}{截止时间: 2022 年 9 月 26 日 (周一)}{陈稼霖}[https://github.com/Chen-Jialin]{SA21038052}

\begin{document}
\begin{prob}[课本习题 1.29]
    \begin{itemize}
        \item[(a)] Gottfried (1966) 在他的书的 247 页上说: 对所有能表示成其宗量的幂级数的函数 $F$ 和 $G$, 从基本对易关系都可以 ``容易地推导'' 出
        \[
            [x_i,G(\bm{p})]=i\hbar\frac{\partial G}{\partial p_i},\quad[p_i,F(\bm{x})]=-i\hbar\frac{\partial F}{\partial x_i}
        \]
        证明这个说法.
        \item[(b)] 求 $[x^2,p^2]$ 的值, 把你的结果与经典的泊松括号 $[x^2,p^2]_{\text{经典}}$ 相比较.
    \end{itemize}
\end{prob}
\begin{sol}
    \begin{itemize}
        \item[(a)] 通过泰勒展开将函数 $F$ 和 $G$ 分别表为其宗量的幂级数和
        \begin{align}
            F(\bm{x})=&\sum_{n=0}^{\infty}\sum_{m=0}^{\infty}\sum_{l=0}^{\infty}\begin{pmatrix}
                n+m+l\\
                n
            \end{pmatrix}\begin{pmatrix}
                m+l\\
                m
            \end{pmatrix}\frac{1}{(n+m+l)!}\frac{\partial^{n+m+l}F}{\partial x_i^n\partial x_j^m\partial x_k^l}x_i^lx_j^mx_k^n\equiv\sum_{nml}f_{nml}x_i^nx_j^mx_k^l,\\
            G(\bm{p})=&\sum_{n=0}^{\infty}\sum_{m=0}^{\infty}\sum_{l=0}^{\infty}\begin{pmatrix}
                n+m+l\\
                n
            \end{pmatrix}\begin{pmatrix}
                m+l\\
                m
            \end{pmatrix}\frac{1}{(n+m+l)!}\frac{\partial^{n+m+l}G}{\partial p_i^n\partial p_j^m\partial p_k^l}p_i^np_j^mp_k^l\equiv\sum_{nml}g_{nml}p_i^np_j^mp_k^l
        \end{align}
        利用基本对易关系
        \begin{align}
            [x_i,p_j]=i\hbar\delta_{ij},
        \end{align}
        有
        \begin{align}
            \notag[x_i,p_i^n]=&p_i[x_i,p_i^{n-1}]+[x_i,p_i]p_i^{n-1}=p_i[x_i,p_i^{n-1}]+i\hbar p_i^{n-1}\\
            \notag=&p_i^2[x_i,p_i^{n-2}]+p_i[x_i,p_i]p_i^{n-2}+i\hbar p_i^{n-1}=p_i^2[x_i,p_i^{n-2}]+2i\hbar p_i^{n-1}\\
            \notag\cdots&\\
            =&ni\hbar p_i^{n-1},\\
            \notag[p_i,x_i^n]=&x_i[p_i,x_i^{n-1}]+[p_i,x_i]x_i^{n-1}=x_i[p_i,x_i^{n-1}]-i\hbar x_i^{n-1}\\
            \notag=&x_i^2[p_i,x_i^{n-2}]+x_i[p_i,x_i]x_i^{n-2}-i\hbar x_i^{n-1}=x_i^2[p_i,x_i^{n-2}]-2i\hbar x_i^{n-1}\\
            \notag\cdots&\\
            =&-ni\hbar x_i^{n-1},
        \end{align}
        从而
        \begin{align}
            [x_i,G(\bm{p})]=&[x_i,\sum_{nml}g_{nml}p_i^np_j^mp_k^l]=\sum_{nml}g_{nml}[x_i,p_i^n]p_j^mp_k^l=i\hbar\sum_{nml}g_{nml}np_i^{n-1}p_j^mp_k^l=i\hbar\frac{\partial G}{\partial p},\\
            [p_i,F(\bm{x})]=&[p_i,\sum_{nml}f_{nml}x_i^nx_j^mx_k^l]=\sum_{nml}g_{nml}[p_i,x_i^n]x_j^mx_k^l=-i\hbar\sum_{nml}f_{nml}nx_i^{n-1}x_j^mx_k^l=-i\hbar\frac{\partial F}{\partial x}.
        \end{align}
        \item[(b)] 
        \begin{align}
            [x^2,p^2]=x[x,p^2]+[x,p^2]x=x\{p[x,p]+[x,p],p\}+\{p[x,p]+[x,p]p\}x=2i\hbar(xp+px)
        \end{align}
        经典的泊松括号:
        \begin{align}
            [x^2,p^2]_{\text{经典}}=\frac{\partial x^2}{\partial x}\frac{\partial p^2}{\partial p}-\frac{\partial x^2}{\partial p}\frac{\partial p^2}{\partial x}=4xp.
        \end{align}
        只需将经典的泊松括号厄米化, 即可得到与量子力学中一致的形式:
        \begin{align}
            [x^2,p^2]_{\text{经典}}=4xp\overset{\text{厄米化}}{\longrightarrow}2xp+2px=\frac{1}{i\hbar}[x^2,p^2].
        \end{align}
    \end{itemize}
\end{sol}

\begin{prob}[课本习题 1.31]
    在正文中我们讨论了 $\mathscr{T}(\mathrm{d}\bm{x}')$ 在位置和动量本征右矢上以及在一个更一般的态右矢 $\lvert\alpha\rangle$ 的效应. 我们还可以研究期待值 $\langle\bm{x}\rangle$ 和 $\langle\bm{p}\rangle$ 在无穷小平移下的行为. 利用 (1.6.25) 式和 (1.6.45) 式并令 $\lvert\alpha\rangle\rightarrow\mathscr{T}(\mathrm{d}\bm{x}')\lvert\alpha\rangle$, 证明在无穷小平移下 $\langle\bm{x}\rangle\rightarrow\langle\bm{x}\rangle+\mathrm{d}\bm{x}'$, $\langle\bm{p}\rangle\rightarrow\langle\bm{p}\rangle$.
\end{prob}
\begin{pf}
    \uline{平移前},
    \begin{align}
        \langle\bm{x}\rangle=&\langle\alpha\rvert\bm{x}\lvert\alpha\rangle,\\
        \langle\bm{p}\rangle=&\langle\alpha\rvert\bm{p}\lvert\alpha\rangle.
    \end{align}
    \uline{平移后}, 利用 (1.6.25) 式 $[\bm{x},\mathscr{T}(\mathrm{d}\bm{x}')]=\mathrm{d}\bm{x}'$, 有
    \begin{align}
        \notag\langle\alpha\rvert\mathscr{T}^{\dagger}(\mathrm{d}\bm{x}')\bm{x}\mathscr{T}(\mathrm{d}\bm{x}')\lvert\alpha\rangle=&\langle\alpha\rvert\mathscr{T}^{\dagger}(\mathrm{d}\bm{x}')[\mathscr{T}(\mathrm{d}\bm{x}')\bm{x}+\mathrm{d}\bm{x}']\lvert\alpha\rangle\\
        \notag=&\langle\alpha\rvert\mathscr{T}^{\dagger}(\mathrm{d}\bm{x}')\mathscr{T}(\mathrm{d}\bm{x}')\bm{x}\lvert\alpha\rangle+\langle\alpha\rvert\mathscr{T}^{\dagger}(\mathrm{d}\bm{x}')\mathrm{d}\bm{x}'\lvert\alpha\rangle\\
        \notag=&\langle\alpha\rvert\bm{x}\lvert\alpha\rangle+\langle\alpha\rvert(1+i\bm{K}\cdot\mathrm{d}\bm{x}')\mathrm{d}\bm{x}'\lvert\alpha\rangle\\
        \notag&(\text{略去高阶小量})\\
        \notag=&\langle\bm{x}\rangle+\langle\alpha\rvert\mathrm{d}\bm{x}'\lvert\alpha\rangle\\
        =&\langle\bm{x}\rangle+\mathrm{d}\bm{x}',
    \end{align}
    由于平移算符 $\mathscr{T}(\mathrm{d}\bm{x}')$ 可表为动量算符 $\bm{p}$ 的幂级数之和, 利用 (1.6.45) 式 $[p_i,p_j]=0$, 有 $[\mathscr{T}(\mathrm{d}\bm{x}'),\bm{p}]=0$, 从而
    \begin{align}
        \langle\alpha\rvert\mathscr{T}^{\dagger}(\mathrm{d}\bm{x}')\bm{p}\mathscr{T}(\mathrm{d}\bm{x}')\lvert\rangle=\langle\alpha\rvert\mathscr{T}^{\dagger}(\mathrm{d}\bm{x}')\mathscr{T}(\mathrm{d}\bm{x}')\bm{p}\lvert\alpha\rangle=\langle\alpha\rvert\bm{p}\lvert\alpha\rangle=\langle\bm{p}\rangle.
    \end{align}
\end{pf}

\begin{prob}[课本习题 1.33]
    \begin{itemize}
        \item[(a)] 证明下列各式:
        \begin{itemize}
            \item[i.] $\langle p'\rvert x\lvert\alpha\rangle=i\hbar\frac{\partial}{\partial p'}\langle p'\vert\alpha\rangle$.
            \item[ii.] $\langle\beta\rvert x\lvert\alpha\rangle=\int\mathrm{d}p'\phi_{\beta}^*(p')i\hbar\frac{\partial}{\partial p'}\phi_{\alpha}(p')$, 其中 $\phi_{\alpha}(p')=\langle p'\vert\alpha\rangle$ 和 $\phi_{\beta}(p')=\langle p'\vert\beta\rangle$ 都是动量空间波函数.
        \end{itemize}
        \item[(b)] 
        \[
            \exp\left(\frac{ix\Xi}{\hbar}\right)
        \]
        的物理意义是什么, 其中 $x$ 是位置算符, 而 $\Xi$ 是某个量纲为动量的数? 证明你的答案的正确性.
    \end{itemize}
\end{prob}
\begin{pf}
    \begin{itemize}
        \item[(a)] 
        \begin{itemize}
            \item[i.] 
            \begin{align}
                \langle p'\rvert x\lvert\alpha\rangle=&\int\mathrm{d}p''\,\langle p'\rvert x\lvert p''\rangle\langle p''\vert\alpha\rangle,
            \end{align}
            其中
            \begin{align}
                \notag\langle p'\rvert x\lvert p''\rangle=&\int\mathrm{d}x'\,\langle p'\rvert x\lvert x'\rangle\langle x'\vert p''\rangle\\
                \notag=&\int\mathrm{d}x'\,x'\langle p'\vert x'\rangle\langle x'\vert p''\rangle\\
                \notag=&\frac{1}{2\pi\hbar}\int\mathrm{d}x'\,x'\exp\left[-i\frac{(p'-p'')x'}{\hbar}\right]\\
                \notag=&\frac{1}{2\pi\hbar}\int\mathrm{d}x'\,i\hbar\frac{\partial}{\partial p'}\exp\left[-i\frac{(p'-p'')x'}{\hbar}\right]\\
                \notag=&i\hbar\frac{\partial}{\partial p'}\frac{1}{2\pi\hbar}\int\mathrm{d}x'\,\exp\left[-i\frac{(p'-p'')x'}{\hbar}\right]\\
                =&i\hbar\frac{\partial}{\partial p'}\delta(p'-p''),
            \end{align}
            故
            \begin{align}
                \langle p'\rvert x\lvert\alpha\rangle=&\int\mathrm{d}p''\,i\hbar\frac{\partial}{\partial p'}\delta(p'-p'')\langle p''\vert\alpha\rangle=i\hbar\frac{\partial}{\partial p'}\langle p'\vert\alpha\rangle.
            \end{align}
            \item[ii.] 
            \begin{align}
                \langle\beta\rvert x\lvert\alpha\rangle=\int\mathrm{d}p'\,\langle\beta\vert p'\rangle\langle p'\rvert x\lvert\alpha\rangle=\int\mathrm{d}p'\,\langle\beta\vert p'\rangle i\hbar\frac{\partial}{\partial p'}\langle p'\vert\alpha\rangle=\int\mathrm{d}p'\,\phi_{\beta}^*(p')i\hbar\frac{\partial}{\partial p'}\phi_{\alpha}(p').
            \end{align}
        \end{itemize}
        \item[(b)] $\exp\left(\frac{ix\Xi}{\hbar}\right)$ 为动量平移算符. 证明如下:\\
        将 $\exp\left(\frac{ix\Xi}{\hbar}\right)$ 作用于动量算符 $p$ 的本征态 $\lvert p'\rangle$ 后, 其动量变为 $p'+\Xi$:
        \begin{align}
            \notag p\exp\left(\frac{ix\Xi}{\hbar}\right)\lvert p'\rangle=&p\sum_{n=0}^{\infty}\frac{1}{n!}\left(\frac{ix\Xi}{\hbar}\right)^n\lvert p'\rangle\\
            \notag&(\text{利用第 1 题中推得的对易关系: }[p,x^n]=-ni\hbar x^{n-1})\\
            \notag=&\left[\sum_{n=0}^{\infty}\frac{1}{n!}\left(\frac{ix\Xi}{\hbar}\right)^np+\sum_{n=1}^{\infty}\frac{1}{(n-1)!}\left(\frac{ix\Xi}{\hbar}\right)^{n-1}\Xi\right]\lvert p'\rangle\\
            \notag=&(p'+\Xi)\sum_{n=0}^{\infty}\frac{1}{n!}\left(\frac{ix\Xi}{\hbar}\right)^n\lvert p'\rangle\\
            =&(p'+\Xi)\exp\left(\frac{ix\Xi}{\hbar}\right)\lvert p'\rangle,
        \end{align}
        故 $\exp\left(\frac{ix\Xi}{\hbar}\right)$ 为动量平移算符, 其带来的动量变化为 $\Xi$.
    \end{itemize}
\end{pf}

\begin{prob}[课本习题 2.2]
    再看一下第 1 章习题 1.11 的哈密顿量. 假定打字员出了一个错, 把 $H$ 写成
    \[
        H=H_{11}\lvert 1\rangle\langle 1\rvert+H_{22}\lvert 2\rangle\langle 2\rvert+H_{12}\lvert 1\rangle\langle 2\rvert.
    \]
    现在什么原理被破坏了? 通过尝试利用这类不合法的哈密顿量求解最一般的问题 (为了简单, 你可以假定 $H_{11}=H_{22}=0$), 阐明你的观点.
\end{prob}
\begin{sol}
    该非厄米的哈密顿量导致时间演化算符失去了幺正性, 从而破坏了量子态在时间演化中的归一性, 即在该哈密顿量下, 量子态与自身的内积并不能总保持为 $1$.

    简单起见, 不妨假定 $H_{11}=H_{22}=0$, 即
    \begin{align}
        H=H_{12}\lvert 1\rangle\langle 2\rvert.
    \end{align}
    该不含时的哈密顿量对应的时间演化算符为
    \begin{align}
        \notag U(t,t_0)=&\exp\left[\frac{-iH(t-t_0)}{\hbar}\right]\\
        \notag=&\sum_{n=0}^{\infty}\frac{1}{n!}\left[\frac{-iH(t-t_0)}{\hbar}\right]^n\\
        \notag&(\text{利用 }H^2=H_{12}^2\lvert 1\rangle\langle 2\vert 1\rangle\langle 2\rvert=0)\\
        \notag=&1-\frac{iH(t-t_0)}{\hbar}\\
        =&1-\frac{iH_{12}(t-t_0)}{\hbar}\lvert 1\rangle\langle 2\rvert.
    \end{align}
    假设 $t_0$ 时刻的归一化量子态 $\lvert\alpha,t_0\rangle=c_1\lvert 1\rangle+c_2\lvert 2\rangle$ 满足归一化条件 $\langle\alpha,t_0\vert\alpha,t_0\rangle=\abs{c_1}^2+\abs{c_2}^2=1$, 则其演化至 $t$ 时刻为
    \begin{align}
        \lvert\alpha,t_0;t\rangle=U(t,t_0)\lvert\alpha,t_0\rangle=\left[1-\frac{iH_{12}(t-t_0)}{\hbar}\lvert 1\rangle\langle 2\rvert\right](c_1\lvert 1\rangle+c_2\lvert 2\rangle)=\left[c_1-\frac{iH_{12}(t-t_0)}{\hbar}c_2\right]\lvert 1\rangle+c_2\lvert 2\rangle.
    \end{align}
    此时量子态与自身的内积含时, 不再满足归一化条件:
    \begin{align}
        \notag\langle\alpha,t_0;t\vert\alpha,t_0;t\rangle=&\left\{\left[c_1+\frac{iH_{12}(t-t_0)}{\hbar}c_2\right]\langle 1\rvert+c_2\langle 2\rvert\right\}\left\{\left[c_1-\frac{iH_{12}(t-t_0)}{\hbar}c_2\right]\lvert 1\rangle+c_2\lvert 2\rangle\right\}\\
        \notag=&\abs{c_1}^2+\frac{\abs{H_{12}}^2(t-t_0)^2}{\hbar^2}\abs{c_2}^2+\abs{c_2}^2\\
        \notag=&1+\frac{\abs{H_{12}}^2(t-t_0)^2}{\hbar^2}\abs{c_2}^2\\
        \geq&1.
    \end{align}
\end{sol}

\begin{prob}[课本习题 2.3]
    一个电子受到一个时间无关的、强度为 $B$ 的沿正 $z$ 方向的均匀磁场的作用. 在 $t=0$ 时已知电子处在 $\bm{S}\cdot\hat{\bm{n}}$ 的本征态上, 本征值为 $\hbar/2$, 其中 $\hat{\bm{n}}$ 是一个单位矢量, 位于 $xz$ 平面上, 与 $z$ 轴夹 $\beta$ 角.
    \begin{itemize}
        \item[(a)] 求找到电子处在 $s_x=\hbar/2$ 态上作为时间函数的概率.
        \item[(b)] 求作为时间函数的 $S_x$ 的期待值.
        \item[(c)] 为让你自己放心, 在 (i) $\beta\rightarrow 0$ 和 (ii) $\beta\rightarrow\pi/2$ 的极端情况下证明你的答案是有意义的.
    \end{itemize}
\end{prob}
\begin{sol}
    \begin{itemize}
        \item[(a)] $t=0$ 时, 电子的状态为
        \begin{align}
            \lvert\alpha,t=0\rangle=&\cos\frac{\beta}{2}\lvert s_z,+\rangle+\sin\frac{\beta}{2}\lvert s_z,-\rangle.
        \end{align}
        电子和磁场相互作用的哈密顿量
        \begin{align}
            H=-\frac{eB}{mc}S_z\equiv\omega S_z
        \end{align}
        不含时, 故对应的时间演化算符为
        \begin{align}
            U(t,0)=\exp\left(\frac{-iHt}{\hbar}\right)=\exp\left(\frac{-i\omega S_zt}{\hbar}\right).
        \end{align}
        $t$ 时刻电子的状态为
        \begin{align}
            \notag\lvert\alpha,t\rangle=&U(t,0)\lvert\alpha,t\rangle\\
            \notag=&\exp(-i\omega S_zt/\hbar)\left(\cos\frac{\beta}{2}\lvert s_z,+\rangle+\sin\frac{\beta}{2}\lvert s_z,-\rangle\right)\\
            =&\exp(-i\omega t/2)\cos\frac{\beta}{2}\lvert s_z,+\rangle+\exp(i\omega t/2)\sin\frac{\beta}{2}\lvert s_z,-\rangle.
        \end{align}
        电子处在 $s_x=\hbar/2$ 态上概率为
        \begin{align}
            \notag P(s_x=\hbar/2)=&\abs{\langle s_x=\hbar/2\vert\alpha,t\rangle}^2\\
            \notag=&\abs{\frac{1}{\sqrt{2}}(\langle s_z,+\rvert+\langle s_z,-\rvert)\left[\exp(-i\omega t/2)\cos\frac{\beta}{2}\lvert s_z,+\rangle+\exp(i\omega t/2)\sin\frac{\beta}{2}\lvert s_z,-\rangle\right]}^2\\
            \notag=&\frac{1}{2}\abs{\exp(-i\omega t/2)\cos\frac{\beta}{2}+\exp(i\omega t/2)\sin\frac{\beta}{2}}^2\\
            =&\frac{1}{2}[1+\sin\beta\cos(\omega t)].
        \end{align}
        \item[(b)] 算符 $S_x$ 在基 $\{\lvert s_z,\pm\rangle\}$ 上展开的形式为
        \begin{align}
            S_x=\frac{\hbar}{2}(\lvert s_z,+\rangle\langle s_z,-\rvert+\lvert s_z,-\rangle\langle s_z,+\rvert).
        \end{align}
        $S_x$ 的期待值
        \begin{align}
            \notag\langle S_x(t)\rangle=&\langle\alpha,t\rvert S_x\lvert\alpha,t\rangle\\
            \notag=&\left[\exp(i\omega t/2)\cos\frac{\beta}{2}\langle s_z,+\rvert+\exp(-i\omega t/2)\sin\frac{\beta}{2}\langle s_z,-\rvert\right]\frac{\hbar}{2}(\lvert s_z,+\rangle\langle s_z,-\rvert+\lvert s_z,-\rangle\langle s_z,+\rvert)\times\\
            \notag&\left[\exp(-i\omega t/2)\cos\frac{\beta}{2}\lvert s_z,+\rangle+\exp(i\omega t/2)\sin\frac{\beta}{2}\lvert s_z,-\rangle\right]\\
            =&\frac{\hbar}{2}\sin\beta\cos(\omega t).
        \end{align}
        \item[(c)] 
        \begin{itemize}
            \item[(i)] 当 $\beta=0$, 则 $t=0$ 时电子的状态为 $\lvert\alpha,t=0\rangle=\lvert s_z,+\rangle$, 该状态为哈密顿量的本征态, 故关于时间保持不变. $t$ 时刻电子处在 $s_x=\hbar/2$ 的概率为
            \begin{align}
                P(s_x=\hbar/2)=\abs{\langle s_x,+\vert s_z,+\rangle}^2=\abs{\frac{1}{\sqrt{2}}(\langle s_z,+\rvert+\langle s_z,-\rvert)\lvert s_z,+\rangle}^2=\frac{1}{2}.
            \end{align}
            这与利用 (a) 中结论得到的
            \begin{align}
                P(s_x=\hbar/2)=\frac{1}{2}[1+\sin 0\cos(\omega t)]=\frac{1}{2}
            \end{align}
            一致.
            $S_x$ 的期待值为
            \begin{align}
                \langle S_x\rangle=\langle s_z,+\rvert S_x\lvert s_z,+\rangle=\langle s_z,+\rvert\frac{\hbar}{2}(\lvert s_z,+\rangle\langle s_z,-\rvert+\lvert s_z,-\rangle\langle s_z,+\rvert)\lvert s_z,+\rangle=0.
            \end{align}
            这与利用 (b) 中结论得到的
            \begin{align}
                \langle S_x\rangle=\frac{\hbar}{2}\sin 0\cos(\omega t)=0
            \end{align}
            一致.
            \item[(ii)] 当 $\beta=\pi/2$, 则 $t=0$ 时电子的状态为
            \begin{align}
                \lvert\alpha,t=0\rangle=\lvert s_x,+\rangle=\frac{1}{\sqrt{2}}(\lvert s_z,+\rangle+\lvert s_z,-\rangle).
            \end{align}
            $t$ 时刻电子的状态为
            \begin{align}
                \notag\lvert\alpha,t\rangle=&U(t,0)\lvert\alpha,t=0\rangle=\exp(-i\omega S_zt/\hbar)\frac{1}{\sqrt{2}}(\lvert s_z,+\rangle+\lvert s_z,-\rangle)\\
                =&\frac{1}{\sqrt{2}}[\exp(-i\omega t/2)\lvert s_z,+\rangle+\exp(i\omega t/2)\lvert s_z,-\rangle]
            \end{align}
            电子处在 $s_x=\hbar/2$ 态上的概率为
            \begin{align}
                \notag P(s_x=\hbar/2)=&\abs{\langle s_x=\hbar/2\vert\alpha,t\rangle}^2\\
                \notag=&\abs{\frac{1}{\sqrt{2}}(\langle s_z,+\rvert+\langle s_z,-\rvert)\frac{1}{\sqrt{2}}[\exp(-i\omega t/2)\lvert s_z,+\rangle+\exp(i\omega t/2)\lvert s_z,-\rangle]}^2\\
                =&\cos^2(\omega t/2).
            \end{align}
            这与利用 (a) 中结论得到的
            \begin{align}
                P(s_x=\hbar/2)=\frac{1}{2}\left[1+\sin\frac{\pi}{2}\cos(\omega t)\right]=\cos^2(\omega t/2)
            \end{align}
            一致. $S_x$ 的期待值为
            \begin{align}
                \notag\langle S_x\rangle=&\langle\alpha,t\rvert S_x\lvert\alpha,t\rangle\\
                \notag=&\frac{1}{\sqrt{2}}[\exp(i\omega t/2)\langle s_z,+\rvert+\exp(-i\omega t/2)\langle s_z,-\rvert]\frac{\hbar}{2}(\lvert s_z,+\rangle\langle s_z,-\rvert+\lvert s_z,-\rangle\langle s_z,+\rvert)\times\\
                \notag&\frac{1}{\sqrt{2}}[\exp(-i\omega t/2)\lvert s_z,+\rangle+\exp(i\omega /2)\lvert s_z,-\rangle]\\
                =&\frac{\hbar}{2}\cos(\omega t).
            \end{align}
            这与利用 (b) 中结论得到的
            \begin{align}
                \langle S_x\rangle=\frac{\hbar}{2}\sin\frac{\pi}{2}\cos(\omega t)=\frac{\hbar}{2}\cos(\omega t)
            \end{align}
            一致.
        \end{itemize}
        综上, (a) 和 (b) 中得到的结论应当是可靠的.
    \end{itemize}
\end{sol}

\begin{prob}[课本习题 2.6]
    考虑一个一维粒子, 其哈密顿量由下式给出
    \[
        H=\frac{p^2}{2m}+V(x).
    \]
    通过计算 $[[H,x],x]$, 证明
    \[
        \sum_{a'}\abs{\langle a''\rvert x\lvert a'\rangle}^2(E_{a'}-E_{a''})=\frac{\hbar^2}{2m},
    \]
    其中 $\lvert a'\rangle$ 是一个能量本征态, 本征值为 $E_{a'}$.
\end{prob}
\begin{pf}
    \begin{align}
        [H,x]=[\frac{p^2}{2m}+V(x),x]=[\frac{p^2}{2m},x]+[V(x),x]=\frac{1}{2m}[p^2,x]=\frac{1}{2m}(p[p,x]+[p,x]p)=-i\frac{\hbar}{m}p.
    \end{align}
    \begin{align}
        [[H,x],x]=[-i\frac{\hbar}{m}p,x]=-\frac{\hbar^2}{m}.
    \end{align}
    一方面,
    \begin{align}
        \notag\sum_{a'}\abs{\langle a''\rvert x\lvert a'\rangle}^2(E_{a'}-E_{a''})=&\sum_{a'}[\langle a''\rvert x\lvert a'\rangle E_{a'}\langle a'\rvert x\lvert a''\rangle-\langle a''\rvert x\lvert a'\rangle E_{a''}\langle a'\rvert x\lvert a''\rangle]\\
        \notag=&\sum_{a'}[\langle a''\rvert xH\lvert a'\rangle\langle a'\rvert x\lvert a''\rangle-\langle a''\rvert Hx\lvert a'\rangle\langle a'\rvert x\lvert a''\rangle]\\
        \notag=&\sum_{a'}\langle a''\rvert(xH-Hx)\lvert a'\rangle\langle a'\rvert x\lvert a''\rangle\\
        =&\langle a''\rvert[x,H]x\lvert a''\rangle,
    \end{align}
    另一方面,
    \begin{align}
        \notag\sum_{a'}\abs{\langle a''\rvert x\lvert a'\rangle}^2(E_{a'}-E_{a''})=&\sum_{a'}[\langle a''\rvert x\lvert a'\rangle E_{a'}\langle a'\rvert x\lvert a''\rangle-\langle a''\rvert x\lvert a'\rangle E_{a''}\langle a'\rvert x\lvert a''\rangle]\\
        \notag=&\sum_{a'}[\langle a''\rvert x\lvert a'\rangle\langle a'\rvert Hx\lvert a''\rangle-\langle a''\rvert x\lvert a'\rangle\langle a'\rvert xH\lvert a''\rangle]\\
        \notag=&\sum_{a'}\langle a''\rvert x\lvert a'\rangle\langle a'\rvert(Hx-xH)\lvert a''\rangle\\
        =&\langle a''\rvert x[H,x]\lvert a''\rangle.
    \end{align}
    以上两式相加得
    \begin{align}
        \notag 2\sum_{a'}\abs{\langle a''\rvert x\lvert a'\rangle}^2(E_{a'}-E_{a''})=&\langle a''\rvert[x,H]x\lvert a''\rangle+\langle a''\rvert x[H,x]\lvert a''\rangle\\
        \notag=&-\langle a''\rvert([H,x]x-x[H,x])\lvert a''\rangle\\
        \notag=&-\langle a''\rvert[[H,x],x]\lvert a''\rangle\\
        =&\frac{\hbar^2}{m},\\
        \Longrightarrow\sum_{a'}\abs{\langle a''\rvert x\lvert a'\rangle}^2(E_{a'}-E_{a''})=&\frac{\hbar^2}{2m}.
    \end{align}
\end{pf}

\begin{prob}[课本习题 2.8]
    考虑一个一维自由粒子的波包. $t=0$ 时它满足最小不确定度关系
    \[
        \langle(\Delta x)^2\rangle\langle(\Delta p)^2\rangle=\frac{\hbar^2}{4}\quad(t=0).
    \]
    此外, 我们知道
    \[
        \langle x\rangle=\langle p\rangle=0\quad(t=0).
    \]
    利用海森堡绘景, 当 $\langle(\Delta x)^2\rangle_{t=0}$ 给定时, 求作为 $t$ $(t\geq 0)$ 的函数的 $\langle(\Delta x)^2\rangle_t$. (提示: 利用你在第 1 章习题 1.18 中得到的不确定度波包的性质.)
\end{prob}
\begin{sol}
    一维自由粒子的哈密顿量为
    \begin{align}
        H=\frac{p^2}{2m}.
    \end{align}
    对应的时间演化算符为
    \begin{align}
        U(t,0)=\exp\left(\frac{-ip^2t}{2m\hbar}\right).
    \end{align}
    从而
    \begin{align}
        \notag\langle(\Delta x)^2\rangle_t=&\langle x^2\rangle_t-\langle x\rangle_t^2\\
        =&\langle\alpha,t=0\rvert U^{\dagger}(t,0)x^2U(t,0)\lvert\alpha,t=0\rangle-\langle\alpha,t=0\rvert U^{\dagger}(t,0)xU(t,0)\lvert\alpha,t=0\rangle^2.
    \end{align}
    利用 Baker-Campbell-Hausdorff 公式 $e^XYe^{-X}=Y+[X,Y]+\frac{1}{2!}[X,[X,Y]]+\frac{1}{3!}[x,[X,[X,Y]]]+\cdots$, 有
    \begin{align}
        \notag U^{\dagger}(t,0)x^2U(t,0)=&\exp\left(\frac{ip^2t}{2m}\right)x^2\exp\left(\frac{-ip^2t}{2m}\right)\\
        \notag=&x^2+[\frac{ip^2t}{2m},x^2]+\frac{1}{2!}[\frac{ip^2t}{2m},[\frac{ip^2t}{2m},x^2]]+\frac{1}{3!}[\frac{ip^2t}{2m},[\frac{ip^2t}{2m},[\frac{ip^2t}{2m},x^2]]]+\cdots\\
        =&x^2+\frac{t}{m}(px+xp)+\left(\frac{t}{m}\right)^2p^2,\\
        \notag U^{\dagger}(t,0)xU(t,0)=&\exp\left(\frac{ip^2t}{2m}\right)x\exp\left(\frac{-ip^2t}{2m}\right)\\
        \notag=&x+[\frac{ip^2t}{2m},x]+\frac{1}{2!}[\frac{ip^2t}{2m},[\frac{ip^2t}{2m},x]]+\frac{1}{3!}[\frac{ip^2t}{2m},[\frac{ip^2t}{2m},[\frac{ip^2t}{2m},x]]]+\cdots\\
        =&x+\frac{t}{m}p.
    \end{align}
    故
    \begin{align}
        \notag\langle(\Delta x)^2\rangle_t=&\langle\alpha,t=0\rvert\left[x^2+\frac{t}{m}(px+xp)+\left(\frac{t}{m}\right)^2p^2\right]\lvert\alpha,t=0\rangle-\langle\alpha,t=0\rvert\left(x+\frac{t}{m}p\right)\lvert\alpha,t=0\rangle\\
        \notag=&\langle x^2\rangle_{t=0}+\frac{t}{m}\langle px+xp\rangle_{t=0}+\left(\frac{t}{m}\right)^2\langle p^2\rangle_{t=0}-\left[\langle x\rangle_{t=0}+\frac{t}{m}\langle p\rangle_{t=0}\right]^2\\
        =&\langle x^2\rangle_{t=0}+\frac{t}{m}\langle px+xp\rangle_{t=0}+\left(\frac{t}{m}\right)^2\langle p^2\rangle_{t=0}.
    \end{align}
    由于该粒子在 $t=0$ 时满足最小不确定度关系, 有
    \begin{gather}
        \langle p^2\rangle_{t=0}=\langle(p-\langle p\rangle_{t=0})^2\rangle=\langle(\Delta p)^2\rangle=\frac{\hbar^2}{4}\frac{1}{\langle(\Delta x)^2\rangle_{t=0}}.
    \end{gather}
    及
    \begin{gather}
        \langle(\Delta x)^2\rangle_{t=0}\langle(\Delta p)^2\rangle_{t=0}=\frac{1}{4}\abs{\langle[\Delta x,\Delta p]\rangle_{t=0}}^2+\frac{1}{4}\abs{\langle\{\Delta x,\Delta p\}\rangle_{t=0}}^2=\frac{\hbar^2}{4}+\frac{1}{4}\abs{\langle\{\Delta x,\Delta p\}\rangle_{t=0}}^2=\frac{\hbar^2}{4},\\
        \Longrightarrow\langle\{\Delta x,\Delta p\}\rangle_{t=0}=\langle xp+px\rangle_{t=0}=0.
    \end{gather}
    故
    \begin{align}
        \langle(\Delta x)^2\rangle_t=\langle(\Delta x)^2\rangle_{t=0}+\frac{\hbar^2t^2}{4m^2}\frac{1}{\langle(\Delta x)^2\rangle_{t=0}}.
    \end{align}
\end{sol}
\end{document}