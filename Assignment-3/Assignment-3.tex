\documentclass{assignment}
\ProjectInfos{高等量子力学}{PHYS5001P}{2022-2023 学年第一学期}{第三次作业}{截止时间: 2022 年 10 月 8 日 (周六)}{陈稼霖}[https://github.com/Chen-Jialin]{SA21038052}

\begin{document}
\begin{prob}[课本习题 2.9]
    设 $\lvert a'\rangle$ 和 $\lvert a''\rangle$ 是厄米算符 $A$ 的本征态, 本征值分别为 $a'$ 和 $a''$ ($a'\neq a''$), 哈密顿量算符由下式给出
    \[
        H=\lvert a'\rangle\delta\langle a''\rvert+\lvert a''\rangle\delta\langle a'\rvert,
    \]
    其中 $\delta$ 只是个实数.
    \begin{itemize}
        \item[(a)] 显然, $\lvert a'\rangle$ 和 $\lvert a''\rangle$ 不是这个哈密顿量的本征态, 写出该哈密顿量的本征态, 他们的能量本征值是什么?
        \item[(b)] 如果已知 $t=0$ 时刻系统处在 $\lvert a'\rangle$ 态上, 在薛定谔绘景中写出 $t>0$ 时的态矢量.
        \item[(c)] 如果已知 $t=0$ 时系统处在 $\lvert a'\rangle$ 态上, 在 $t>0$ 时找到该系统在 $\lvert a''\rangle$ 态的概率是多少?
        \item[(d)] 你能想出与这个问题对应的一种物理情况吗?
    \end{itemize}
\end{prob}
\begin{sol}
    \begin{itemize}
        \item[(a)] 以 $\{\lvert a'\rangle,\lvert a''\rangle\}$ 为基, 该哈密顿量的矩阵表示为
        \begin{align}
            H=\begin{bmatrix}
                0&\delta\\
                \delta&0
            \end{bmatrix}.
        \end{align}
        利用
        \begin{align}
            \abs{H-EI}=\begin{vmatrix}
                -E&\delta\\
                \delta&-E
            \end{vmatrix}=E^2-\delta^2=0,
        \end{align}
        解得能量本征值为
        \begin{align}
            E_1=\delta,\quad E_2=-\delta.
        \end{align}
        将这两个能量本征值代入
        \begin{align}
            H\lvert E_{1/2}\rangle=E_{1/2}\lvert E_{1/2}\rangle
        \end{align}
        解得其对应的本征态分别为
        \begin{align}
            \lvert E_1\rangle=\frac{1}{\sqrt{2}}\begin{bmatrix}
                1\\
                1
            \end{bmatrix}=\frac{1}{\sqrt{2}}(\lvert a'\rangle+\lvert a''\rangle),\quad\lvert E_2\rangle=\frac{1}{\sqrt{2}}\begin{bmatrix}
                1\\
                -1
            \end{bmatrix}=\frac{1}{\sqrt{2}}(\lvert a'\rangle-\lvert a''\rangle).
        \end{align}
        \item[(b)] 该哈密顿量对应的时间演化算符为
        \begin{align}
            U(t,0)=\exp(-iHt/\hbar)=\exp(-i\delta t/\hbar)\lvert E_1\rangle\langle E_1\rvert+\exp(i\delta t/\hbar)\lvert E_2\rangle\langle E_2\rvert.
        \end{align}
        在薛定谔绘景中, $t>0$ 时的态矢量为
        \begin{align}
            \notag\lvert a',0;t\rangle=&U(t,0)\lvert a'\rangle\\
            \notag=&[\exp(-i\delta t/\hbar)\lvert E_1\rangle\langle E_1\rvert+\exp(i\delta t/\hbar)\lvert E_2\rangle\langle E_2\rvert]\frac{1}{\sqrt{2}}(\lvert E_1\rangle+\lvert E_2\rangle)\\
            \notag=&\frac{1}{\sqrt{2}}[\exp(-i\delta t/\hbar)\lvert E_1\rangle+\exp(i\delta t/\hbar)\lvert E_2\rangle]\\
            =&\cos(\delta t/\hbar)\lvert a'\rangle-i\sin(\delta t/\hbar)\lvert a''\rangle.
        \end{align}
        \item[(c)] 在 $t>0$ 时找到该系统在 $\lvert a''\rangle$ 态的概率为
        \begin{align}
            P(a'')=\abs{\langle a''\vert a',0;t\rangle}^2=\sin^2(\delta t/\hbar).
        \end{align}
        \item[(d)] 以磁场中的电子自旋为例. 假设磁场沿 $z$ 轴方向, $\bm{B}=B\hat{\bm{z}}$. $\lvert a'\rangle=\lvert s_x,+\rangle$ 和 $\lvert a''\rangle=\lvert s_x,-\rangle$ 为厄米算符 $S_x$ 的本征态. 哈密顿量算符可表为
        \begin{align}
            \notag H=&-\left(\frac{eB}{mc}\right)S_z\\
            \notag=&-\frac{e\hbar B}{2mc}\lvert s_z,+\rangle\langle s_z,+\rvert+\frac{e\hbar B}{2mc}\lvert s_z,-\rangle\langle s_z,-\rvert\\
            \notag=&-\frac{e\hbar B}{2mc}\frac{1}{\sqrt{2}}(\lvert s_x,+\rangle+\lvert s_x,-\rangle)\frac{1}{\sqrt{2}}(\langle s_x,+\rvert+\langle s_x,-\rvert)+\frac{e\hbar B}{2mc}\frac{1}{\sqrt{2}}(\lvert s_x,+\rangle-\lvert s_x,-\rangle)\frac{1}{\sqrt{2}}(\langle s_x,+\rvert-\langle s_x,-\rvert)\\
            \notag=&\delta\lvert s_x,+\rangle\langle s_x,-\rvert+\delta\lvert s_x,-\rangle\langle s_x,+\rvert,
        \end{align}
        其中 $\delta=-\frac{e\hbar B}{2mc}$. 由 (b) 和 (c) 中得到的结论, 如果 $t=0$ 时刻电子处在 $\lvert s_x,+\rangle$ 态上, 则其在 $t>0$ 时的状态为
        \begin{align}
            \lvert s_x,+,0;t\rangle=\cos(\delta t/\hbar)\lvert s_x,+\rangle-i\sin(\delta t/\hbar)\lvert s_x,-\rangle.
        \end{align}
        此时找到电子在 $\lvert s_x,-\rangle$ 态的概率为
        \begin{align}
            P(s_x=-\hbar/2)=\sin^2(\delta t/\hbar).
        \end{align}
    \end{itemize}
\end{sol}

\begin{prob}[课本习题 2.10]
    含有一个粒子的一个盒子用一个薄的隔板分成左右两个隔间. 如果已知该粒子确定无疑地处在右 (左) 边, 则状态用 $\lvert R\rangle$ ($\lvert L\rangle$) 表示, 在那里我们忽略了在盒子的每一半中的空间变量, 然后, 最一般的态矢量可以写成
    \[
        \lvert\alpha\rangle=\lvert R\rangle\langle R\vert\alpha\rangle+\lvert L\rangle\langle L\vert\alpha\rvert,
    \]
    其中 $\langle R\vert\alpha\rangle$ 和 $\langle L\vert\alpha\rangle$ 可以看作 ``波函数'', 该粒子可以隧穿过隔板; 这种隧道效应由哈密顿量
    \[
        H=\Delta(\lvert L\rangle\langle R\rvert+\lvert R\rangle\langle L\rvert)
    \]
    表征, 其中的 $\Delta$ 是个有能量量纲的实数.
    \begin{itemize}
        \item[(a)] 求归一化的能量本征右矢, 相应的能量本征值是什么?
        \item[(b)] 在薛定谔绘景中基右矢 $\lvert R\rangle$ 和 $\lvert L\rangle$ 是固定不变的, 而态矢量随时间运动. 假定系统就由上面在 $t=0$ 时给定的 $\lvert\alpha\rangle$ 表示. 通过用适当的时间演化算符作用于 $\lvert\alpha\rangle$, 求 $t>0$ 时的态矢量 $\lvert\alpha,t_0;t\rangle$.
        \item[(c)] 假定 $t=0$ 时粒子确定无疑地处在右边, 观测到粒子在左边的、作为时间函数的概率是多少?
        \item[(d)] 写出波函数 $\langle R\vert\alpha,t_0;t\rangle$ 和 $\langle L\vert\alpha,t_0;t\rangle$ 的耦合薛定谔方程, 证明该耦合薛定谔方程的解正是你在 (b) 中所预期的.
        \item[(e)] 假定打印机出了个错, 把 $H$ 写成了
        \[
            H=\Delta\lvert L\rangle\langle R\rvert.
        \]
        通过明显地求解具有该哈密顿量的最普遍的时间演化问题, 证明概率守恒被破坏了.
    \end{itemize}
\end{prob}
\begin{sol}
    \begin{itemize}
        \item[(a)] 以 $\{\lvert R\rangle,\lvert L\rangle\}$ 为基, 该哈密顿量的矩阵表示为
        \begin{align}
            H=\begin{bmatrix}
                0&\Delta\\
                \Delta&0
            \end{bmatrix}.
        \end{align}
        利用
        \begin{align}
            \abs{H-EI}=\begin{vmatrix}
                -E&\Delta\\
                \Delta&-E
            \end{vmatrix}=E^2-\Delta^2=0,
        \end{align}
        解得能量本征值为
        \begin{align}
            E_1=\Delta,\quad E_2=-\Delta.
        \end{align}
        将这两个能量本征值代入
        \begin{align}
            H\lvert E_{1/2}\rangle=E_{1/2}\lvert E_{1/2}\rangle
        \end{align}
        解得其对应的归一的能量本征右矢分别为
        \begin{align}
            \lvert E_1\rangle=\frac{1}{\sqrt{2}}\begin{bmatrix}
                1\\
                1
            \end{bmatrix}=\frac{1}{\sqrt{2}}(\lvert R\rangle+\lvert L\rangle),\quad\lvert E_2\rangle=\frac{1}{\sqrt{2}}\begin{bmatrix}
                1\\
                -1
            \end{bmatrix}=\frac{1}{\sqrt{2}}(\lvert R\rangle-\lvert L\rangle).
        \end{align}
        \item[(b)] 该哈密顿量对应的时间演化算符为
        \begin{align}
            \notag U(t,t_0)=&\exp[-iH(t-t_0)/\hbar]\\
            \notag=&\exp[-i\Delta(t-t_0)/\hbar]\lvert E_1\rangle\langle E_1\rvert+\exp[i\Delta(t-t_0)/\hbar]\lvert E_2\rangle\langle E_2\rvert\\
            \notag=&\exp[-i\Delta(t-t_0)/\hbar]\frac{1}{\sqrt{2}}(\lvert R\rangle+\lvert L\rangle)\frac{1}{\sqrt{2}}(\langle R\rvert+\langle L\rvert)+\exp[i\Delta(t-t_0)/\hbar]\frac{1}{\sqrt{2}}(\lvert R\rangle-\lvert L\rangle)\frac{1}{\sqrt{2}}(\langle R\rvert-\langle L\rvert)\\
            =&\cos[\Delta(t-t_0)/\hbar](\lvert R\rangle\langle R\rvert+\lvert L\rangle\langle L\rvert)-i\sin[\Delta(t-t_0)/\hbar](\lvert R\rangle\langle L\rvert+\lvert L\rangle\langle R\rvert).
        \end{align}
        $t>0$ 时的态矢量为
        \begin{align}
            \notag\lvert\alpha,t_0=0;t\rangle=&U(t,t_0=0)\lvert\alpha,t_0=0;t\rangle\\
            \notag=&[\cos(\Delta t/\hbar)(\lvert R\rangle\langle R\rvert+\lvert L\rangle\langle L\rvert)-i\sin(\Delta t/\hbar)(\lvert R\rangle\langle L\rvert+\lvert L\rangle\langle R\rvert)](\lvert R\rangle\langle R\vert\alpha\rangle+\lvert L\rangle\langle L\vert\alpha\rvert)\\
            =&[\langle R\vert\alpha\rangle\cos(\Delta t/\hbar)-i\langle L\vert\alpha\rangle\sin(\Delta t/\hbar)]\lvert R\rangle+[\langle L\vert\alpha\rangle\cos(\Delta t/\hbar)-i\langle R\vert\alpha\rangle\sin(\Delta t/\hbar)]\lvert L\rangle.
        \end{align}
        \item[(c)] 由 (b) 中得到的结论, 若 $\lvert\alpha,t_0=0\rangle=\lvert R\rangle$, 则 $t>0$ 时的态矢量为
        \begin{align}
            \lvert\alpha,t_0=0;t\rangle=\cos(\Delta t/\hbar)\lvert R\rangle-i\sin(\Delta t/\hbar)\lvert L\rangle.
        \end{align}
        观测到粒子在左边的概率为
        \begin{align}
            P(L)=\abs{\langle L\vert\alpha\rangle}^2=\sin^2(\Delta t/\hbar).
        \end{align}
        \item[(d)] 波函数 $\langle R\vert\alpha,t_0;t\rangle$ 和 $\langle L\vert\alpha,t_0;t\rangle$ 的耦合薛定谔方程分别为
        \begin{align}
            i\hbar\frac{\partial}{\partial t}\langle R\vert\alpha,t_0;t\rangle=&\langle R\rvert H\lvert\alpha,t_0;t\rangle,\\
            i\hbar\frac{\partial}{\partial t}\langle L\vert\alpha,t_0;t\rangle=&\langle L\rvert H\lvert\alpha,t_0;t\rangle,
        \end{align}
        代入哈密顿量的具体形式 $H=\Delta(\lvert L\rangle\langle R\rvert+\lvert R\rangle\langle L\rvert)$, 有
        \begin{align}
            i\hbar\frac{\partial}{\partial t}\langle R\vert\alpha,t_0l;t\rangle=&\Delta\langle L\vert\alpha,t_0;t\rangle,\\
            i\hbar\frac{\partial}{\partial t}\langle L\vert\alpha,t_0;t\rangle=&\Delta\langle R\vert\alpha,t_0;t\rangle.
        \end{align}
        将 (b) 中得到的 $\lvert\alpha,t_0;t\rangle$ 代入可得
        \begin{align}
            \notag\text{方程左边}=&i\hbar\frac{\partial}{\partial t}\langle R\vert\alpha,t_0;t\rangle\\
            \notag=&i\hbar\frac{\partial}{\partial t}[\langle R\vert\alpha\rangle\cos(\Delta t/\hbar)-i\langle L\vert\alpha\rangle\sin(\Delta t/\hbar)]\\
            \notag=&\Delta[-i\langle R\vert\alpha\rangle\sin(\Delta t/\hbar)+\langle L\vert\alpha\rangle\cos(\Delta t/\hbar)]\\
            =&\Delta\langle L\vert\alpha,t_0;t\rangle=\text{方程右边},\\
            \notag\text{方程左边}=&i\hbar\frac{\partial}{\partial t}\langle R\vert\alpha,t_0;t\rangle\\
            \notag=&i\hbar\frac{\partial}{\partial t}[\langle L\vert\alpha\rangle\cos(\Delta t/\hbar)-i\langle R\vert\alpha\rangle\sin(\Delta t/\hbar)]\\
            \notag=&\Delta[-i\langle L\vert\alpha\rangle\sin(\Delta t/\hbar)+\langle R\vert\alpha\rangle\cos(\Delta t/\hbar)]\\
            =&\Delta\langle R\vert\alpha,t_0;t\rangle=\text{方程右边}.
        \end{align}
        故该耦合薛定谔方程的解正是在 (b) 中所预期的.
        \item[(e)] 该错误哈密顿量对应的时间演化算符为
        \begin{align}
            \notag U(t,t_0)=&\exp\left[\frac{-iH(t-t_0)}{\hbar}\right]\\
            \notag=&\sum_{n=0}^{\infty}\frac{1}{n!}\left[\frac{-iH(t-t_0)}{\hbar}\right]^n\\
            \notag&(\text{利用 }H^2=\Delta\lvert L\rangle\langle R\vert L\rangle\langle R\rvert)\\
            =&1-\frac{i\Delta(t-t_0)}{\hbar}\lvert L\rangle\langle R\rvert.
        \end{align}
        假设 $t=0$ 时,
        \begin{align}
            \langle\alpha\vert\alpha\rangle=\abs{\langle R\vert\alpha\rangle}^2+\abs{\langle L\vert\alpha\rangle}^2=1,
        \end{align}
        即在盒子中 (无论哪个隔间) 找到粒子的概率为 $1$.
        $t>0$ 时系统的态矢量为
        \begin{align}
            \notag\lvert\alpha,t_0=0;t\rangle=&U(t,0)\lvert\alpha\rangle\\
            \notag=&\left(1-\frac{i\Delta t}{\hbar}\lvert L\rangle\langle R\lvert\right)(\lvert R\rangle\langle R\vert\alpha\rangle+\lvert L\rangle\langle L\vert\alpha\rangle)\\
            =&\lvert R\rangle\langle R\vert\alpha\rangle+\lvert L\rangle\left(\langle L\vert\alpha\rangle-\frac{i\Delta t}{\hbar}\langle R\vert\alpha\rangle\right).
        \end{align}
        此时
        \begin{align}
            \notag\langle\alpha,t_0;t\vert\alpha,t_0;t\rangle=&\left[\langle\alpha\vert R\rangle\langle R\rvert+\left(\langle\alpha\vert L\rangle+\frac{i\Delta t}{\hbar}\langle\alpha\vert R\rangle\right)\langle L\rvert\right]\left[\lvert R\rangle\langle R\vert\alpha\rangle+\lvert L\rangle\left(\langle L\vert\alpha\rangle-\frac{i\Delta t}{\hbar}\langle R\vert\alpha\rangle\right)\right]\\
            \notag=&\abs{\langle R\vert\alpha\rangle}^2+\abs{\langle L\vert\alpha\rangle}^2+\left(\frac{\Delta t}{\hbar}\right)^2\abs{\langle R\vert\alpha\rangle}^2\\
            \notag=&1+\left(\frac{\Delta t}{\hbar}\right)^2\abs{\langle R\vert\alpha\rangle}^2\\
            \geq&1.
        \end{align}
        这说明概率守恒被破坏了.
    \end{itemize}
\end{sol}

\begin{prob}[课本习题 2.12]
    考虑一个处在一维简谐振子位势中的粒子. 假定在 $t=0$ 时态矢量为
    \[
        \exp\left(\frac{-ipa}{\hbar}\right)\lvert 0\rangle,
    \]
    其中 $p$ 是动量算符, 而 $a$ 是某个具有长度量纲的数. $\lvert 0\rangle$ 是这样一个态, 它使 $\langle x\rangle=0=\langle p\rangle$, 利用海森堡绘景求 $t\geq 0$ 时的期待值 $\langle x\rangle$.
\end{prob}
\begin{sol}
    简谐振子的哈密顿量为
    \begin{align}
        H=\frac{p^2}{2m}+\frac{m\omega^2x^2}{2}.
    \end{align}
    在海森堡绘景中, 位置算符和动量算符的运动方程分别为
    \begin{align}
        \frac{\partial x}{\partial t}=&\frac{1}{i\hbar}[x,H]=\frac{1}{i\hbar}[x,\frac{p^2}{2m}+\frac{m\omega^2x^2}{2}]=\frac{1}{2im\hbar}(p[x,p]+[x,p]p)=\frac{p}{m},\\
        \frac{\partial p}{\partial t}=&\frac{1}{i\hbar}[p,H]=\frac{1}{i\hbar}[p,\frac{p^2}{2m}+\frac{m\omega^2x^2}{2}]=\frac{m\omega^2}{2i\hbar}(x[p,x]+[p,x]x)=-m\omega^2x,
    \end{align}
    联立这两个耦合的运动方程解得 $t\geq 0$ 时的位置算符为
    \begin{align}
        x(t)=x(0)\cos(\omega t)+\frac{p(0)}{m\omega}\sin(\omega t).
    \end{align}
    $t\geq 0$ 时的位置期待值为
    \begin{align}
        \notag\langle x(t)\rangle=&\langle 0\rvert\exp\left(\frac{ipa}{\hbar}\right)x(t)\exp\left(\frac{-ipa}{\hbar}\right)\lvert 0\rangle\\
        \notag=&\langle 0\rvert\exp\left(\frac{ipa}{\hbar}\right)x(0)\exp\left(\frac{-ipa}{\hbar}\right)\lvert 0\rangle\cos(\omega t)+\langle 0\rvert\exp\left(\frac{ipa}{\hbar}\right)p(0)\exp\left(\frac{-ipa}{\hbar}\right)\lvert 0\rangle\\
        \notag&(\text{利用 Baker–Campbell–Hausdorff 公式})\\
        \notag=&\langle 0\rvert[x(0)+a]\lvert 0\rangle\cos(\omega t)\\
        =&a\cos(\omega t).
    \end{align}
\end{sol}

\begin{prob}[课本习题 2.14]
    考虑一个一维简谐振子,
    \begin{itemize}
        \item[(a)] 利用
        \[
            \left.\begin{array}{l}
                a\\
                a^{\dagger}
            \end{array}\right\}=\sqrt{\frac{m\omega}{2\hbar}}\left(x\pm\frac{ip}{m\omega}\right),\quad\left.\begin{array}{l}
                a\lvert n\rangle\\
                a^{\dagger}\lvert n\rangle
            \end{array}\right\}=\left\{\begin{array}{l}
                \sqrt{n}\lvert n-1\rangle\\
                \sqrt{n+1}\lvert n+1\rangle
            \end{array}\right.,
        \]
        计算 $\langle m\rvert x\lvert n\rangle$, $\langle m\rvert p\lvert n\rangle$, $\langle m\rvert x^2\lvert n\rangle$ 和 $\langle m\rvert p^2\lvert n\rangle$.
        \item[(b)] 维里定理可表述为
        \[
            \langle\frac{\bm{p}^2}{m}\rangle=\langle\bm{x}\cdot\nabla V\rangle,\quad\text{三维时},\quad\text{或}\quad\langle\frac{p^2}{m}\rangle=\langle x\frac{\mathrm{d}V}{\mathrm{d}x}\rangle\quad\text{一维时},
        \]
        检验对于动能和势能在一个能量本征态上的期待值, 维里定理成立. (按勘误表要求, 这里给出了维里定理. --- 译者注)
    \end{itemize}
\end{prob}
\begin{sol}
    \begin{itemize}
        \item[(a)] 利用
        \[
            \left.\begin{array}{l}
                a\\
                a^{\dagger}
            \end{array}\right\}=\sqrt{\frac{m\omega}{\hbar}}\left(x\pm\frac{ip}{m\omega}\right),
        \]
        有
        \begin{align}
            x=&\sqrt{\frac{\hbar}{2m\omega}}(a+a^{\dagger}),\\
            p=&i\sqrt{\frac{m\hbar\omega}{2}}(-a+a^{\dagger}).
        \end{align}
        从而
        \begin{align}
            \langle m\rvert x\lvert n\rangle=&\sqrt{\frac{\hbar}{2m\omega}}(\langle m\rvert a\lvert n\rangle+\langle m\rvert a^{\dagger}\lvert n\rangle)=\sqrt{\frac{\hbar}{2m\omega}}(\sqrt{n}\delta_{m,n-1}+\sqrt{n+1}\delta_{m,n+1}),\\
            \langle m\rvert p\lvert n\rangle=&i\sqrt{\frac{m\hbar\omega}{2}}(-\langle m\rvert a\lvert n\rangle+\langle m\rvert a^{\dagger}\lvert n\rangle)=i\sqrt{\frac{m\hbar\omega}{2}}(-\sqrt{n}\delta_{m,n-1}+\sqrt{n-1}\delta_{m,n-1}+i\sqrt{n+1}\delta_{m,n+1})\\
            \notag\langle m\rvert x^2\lvert n\rangle=&\frac{\hbar}{2m\omega}\langle m\rvert(a+a^{\dagger})^2\lvert n\rangle=\frac{\hbar}{2m\omega}[\langle m\rvert a^2\lvert n\rangle+\langle m\rvert aa^{\dagger}\lvert n\rangle+\langle m\rvert a^{\dagger}a\lvert n\rangle+\langle m\rvert(a^{\dagger})^2\lvert n\rangle]\\
            =&\frac{\hbar}{2m\omega}[\sqrt{n(n-1)}\delta_{m,n-2}+(2n+1)\delta_{m,n}+\sqrt{(n+1)(n+2)}\delta_{m,n+2}],\\
            \notag\langle m\rvert p^2\lvert n\rangle=&-\frac{m\hbar\omega}{2}\langle m\rvert(-a+a^{\dagger})^2\lvert n\rangle=-\frac{m\hbar\omega}{2}[\langle m\rvert a^2\lvert n\rangle-\langle m\rvert aa^{\dagger}\lvert n\rangle-\langle m\rvert a^{\dagger}a\lvert n\rangle+\langle m\rvert(a^{\dagger})^2\lvert n\rangle]\\
            =&-\frac{m\hbar\omega}{2}[\sqrt{n(n-1)}\delta_{m,n-2}-(2n+1)\delta_{m,n}+\sqrt{(n+1)(n+2)}\delta_{m,n+2}].
        \end{align}
        \item[(b)] 一维时, 对于能量本征态 $\lvert n\rangle$,
        \begin{align}
            \langle\frac{p^2}{m}\rangle=&-\frac{\hbar\omega}{2}\langle n\rvert p^2\lvert n\rangle=\hbar\omega\left(n+\frac{1}{2}\right),\\
            \langle x\frac{\mathrm{d}V}{\mathrm{d}x}\rangle=&\langle x\frac{\mathrm{d}(m\omega^2x^2/2)}{\mathrm{d}x}\rangle=\langle m\omega^2x^2\rangle=m\omega^2\langle n\rvert x^2\lvert n\rangle=\hbar\omega\left(n+\frac{1}{2}\right)\\
            \Longrightarrow\langle\frac{p^2}{m}\rangle=&\langle x\frac{\mathrm{d}V}{\mathrm{d}x}\rangle.
        \end{align}
        三维时, 对于能量本征态 $\lvert n_x,n_y,n_z\rangle$,
        \begin{align}
            \notag\langle\frac{\bm{p}^2}{m}\rangle=&\frac{1}{m}\langle n_x,n_y,n_z\rvert(p_x^2+p_y^2+p_z^2)\lvert n_x,n_y,n_z\rangle=\frac{1}{m}[\langle n_x\rvert p_x^2\lvert n_x\rangle+\langle n_y\rvert p_y^2\lvert n_y\rangle+\langle n_z\rvert p_z^2\lvert n_z\rangle]\\
            =&\hbar\omega_x\left(n_x+\frac{1}{2}\right)+\hbar\omega_y\left(n_y+\frac{1}{2}\right)+\hbar\omega_z\left(n_z+\frac{1}{2}\right),\\
            \notag\langle\bm{x}\cdot\nabla V\rangle=&\langle x\frac{\mathrm{d}(m\omega_x^2x^2/2)}{\mathrm{d}x}+y\frac{\mathrm{d}(m\omega_y^2y^2/2)}{\mathrm{d}y}+z\frac{\mathrm{d}(m\omega_z^2z^2/2)}{\mathrm{d}z}\rangle\\
            \notag=&\langle n_x,n_y,n_z\rvert m\omega_x^2x^2+m\omega_y^2y^2+m\omega_z^2z^2\lvert n_x,n_y,n_z\rangle\\
            \notag=&m\omega_x^2\langle n_x\rvert x^2\lvert n_x\rangle+m\omega_y^2\langle n_y\rvert y^2\lvert n_y\rangle+m\omega_z^2\langle n_z\rvert z^2\lvert n_z\rangle\\
            =&\hbar\omega_x\left(n_x+\frac{1}{2}\right)+\hbar\omega_y\left(n_y+\frac{1}{2}\right)+\hbar\omega_z\left(n_z+\frac{1}{2}\right),\\
            \Longrightarrow\langle\frac{\bm{p}^2}{m}\rangle=&\langle\bm{x}\cdot\nabla V\rangle.
        \end{align}
        故对于动能和势能在一个能量本征态上的期待值, 维里定理成立.
    \end{itemize}
\end{sol}

\begin{prob}[课本习题 2.22]
    考虑一个质量为 $m$ 的粒子处于下列形式中的一维势中:
    \[
        V(x)=\left\{\begin{array}{ll}
            \frac{1}{2}kx^2,&\text{对于 }x>0\\
            \infty,&\text{对于 }x<0.
        \end{array}\right.
    \]
    \begin{itemize}
        \item[(a)] 其基态能量是什么?
        \item[(b)] 对于该基态的期待值 $\langle x^2\rangle$ 是什么?
    \end{itemize}
\end{prob}
\begin{sol}
    \begin{itemize}
        \item[(a)] 该粒子的哈密顿量为
        \begin{align}
            H=\left\{\begin{array}{ll}
                \frac{p^2}{2m}+\frac{1}{2}kx^2,&\text{对于 }x>0\\
                \infty,&\text{对于 }x<0.
            \end{array}\right.
        \end{align}
        在 $x>0$ 处, 薛定谔方程为
        \begin{align}
            \left(\frac{\hbar^2}{2m}\frac{\partial^2}{\partial x^2}+\frac{1}{2}kx^2\right)\psi=E\psi,\quad\text{for }x>0,
        \end{align}
        这与谐振子的薛定谔方程相同, 因此该粒子在 $x>0$ 处的波函数应为谐振子波函数的线性叠加的形式.
        而在 $x<0$ 处, 由于势垒无穷高, 故
        \begin{align}
            \psi=0,\quad\text{for }x<0.
        \end{align}
        考虑到波函数的连续性, 该粒子的波函数需满足 $\psi(x=0)=0$.

        综合这两点来看, 该粒子的波函数在 $x>0$ 处应为谐振子的奇对称的波函数的线性叠加, 在 $x<0$ 处应 $=0$.
        因此, 该粒子的基态波函数在 $x>0$ 处应为谐振子标号 $n=1$ 的波函数 (的 $2$ 倍), 而在 $x<0$ 处应 $=0$:
        \begin{align}
            \psi(x)=\left\{\begin{array}{ll}
                \sqrt{2}\langle x\rvert 1\rangle,&\text{for }x>0\\
                0,&\text{for }x<0,
            \end{array}\right.
        \end{align}
        其中 $\lvert 1\rangle$ 为谐振子标号 $n=1$ 的态, $\langle x'\vert 1\rangle$ 为其对应的波函数.
        该粒子的基态能量为
        \begin{align}
            E=\int_{-\infty}^{\infty}\mathrm{d}x'\,\psi^*(x')H\psi(x')=\hbar\omega\langle 1\rvert\left(N+\frac{1}{2}\right)\lvert 1\rangle=\frac{3}{2}\hbar\omega=\frac{3}{2}\hbar\sqrt{\frac{k}{m}}.
        \end{align}
        \item[(b)] 该基态的期待值
        \begin{align}
            \langle x^2\rangle=\int_{-\infty}^{\infty}\mathrm{d}x'\,\psi^*(x')x'^2\psi(x')=\langle 1\rvert x^2\lvert 1\rangle=\frac{3}{2}\frac{\hbar}{m\omega}.
        \end{align}
        (此处利用了课本习题 2.14 (a) 中得到的结论.)
    \end{itemize}
\end{sol}

\begin{prob}[课本习题 2.24]
    考虑一维粒子被一个 $\delta$ 函数势
    \[
        V(x)=-\nu_0\delta(x),\quad(\nu_0\text{ 为正实数})
    \]
    束缚于一个固定的中心位置处, 求波函数和基态束缚能. 有激发的束缚态吗?
\end{prob}
\begin{prob}
    该粒子的哈密顿量为
    \begin{align}
        H=\frac{p^2}{2m}+V(x)=\left\{\begin{array}{ll}
            \frac{p^2}{2m},&x\neq 0,\\
            \frac{p^2}{2m}-\nu_0\delta(x),&x=0.
        \end{array}\right.
    \end{align}
    在 $x\neq 0$ 处的薛定谔方程为
    \begin{align}
        -\frac{\hbar^2}{2m}\frac{\partial^2\psi}{\partial x^2}=E\psi.
    \end{align}
    考虑到束缚态波函数在无穷远处的函数值为 $0$, $x<0$ 处和 $x>0$ 处波函数的通解分别为
    \begin{align}
        \psi(x<0)=Ae^{kx},\\
        \psi(x>0)=Be^{-kx},
    \end{align}
    其中 $k=\frac{\sqrt{2mE}}{\hbar}$.
    利用连续性条件
    \begin{gather}
        \psi(x=0^-)=A=\psi(x=0^+)=B,\\
        \psi'(x=0^+)-\psi'(x=0^-)=-Ak-Bk=\int_{0^-}^{0^+}\frac{\partial^2\psi}{\partial x'^2}\,\mathrm{d}x'=\frac{2m}{\hbar^2}\int_{0^-}^{0^+}[-\nu_0\delta(x)-E]\psi(0)\,\mathrm{d}x'=-\frac{2m\nu_0}{\hbar^2}A,
    \end{gather}
    解得
    \begin{align}
        A=B,\quad k=\frac{m\nu_0}{\hbar^2}.
    \end{align}
    考虑到波函数满足归一化条件, 得波函数
    \begin{align}
        \psi(x)=\frac{\sqrt{m\nu_0}}{\hbar}e^{-m\nu_0\abs{x}/\hbar^2},
    \end{align}
    基态束缚能为
    \begin{align}
        E=\frac{\hbar^2\Omega^2}{2m}.
    \end{align}

    该系统仅有一个束缚基态, 没有激发的束缚态
\end{prob}
\end{document}