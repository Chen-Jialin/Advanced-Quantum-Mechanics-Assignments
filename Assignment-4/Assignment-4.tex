\documentclass{assignment}
\ProjectInfos{高等量子力学}{PHYS5001P}{2022-2023 学年第一学期}{第四次作业}{截止时间: 2022 年  月  日 (周一)}{陈稼霖}[https://github.com/Chen-Jialin]{SA21038052}

\begin{document}
\begin{prob}[课本习题 2.24]
    考虑一维粒子被一个 $\delta$ 函数势
    \[
        V(x)=-\nu_0\delta(x),\quad(\nu_0\text{ 为正实数})
    \]
    束缚于一个固定的中心位置处, 求波函数和基态束缚能. 有激发的束缚态吗?
\end{prob}
\begin{prob}
    该粒子的哈密顿量为
    \begin{align}
        H=\frac{p^2}{2m}+V(x)=\left\{\begin{array}{ll}
            \frac{p^2}{2m},&x\neq 0,\\
            \frac{p^2}{2m}-\nu_0\delta(x),&x=0.
        \end{array}\right.
    \end{align}
    在 $x\neq 0$ 处的薛定谔方程为
    \begin{align}
        -\frac{\hbar^2}{2m}\frac{\partial^2\psi}{\partial x^2}=E\psi.
    \end{align}
    考虑到束缚态波函数在无穷远处的函数值为 $0$, $x<0$ 处和 $x>0$ 处波函数的通解分别为
    \begin{align}
        \psi(x<0)=Ae^{kx},\\
        \psi(x>0)=Be^{-kx},
    \end{align}
    其中 $k=\frac{\sqrt{2mE}}{\hbar}$.
    利用连续性条件
    \begin{gather}
        \psi(x=0^-)=A=\psi(x=0^+)=B,\\
        \psi'(x=0^+)-\psi'(x=0^-)=-Ak-Bk=\int_{0^-}^{0^+}\frac{\partial^2\psi}{\partial x'^2}\,\mathrm{d}x'=\frac{2m}{\hbar^2}\int_{0^-}^{0^+}[-\nu_0\delta(x)-E]\psi(0)\,\mathrm{d}x'=-\frac{2m\nu_0}{\hbar^2}A,
    \end{gather}
    解得
    \begin{align}
        A=B,\quad k=\frac{m\nu_0}{\hbar^2}.
    \end{align}
    考虑到波函数满足归一化条件, 得波函数
    \begin{align}
        \psi(x)=\frac{\sqrt{m\nu_0}}{\hbar}e^{-m\nu_0\abs{x}/\hbar^2},
    \end{align}
    基态束缚能为
    \begin{align}
        E=\frac{\hbar^2\Omega^2}{2m}.
    \end{align}

    该系统仅有一个束缚基态, 没有激发的束缚态
\end{prob}
\end{document}