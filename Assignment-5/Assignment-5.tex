\documentclass{assignment}
\ProjectInfos{高等量子力学}{PHYS5001P}{2022-2023 学年第一学期}{第五次作业}{截止时间: 2022 年 10 月 24 日 (周一)}{陈稼霖}[https://github.com/Chen-Jialin]{SA21038052}

\begin{document}
\begin{prob}[课本习题 3.1]
    求 $\bm{\sigma}_y=\begin{pmatrix}
        0&-i\\
        i&0
    \end{pmatrix}$ 的本征值和本征矢. 假定一个电子处在自旋态 $\begin{pmatrix}
        \alpha\\
        \beta
    \end{pmatrix}$ 上. 如果测得 $s_y$, 结果为 $\hbar/2$ 的概率是什么?
\end{prob}
\begin{sol}
    通过
    \begin{align}
        \abs{\sigma-\lambda I}=\begin{vmatrix}
            -\lambda&-i\\
            i&-\lambda
        \end{vmatrix}=\lambda^2-1=0,
    \end{align}
    解得 $\bm{\sigma}$ 的本征值为
    \begin{align}
        \lambda_1=1,\quad\lambda_2=-1.
    \end{align}
    将这两个本征值代入本征方程
    \begin{align}
        \bm{\sigma}_y\begin{bmatrix}
            x_1\\
            x_2
        \end{bmatrix}=\begin{bmatrix}
            0&-i\\
            i&0
        \end{bmatrix}\begin{bmatrix}
            x_1\\
            x_2
        \end{bmatrix}=\lambda_{1,2}\begin{bmatrix}
            x_1\\
            x_2
        \end{bmatrix},
    \end{align}
    解得对应的归一化本征态分别为
    \begin{align}
        \frac{1}{\sqrt{2}}\begin{bmatrix}
            1\\
            i
        \end{bmatrix},\quad\frac{1}{\sqrt{2}}\begin{bmatrix}
            1\\
            -i
        \end{bmatrix}.
    \end{align}

    测得该电子的 $s_y$ 为 $\hbar/2$ 的概率为
    \begin{align}
        P(s_y=\frac{\hbar}{2})=\abs{\frac{1}{\sqrt{2}}\begin{bmatrix}
            1&i
        \end{bmatrix}\begin{bmatrix}
            \alpha\\
            \beta
        \end{bmatrix}}^2=\frac{1}{2}\abs{\alpha-i\beta}^2.
    \end{align}
\end{sol}

\begin{prob}[课本习题 3.2]
    对于一个自旋 $\frac{1}{2}$ 的粒子, 在存在一个磁场 $\bm{B}=B_x\hat{\bm{x}}+B_y\hat{\bm{y}}+B_z\hat{\bm{z}}$ 情况下, 通过利用泡利矩阵显示结构, 求哈密顿量
    \[
        H=-\frac{2\mu}{\hbar}\bm{S}\cdot\bm{B}
    \]
    的本征值.
\end{prob}
\begin{sol}
    该粒子的哈密顿量可用泡利矩阵表为
    \begin{align}
        H=-\frac{2\mu}{\hbar}\frac{\hbar}{2}(\sigma_x\hat{\bm{x}}+\sigma_y\hat{\bm{y}}+\sigma_z\hat{\bm{z}})\cdot(B_x\hat{\bm{x}}+B_y\hat{\bm{y}}+B_z\hat{\bm{z}})=-\mu(B_x\sigma_x+B_y\sigma_y+B_z\sigma_z)=-\mu\begin{bmatrix}
            B_z&B_x-iB_y\\
            B_x+iB_y&-B_z
        \end{bmatrix}.
    \end{align}
    通过
    \begin{align}
        \abs{H-EI}=\begin{vmatrix}
            -\mu B_z-E&-\mu(B_x-iB_y)\\
            -\mu(\mu B_x+iB_y)&\mu B_z-E
        \end{vmatrix}=(-\mu B_z-E)(\mu B_z-E)-\mu^2(B_x-iB_y)(B_x+iB_y)=0,
    \end{align}
    解得哈密顿量的本征值为
    \begin{align}
        E_1=\mu\sqrt{B_x^2+B_y^2+B_z^2},\quad E_1=-\mu\sqrt{B_x^2+B_y^2+B_z^2}.
    \end{align}
\end{sol}

\begin{prob}[课本习题 3.5]
    考虑一个自旋为 $1$ 的粒子. 求
    \[
        S_z(S_z+\hbar)(S_z-\hbar)\quad\text{和}\quad S_x(S_x+\hbar)(S_x-\hbar)
    \]
    的矩阵元.
\end{prob}
\begin{pf}
    以 $\{\lvert s_z=+\hbar\rangle,\lvert s_z=0\rangle,\lvert s_z=-\hbar\rangle\}$ 为基, $S_z$ 的矩阵表示为
    \begin{align}
        S_z=\hbar\begin{bmatrix}
            1&0&0\\
            0&0&0\\
            0&0&-1
        \end{bmatrix}.
    \end{align}
    将其代入题干中的式子可得
    \begin{align}
        S_z(S_z+\hbar)(S_x-\hbar)=\hbar\begin{bmatrix}
            1&0&0\\
            0&0&0\\
            0&0&-1
        \end{bmatrix}\hbar\begin{bmatrix}
            2&0&0\\
            0&1&0\\
            0&0&0
        \end{bmatrix}\hbar\begin{bmatrix}
            0&0&0\\
            0&-1&0\\
            0&0&-2
        \end{bmatrix}=\begin{bmatrix}
            0&0&0\\
            0&0&0\\
            0&0&0
        \end{bmatrix}.
    \end{align}

    考虑到 $x$ 轴和 $z$ 轴是等价的, 故有
    \begin{align}
        S_x(S_x+\hbar)(S_x-\hbar)=\begin{bmatrix}
            0&0&0\\
            0&0&0\\
            0&0&0
        \end{bmatrix}.
    \end{align}
\end{pf}

\begin{prob}[课本习题 3.9]
    考虑一个由下式表示的欧拉转动序列
    \[
        \mathcal{D}^{(1/2)}=\exp\left(\frac{-i\sigma_3\alpha}{2}\right)\exp\left(\frac{-i\sigma_2\beta}{2}\right)\exp\left(\frac{-i\sigma_3\gamma}{2}\right).
    \]
    由于转动的群性质, 这一序列操作等价于绕某个轴转一个 $\theta$ 角的单一转动. 求 $\theta$.
\end{prob}
\begin{sol}
    该欧拉转动序列
    \begin{align}
        \notag\mathcal{D}^{(1/2)}=&\exp\left(\frac{-i\sigma_3\alpha}{2}\right)\exp\left(\frac{-i\sigma_2\beta}{2}\right)\exp\left(\frac{-i\sigma_3\gamma}{2}\right)\\
        \notag=&\begin{bmatrix}
            e^{-i\alpha/2}&0\\
            0&e^{i\alpha/2}
        \end{bmatrix}\begin{bmatrix}
            \cos(\beta/2)&-\sin(\beta/2)\\
            \sin(\beta/2)&\cos(\beta/2)
        \end{bmatrix}\begin{bmatrix}
            e^{-i\gamma/2}&0\\
            0&e^{i\gamma/2}
        \end{bmatrix}\\
        =&\begin{bmatrix}
            e^{-i(\alpha+\gamma)/2}\cos(\beta/2)&-e^{-i(\alpha-\gamma)/2}\sin(\beta/2)\\
            e^{i(\alpha-\gamma)/2}\sin(\beta/2)&e^{i(\alpha+\gamma)/2}\cos(\beta/2)
        \end{bmatrix}.
    \end{align}
    对比自旋 $\frac{1}{2}$ 系统中绕轴 $\hat{\bm{n}}$ 旋转 $\theta$ 角的算符
    \begin{align}
        U=\exp\left(\frac{-i\bm{\sigma}\cdot\hat{\bm{n}}\theta}{2}\right)=\begin{bmatrix}
            \cos\left(\frac{\theta}{2}\right)-in_z\sin\left(\frac{\theta}{2}\right)&(-in_x-n_y)\sin\left(\frac{\theta}{2}\right)\\
            (-in_x+n_y)\sin\left(\frac{\theta}{2}\right)&\cos\left(\frac{\theta}{2}\right)+in_z\sin\left(\frac{\phi}{2}\right)
        \end{bmatrix}
    \end{align}
    可知, 矩阵对角元之和即为旋转角 $\theta$ 的 $\frac{1}{2}$ 的余弦的 2 倍,
    \begin{align}
        e^{-i(\alpha+\gamma)/2}\cos(\beta/2)+e^{i(\alpha+\gamma)/2}\cos(\beta/2)=2\cos(\frac{\alpha+\gamma}{2})\cos\left(\frac{\beta}{2}\right)=2\cos\left(\frac{\theta}{2}\right),
    \end{align}
    解得等价转动的角度为
    \begin{align}
        \theta=2\arccos\left[\cos\left(\frac{\alpha+\gamma}{2}\right)\cos\left(\frac{\beta}{2}\right)\right].
    \end{align}
\end{sol}

\begin{prob}[课本习题 3.14]
    已知 $3\times 3$ 矩阵 $G_i(i=1,2,3)$, 其矩阵元由下式给出:
    \[
        (G_i)_{jk}=-i\hbar\varepsilon_{ijk},
    \]
    其中 $j$ 和 $k$ 是行和列指标, 证明它满足角动量对易关系. 把 $G_i$ 与比较常用的角动量算符 $J_i$, 在 $J_3$ 取为对角的情况下的 $3\times 3$ 表示联系起来, 实现该联系的变换矩阵的物理 (或几何) 意义是什么? 把得到的结果与无穷小转动下的
    \[
        \bm{V}\rightarrow\bm{V}+\hat{\bm{n}}\delta\phi\times\bm{V}
    \]
    联系起来. (注: 这个问题可能有助于理解光子的自旋.)
\end{prob}
\begin{pf}
    
\end{pf}

\begin{prob}[课本习题 3.15]
    \begin{itemize}
        \item[(a)] 令 $\bm{J}$ 是角动量. (它可以是轨道角动量 $\bm{L}$, 自旋 $\bm{S}$, 或 $\bm{J}_{\text{总}}$.) 利用 $J_x$, $J_y$, $J_z$ ($J_{\pm}\equiv J_x\pm iJ_y$) 满足通常角动量对易关系的事实, 证明
        \[
            \bm{J}^2=J_z^2+J_+J_--\hbar J_z.
        \]
        \item[(b)] 利用 (a) (或其他方式) 推导出在
        \[
            J_-\psi_{jm}=c_-\psi_{j,m-1}
        \]
        中的系数 $c_-$ 的 ``著名'' 的表达式.
    \end{itemize}
\end{prob}
\begin{pf}
    
\end{pf}
\end{document}