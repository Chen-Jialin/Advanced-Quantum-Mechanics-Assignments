\documentclass{assignment}
\ProjectInfos{高等量子力学}{PHYS5001P}{2022-2023 学年第一学期}{第五次作业}{截止时间: 2022 年 10 月 24 日 (周一)}{陈稼霖}[https://github.com/Chen-Jialin]{SA21038052}

\begin{document}
\begin{prob}[课本习题 3.1]
    求 $\bm{\sigma}_y=\begin{pmatrix}
        0&-i\\
        i&0
    \end{pmatrix}$ 的本征值和本征矢. 假定一个电子处在自旋态 $\begin{pmatrix}
        \alpha\\
        \beta
    \end{pmatrix}$ 上. 如果测得 $s_y$, 结果为 $\hbar/2$ 的概率是什么?
\end{prob}
\begin{sol}
    通过
    \begin{align}
        \abs{\sigma-\lambda I}=\begin{vmatrix}
            -\lambda&-i\\
            i&-\lambda
        \end{vmatrix}=\lambda^2-1=0,
    \end{align}
    解得 $\bm{\sigma}$ 的本征值为
    \begin{align}
        \lambda_1=1,\quad\lambda_2=-1.
    \end{align}
    将这两个本征值代入本征方程
    \begin{align}
        \bm{\sigma}_y\begin{bmatrix}
            x_1\\
            x_2
        \end{bmatrix}=\begin{bmatrix}
            0&-i\\
            i&0
        \end{bmatrix}\begin{bmatrix}
            x_1\\
            x_2
        \end{bmatrix}=\lambda_{1,2}\begin{bmatrix}
            x_1\\
            x_2
        \end{bmatrix},
    \end{align}
    解得对应的归一化本征态分别为
    \begin{align}
        \frac{1}{\sqrt{2}}\begin{bmatrix}
            1\\
            i
        \end{bmatrix},\quad\frac{1}{\sqrt{2}}\begin{bmatrix}
            1\\
            -i
        \end{bmatrix}.
    \end{align}

    测得该电子的 $s_y$ 为 $\hbar/2$ 的概率为
    \begin{align}
        P(s_y=\frac{\hbar}{2})=\abs{\frac{1}{\sqrt{2}}\begin{bmatrix}
            1&i
        \end{bmatrix}\begin{bmatrix}
            \alpha\\
            \beta
        \end{bmatrix}}^2=\frac{1}{2}\abs{\alpha-i\beta}^2.
    \end{align}
\end{sol}

\begin{prob}[课本习题 3.2]
    对于一个自旋 $\frac{1}{2}$ 的粒子, 在存在一个磁场 $\bm{B}=B_x\hat{\bm{x}}+B_y\hat{\bm{y}}+B_z\hat{\bm{z}}$ 情况下, 通过利用泡利矩阵显示结构, 求哈密顿量
    \[
        H=-\frac{2\mu}{\hbar}\bm{S}\cdot\bm{B}
    \]
    的本征值.
\end{prob}
\begin{sol}
    该粒子的哈密顿量可用泡利矩阵表为
    \begin{align}
        H=-\frac{2\mu}{\hbar}\frac{\hbar}{2}(\sigma_x\hat{\bm{x}}+\sigma_y\hat{\bm{y}}+\sigma_z\hat{\bm{z}})\cdot(B_x\hat{\bm{x}}+B_y\hat{\bm{y}}+B_z\hat{\bm{z}})=-\mu(B_x\sigma_x+B_y\sigma_y+B_z\sigma_z)=-\mu\begin{bmatrix}
            B_z&B_x-iB_y\\
            B_x+iB_y&-B_z
        \end{bmatrix}.
    \end{align}
    通过
    \begin{align}
        \abs{H-EI}=\begin{vmatrix}
            -\mu B_z-E&-\mu(B_x-iB_y)\\
            -\mu(\mu B_x+iB_y)&\mu B_z-E
        \end{vmatrix}=(-\mu B_z-E)(\mu B_z-E)-\mu^2(B_x-iB_y)(B_x+iB_y)=0,
    \end{align}
    解得哈密顿量的本征值为
    \begin{align}
        E_1=\mu\sqrt{B_x^2+B_y^2+B_z^2},\quad E_1=-\mu\sqrt{B_x^2+B_y^2+B_z^2}.
    \end{align}
\end{sol}

\begin{prob}[课本习题 3.5]
    考虑一个自旋为 $1$ 的粒子. 求
    \[
        S_z(S_z+\hbar)(S_z-\hbar)\quad\text{和}\quad S_x(S_x+\hbar)(S_x-\hbar)
    \]
    的矩阵元.
\end{prob}
\begin{pf}

\end{pf}
\end{document}