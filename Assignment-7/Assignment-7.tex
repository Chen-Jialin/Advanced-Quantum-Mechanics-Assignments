\documentclass{assignment}
\ProjectInfos{高等量子力学}{PHYS5001P}{2022-2023 学年第一学期}{第七次作业}{截止时间: 2022 年 11 月 7 日 (周一)}{陈稼霖}[https://github.com/Chen-Jialin]{SA21038052}

\begin{document}
\begin{prob}[课本习题 3.28]
    考虑两个自旋 $\frac{1}{2}$ 的粒子组成的一个系统. 观察者 A 专门测量其中一个离子的自旋分量 ($s_{1z}$, $s_{1x}$, 等等), 同时观察者 B 测量另一个粒子的自旋分量. 假定已知系统处在自旋单态, 即 $S_{\text{总}}=0$.
    \begin{itemize}
        \item[(a)] 当观察者 B 不做任何测量时, 观察者 A 得到 $s_{1z}=\hbar/2$ 的概率是什么? 对于 $s_{1x}=\hbar/2$ 求解同样问题.
        \item[(b)] 观察者 B 肯定地确认粒子 2 的自旋处于 $s_{2z}=\hbar/2$ 态. 如果观察者 A (i) 测量 $s_{1z}$; (ii) 测 $s_{1x}$, 则对观察者 A 的测量结果能给出的结论是什么? 解释你的答案.
    \end{itemize}
\end{prob}
\begin{sol}
    \begin{itemize}
        \item[(a)] 这两个粒子组成的系统处于自旋单态, 可表为
        \begin{align}
            \lvert\psi\rangle=\frac{1}{\sqrt{2}}(\hat{\bm{z}}+;\hat{\bm{z}}-\rangle+\lvert\hat{\bm{z}}-;\hat{\bm{z}}+\rangle)=\frac{1}{\sqrt{2}}(\hat{\bm{x}}-;\hat{\bm{x}}+\rangle+\lvert\hat{\bm{x}}+;\hat{\bm{x}}-\rangle).
        \end{align}
        当观察者 B 不做任何测量时, 观察者 A 得到 $s_{1z}=\hbar/2$ 的概率为 $\frac{1}{2}$;\\
        当观察者 B 不做任何测量时, 观察者 A 得到 $s_{1x}=\hbar/2$ 的概率为 $\frac{1}{2}$.
        \item[(b)] 当观察者 B 肯定地确认粒子 2 的自旋处于 $s_{2z}=\hbar/2$ 后, 系统的状态塌缩至
        \begin{align}
            \lvert\psi'\rangle=\lvert\hat{\bm{z}}-;\hat{\bm{z}}+\rangle=\lvert\hat{\bm{z}}-\rangle\otimes\frac{1}{\sqrt{2}}(\lvert\hat{\bm{x}}+\rangle-\lvert\hat{\bm{x}}-\rangle).
        \end{align}
        \begin{itemize}
            \item[(i)] 如果观察者测量 $s_{1z}$, 则有 $100\%$ 的概率得到 $s_{1z}=-\hbar/2$.
            \item[(ii)] 如果观察者测量 $s_{1x}$, 则有 $\frac{1}{2}$ 的概率得到 $s_{1x}=\hbar/2$, 有 $\frac{1}{2}$ 的概率得到 $s_{1x}=-\hbar/2$.
        \end{itemize}
    \end{itemize}
\end{sol}

\begin{prob}[课本习题 3.30]
    \begin{itemize}
        \item[(a)] 用两个不同的矢量 $\bm{U}=(U_x,U_y,U_z)$ 和 $\bm{V}=(V_x,V_y,V_z)$ 构造一个秩为 $1$ 的球张量. 明确地用 $U_{x,y,z}$ 和 $V_{x,y,z}$ 写出 $T_{\pm 1,0}^{(1)}$.
        \item[(b)] 用两个不同的矢量 $\bm{U}$ 和 $\bm{V}$ 构造一个秩为 $2$ 的球张量. 明确地用 $U_{x,y,z}$ 和 $V_{x,y,z}$ 写出 $\bm{T}_{\pm 2,\pm 1,0}^{(2)}$.
    \end{itemize}
\end{prob}
\begin{sol}
    \begin{itemize}
        \item[(a)] 由
        \begin{align}
            Y_{l=1}^{m=0}(\bm{U})=&\sqrt{\frac{3}{4\pi}}\cos\theta=\sqrt{\frac{3}{4\pi}}\frac{U_z}{\abs{U}},\\
            Y_{l=1}^{m=\pm 1}(\bm{U})=&\mp\sqrt{\frac{3}{8\pi}}\sin\theta e^{\pm i\theta}=\mp\sqrt{\frac{3}{4\pi}}\frac{U_x\pm iU_y}{\abs{U}},
        \end{align}
        得单由 $\bm{U}$ 构造的秩为 $1$ 的球张量
        \begin{align}
            U_0^{(1)}=&U_z,\\
            U_{\pm 1}^{(1)}=&\mp\frac{U_x\pm iU_y}{\sqrt{2}}.
        \end{align}

        对 $\bm{V}$ 同理有
        \begin{align}
            V_0^{(1)}=&V_z,\\
            V_{\pm 1}^{(1)}=&\mp\frac{V_x\pm iV_y}{\sqrt{2}}.
        \end{align}

        利用定理
        \begin{align}
            T_q^{(k)}=\sum_{q_1}\sum_{q_2}\langle k_1k_2;q_1q_2\vert k_1k_2;kq\rangle X_{q_1}^{(k_1)}Z_{q_2}^{(k_2)},
        \end{align}
        有
        \begin{align}
            T_{q}^{(1)}=\sum_{q_1}\sum_{q_2}\langle 11;q_1q_2\vert 11;1q\rangle U_{q_1}^{(1)}V_{q_2}^{(2)},
        \end{align}
        从而
        \begin{align}
            \notag T_{+1}^{(1)}=&\langle 11;0,+1\vert 11;1,+1\rangle U_{0}^{(1)}V_{+1}^{(1)}+\langle 11;+1,0\vert 11;1,+1\rangle U_{+1}^{(1)}V_{0}^{(1)}\\
            \notag=&-\frac{1}{\sqrt{2}}U_z\frac{-V_x-iV_y}{\sqrt{2}}+\frac{1}{\sqrt{2}}\frac{-U_x-iU_y}{\sqrt{2}}V_z\\
            =&\frac{1}{2}(U_zV_x+iU_zV_y-U_xV_z-iU_yV_z),\\
            \notag T_{0}^{(1)}=&\langle 11;-1,+1\vert 11;10\rangle U_{-1}^{(1)}V_{+1}^{(1)}+\langle 11;00\vert 11;10\rangle U_{0}^{(1)}V_{0}^{(1)}+\langle 11;+1,-1\vert 11;10\rangle U_{+1}^{(1)}V_{-1}^{(1)}\\
            \notag=&-\frac{1}{\sqrt{2}}\frac{U_x-iU_y}{\sqrt{2}}\frac{-V_x-iV_y}{\sqrt{2}}+\frac{1}{\sqrt{2}}\frac{-U_x-iU_y}{\sqrt{2}}\frac{V_x-iV_y}{\sqrt{2}}\\
            =&\frac{i}{\sqrt{2}}(U_xV_y-U_yV_x),\\
            \notag T_{-1}^{(1)}=&\langle 11;-1,0\vert 11;1,-1\rangle U_{-1}^{(1)}V_{0}^{(1)}+\langle 11;0,-1\vert 11;1,-1\rangle U_{0}^{(1)}V_{-1}^{(1)}\\
            \notag=&-\frac{1}{\sqrt{2}}\frac{U_x-iU_y}{\sqrt{2}}V_z+\frac{1}{\sqrt{2}}U_z\frac{V_x-iV_y}{\sqrt{2}}\\
            =&\frac{1}{2}(-U_xV_z+iU_yV_z+U_zV_x-iU_zV_y).
        \end{align}
        \item[(b)] 利用定理
        \begin{align}
            T_q^{(k)}=\sum_{q_1}\sum_{q_2}\langle k_1k_2;q_1q_2\vert k_1k_2;kq\rangle X_{q_1}^{(k_1)}Z_{q_2}^{(k_2)},
        \end{align}
        有
        \begin{align}
            T_q^{(2)}=\sum_{q_1}\sum_{q_2}\langle 11;q_1q_2\vert 11;2q\rangle U_{q_1}^{(1)}V_{q_2}^{(1)},
        \end{align}
        从而
        \begin{align}
            T_{+2}^{(2)}=&\langle 11;+1,+1\vert 11;2,+2\rangle U_{+1}^{(1)}V_{+1}^{(1)}=\frac{-U_x-iU_y}{\sqrt{2}}\frac{-V_x-iU_y}{\sqrt{2}}=\frac{1}{2}(U_xV_x+iU_xV_y+iU_yV_x+U_yV_y),\\
            \notag T_{+1}^{(2)}=&\langle 11;0,+1\vert 11;2,+1\rangle U_{0}^{(1)}V_{+1}^{(1)}+\langle 11;+1,0\vert 11;2,+1\rangle U_{+1}^{(1)}V_{0}^{(1)}\\
            \notag=&\frac{1}{\sqrt{2}}U_z\frac{-V_x-iV_y}{\sqrt{2}}+\frac{1}{\sqrt{2}}\frac{-U_x-iU_y}{\sqrt{2}}V_z\\
            =&-\frac{1}{2}[U_z(V_x+iV_y)+(U_x+iU_y)V_z],\\
            \notag T_{0}^{(2)}=&\langle 11;-1,+1\vert 11;20\rangle U_{-1}^{(1)}V_{+1}^{(1)}+\langle 11;00\vert 11;20\rangle U_{0}^{(1)}V_{0}^{(1)}+\langle 11;+1,-1\vert 11;20\rangle U_{+1}^{(1)}V_{-1}^{(1)}\\
            \notag=&\frac{1}{\sqrt{6}}\frac{U_x-iU_y}{\sqrt{2}}\frac{-V_x-iV_y}{\sqrt{2}}+\frac{2}{\sqrt{6}}U_zV_z+\frac{1}{\sqrt{6}}\frac{-U_x-iU_y}{\sqrt{2}}\frac{V_x-iV_y}{\sqrt{2}}\\
            =&\frac{1}{\sqrt{6}}(-U_xV_x-U_yV_y+2U_zV_z),\\
            \notag T_{-1}^{(2)}=&\langle 11;-1,0\vert 11;2,-1\rangle U_{-1}^{(1)}V_{0}^{(1)}+\langle 11;0,-1\vert 11;2,-1\rangle U_{0}^{(1)}V_{-1}^{(1)}\\
            \notag=&\frac{1}{\sqrt{2}}\frac{U_x-iU_y}{\sqrt{2}}V_z+\frac{1}{\sqrt{2}}U_z\frac{V_x-iV_y}{\sqrt{2}}\\
            =&\frac{1}{2}(U_xV_z-iU_yV_z+U_zV_x-iU_zV_y),\\
            \notag T_{-2}^{(2)}=&\langle 11;-1,-1\vert 11;2,-2\rangle U_{-1}^{(1)}V_{-1}^{(1)}=\frac{U_x-iU_y}{\sqrt{2}}\frac{V_x-iU_y}{\sqrt{2}}=\frac{1}{2}(U_xV_x-iU_xV_y-iU_yV_x-U_yV_y).
        \end{align}
    \end{itemize}
\end{sol}

\begin{prob}[课本习题 3.32]
    \begin{itemize}
        \item[(a)] 把 $xy$, $xz$ 和 $(x^2-y^2)$ 写成一个秩为 $2$ 的球 (不可约) 张量的分量.
        \item[(b)] 期待值
        \[
            Q=e\langle\alpha,j,m=j\rvert(3z^2-r^2)\lvert\alpha,j,m=j\rangle
        \]
        被称为 \textit{四极矩}. 利用 $Q$ 和适当的克莱布什-戈丹系数, 求
        \[
            e\langle\alpha,j,m'\rvert(x^2-y^2)\lvert\alpha,j,m=j\rangle,
        \]
        其中 $m'=j,j-1,j-2,\cdots$.
    \end{itemize}
\end{prob}
\begin{sol}
    \begin{itemize}
        \item[(a)] 由
        \begin{align}
            Y_2^0(\bm{r})=&\sqrt{\frac{5}{16\pi}}(3\cos^2\theta-1)=\sqrt{\frac{5}{16\pi}}\left(3\frac{z^2}{r^2}-1\right),\\
            Y_2^{\pm 1}(\bm{r})=&\mp\sqrt{\frac{15}{8\pi}}\sin\theta\cos\theta e^{\pm i\phi}=\mp\sqrt{\frac{15}{8\pi}}\frac{(x\pm iy)z}{r^2},\\
            Y_2^{\pm 2}(\bm{r})=&\sqrt{\frac{15}{32\pi}}\sin^2\theta e^{\pm 2i\phi}=\sqrt{\frac{15}{32\pi}}\frac{(x\pm iy)^2}{r^2},
        \end{align}
        得
        \begin{align}
            xy=&\frac{1}{4i}\left[\sqrt{\frac{32\pi}{15}}r^2Y_2^{+2}(\bm{r})-\sqrt{\frac{32\pi}{15}}r^2Y_2^{-2}(\bm{r})\right]=i\sqrt{\frac{2\pi}{15}}r^2[Y_2^{-2}(\bm{r})-Y_2^{+2}(\bm{r})],\\
            xz=&\frac{1}{2}\left[\sqrt{\frac{8\pi}{15}}r^2Y_2^{-1}(\bm{r})-\sqrt{\frac{8\pi}{15}}r^2Y_2^{+1}(\bm{r})\right]=\sqrt{\frac{2\pi}{15}}r^2[Y_2^{-1}(\bm{r})-Y_2^{+1}(\bm{r})],\\
            (x^2-y^2)=&\frac{1}{2}\left[\sqrt{\frac{32\pi}{15}}r^2Y_2^{+2}(\bm{r})+\sqrt{\frac{32\pi}{15}}r^2Y_2^{-2}(\bm{r})\right]=\sqrt{\frac{8\pi}{15}}r^2[Y_2^{+2}(\bm{r})+Y_2^{-2}(\bm{r})].
        \end{align}
        \item[(b)] 利用前一小题的结论,
        \begin{align}
            e\langle\alpha,j,m'\rvert(x^2-y^2)\lvert\alpha,j,m=j\rangle=e\sqrt{\frac{8\pi}{15}}\langle\alpha,j,m'\rvert r^2[Y_2^{-2}(\bm{r})+Y_2^{+2}(\bm{r})]\lvert\alpha,j,m=j\rangle.
        \end{align}
        利用 Wigner-Eckart 定理,
        \begin{align}
            \notag e\langle\alpha,j,m'\rvert(x^2-y^2)\lvert\alpha,j,m=j\rangle=&e\sqrt{\frac{8\pi}{15}}[\langle j2;j,-2\vert j2;jm'\rangle+\langle j2;j,+2\vert j2;jm'\rangle]\frac{\langle\alpha,j\rVert Y^{(2)}\lVert\alpha,j\rangle}{\sqrt{2j+1}}\\
            =&e\sqrt{\frac{8\pi}{15}}\langle j2;j,-2\vert j2;jm'\rangle\frac{\langle\alpha,j\rVert Y^{(2)}\lVert\alpha,j\rangle}{\sqrt{2j+1}}
        \end{align}

        另一方面,
        \begin{align}
            \notag Q=&e\langle\alpha,j,m=j\rvert(3z^2-r^2)\lvert\alpha,j,m=j\rangle=e\sqrt{\frac{16\pi}{5}}\langle\alpha,j,m'\rvert Y_2^0(\bm{r})\lvert\alpha,j,m=j\rangle\\
            =&e\sqrt{\frac{16\pi}{5}}\langle j2;j0\vert j2;jj\rangle\frac{\langle\alpha,j\rVert Y^{(2)}\lVert\alpha,j\rangle}{\sqrt{2j+1}}.
        \end{align}
        故
        \begin{align}
            e\langle\alpha,j,m'\rvert(x^2-y^2)\lvert\alpha,j,m=j\rangle=\frac{Q}{\sqrt{2}}\frac{\langle j2;j,-2\vert j2,jm'\rangle}{\langle j2;j0\vert j2;jj\rangle}.
        \end{align}
    \end{itemize}
\end{sol}

\begin{prob}[课本习题 3.10]
    \begin{itemize}
        \item[(a)] 考虑全同制备的自旋 $\frac{1}{2}$ 系统的一个纯系综. 假定期待值 $\langle S_x\rangle$ 和 $\langle S_z\rangle$ 已知, 而 $\langle S_y\rangle$ 的符号也已知. 证明如何确定态矢量. 为什么不必知道 $\langle S_y\rangle$ 的大小?
        \item[(b)] 考虑一个自旋 $\frac{1}{2}$ 系统的混合系综. 假定系综平均值 $[S_x]$, $[S_y]$, $[S_z]$ 都是已知的, 证明如何可以构造表征这个系综的 $2\times 2$ 密度矩阵.
    \end{itemize}
\end{prob}
\begin{pf}
    \begin{itemize}
        \item[(a)] 设该纯系综的状态为
        \begin{align}
            \lvert\alpha\rangle=\cos\frac{\beta}{2}\lvert+\rangle+e^{i\alpha}\sin\frac{\beta}{2}\lvert-\rangle.
        \end{align}
        期待值
        \begin{align}
            \notag\langle S_x\rangle=&\langle\alpha\rvert S_x\lvert\alpha\rangle=\left(\cos\frac{\beta}{2}\langle+\rvert+e^{-i\alpha}\sin\frac{\beta}{2}\langle-\rvert\right)\frac{\hbar}{2}(\lvert+\rangle\langle-\rvert+\lvert-\rangle\langle+\rvert)\left(\cos\frac{\beta}{2}\lvert+\rangle+e^{i\alpha}\sin\frac{\beta}{2}\lvert-\rangle\right)\\
            =&\frac{\hbar}{2}\left(e^{i\alpha}\cos\frac{\beta}{2}\sin\frac{\beta}{2}+e^{-i\alpha}\sin\frac{\beta}{2}\cos\frac{\beta}{2}\right)=\frac{\hbar}{2}\cos\alpha\sin\beta,\\
            \notag\langle S_z\rangle=&\langle\alpha\rvert S_z\lvert\alpha\rangle=\left(\cos\frac{\beta}{2}\langle+\rvert+e^{-i\alpha}\sin\frac{\beta}{2}\langle-\rvert\right)\frac{\hbar}{2}(\lvert+\rangle\langle+\rvert-\lvert-\rangle\langle-\rvert)\left(\cos\frac{\beta}{2}\lvert+\rangle+e^{i\alpha}\sin\frac{\beta}{2}\lvert-\rangle\right)\\
            =&\frac{\hbar}{2}\left(\cos^2\frac{\beta}{2}-\sin^2\frac{\beta}{2}\right)=\frac{\hbar}{2}\cos\beta,\\
            \notag\langle S_y\rangle=&\langle\alpha\rvert S_y\lvert\alpha\rangle=\langle\alpha\rvert S_z\lvert\alpha\rangle=\left(\cos\frac{\beta}{2}\langle+\rvert+e^{-i\alpha}\sin\frac{\beta}{2}\langle-\rvert\right)\frac{\hbar}{2}(-i\lvert+\rangle\langle-\rvert+i\lvert-\rangle\langle+\rvert)\left(\cos\frac{\beta}{2}\lvert+\rangle+e^{i\alpha}\sin\frac{\beta}{2}\lvert-\rangle\right)\\
            =&\frac{\hbar}{2}\left(-ie^{i\alpha}\cos\frac{\beta}{2}\sin\frac{\beta}{2}+ie^{-i\alpha}\sin\frac{\beta}{2}\cos\frac{\beta}{2}\right)=\frac{\hbar}{2}\sin\alpha\sin\beta.
        \end{align}
        因此由已知的 $\langle S_x\rangle$ 和 $\langle S_z\rangle$ 就可得
        \begin{align}
            \langle S_y\rangle=\pm\sqrt{\left(\frac{\hbar}{2}\right)^2-\langle S_x\rangle^2-\langle S_z\rangle^2}.
        \end{align}
        当 $\langle S_y\rangle$ 符号已知, $\langle S_y\rangle$ 就可被完全确定下来.
        此时, $\alpha$ 和 $\beta$ 也被确定下来:
        \begin{align}
            \alpha=&\arctan\frac{\langle S_y\rangle}{\langle S_x\rangle},\\
            \beta=&\arccos\frac{\langle S_z\rangle}{\hbar/2},
        \end{align}
        即可确定态矢量 $\lvert\alpha\rangle$.
        \item[(b)] 设该混合系综的密度矩阵为
        \begin{align}
            \rho=\begin{bmatrix}
                a&b\\
                c&d
            \end{bmatrix},
        \end{align}
        其满足
        \begin{gather}
            \rho=\rho^{\dagger}\Longrightarrow b=c^*,\text{ and }a,d\in\mathbb{R},\\
            \tr(\rho)=a+d=1.
        \end{gather}
        系综平均值
        \begin{align}
            [S_x]=&\tr(\rho S_x)=\tr\left(\begin{bmatrix}
                a&b\\
                c&d
            \end{bmatrix}\frac{\hbar}{2}\begin{bmatrix}
                0&1\\
                1&0
            \end{bmatrix}\right)=\frac{\hbar}{2}(b+c),\\
            [S_y]=&\tr(\rho S_y)=\tr\left(\begin{bmatrix}
                a&b\\
                c&d
            \end{bmatrix}\frac{\hbar}{2}\begin{bmatrix}
                0&-i\\
                i&0
            \end{bmatrix}\right)=i\frac{\hbar}{2}(b-c),\\
            [S_z]=&\tr(\rho S_z)=\tr\left(\begin{bmatrix}
                a&b\\
                c&d
            \end{bmatrix}\frac{\hbar}{2}\begin{bmatrix}
                1&0\\
                0&-1
            \end{bmatrix}\right)=\frac{\hbar}{2}(a-d).
        \end{align}
        当 $[S_x]$, $[S_y]$, $[S_z]$ 已知, 则
        \begin{align}
            a=&\frac{1}{2}\left(1+\frac{[S_z]}{\hbar/2}\right),\\
            b=&\frac{1}{\hbar}([S_x]-i[S_y]),\\
            c=&\frac{1}{\hbar}([S_x]+i[S_y]),\\
            d=&\frac{1}{2}\left(1-\frac{[S_z]}{\hbar/2}\right),
        \end{align}
        由此可以构造出表征该混合系综的 $2\times 2$ 密度矩阵.
    \end{itemize}
\end{pf}

\begin{prob}[课本习题 3.11]
    \begin{itemize}
        \item[(a)] 证明密度算符 (在薛定谔绘景中) 的时间演化由下式给定
        \[
            \rho(t)=\mathcal{U}(t,t_0)\rho(t_0)\mathcal{U}^{\dagger}(t,t_0).
        \]
        \item[(b)] 假定在 $t=0$ 时有一个纯系统. 证明只要时间演化由薛定谔方程控制, 则它不可能演化成一个混合系统.
    \end{itemize}
\end{prob}
\begin{pf}
    \begin{itemize}
        \item[(a)] 设 $t_0$ 时刻密度算符为
        \begin{align}
            \rho(t_0)=\sum_iw_i\lvert\alpha^{(i)},t_0\rangle\langle\alpha^{(i)},t_0\rvert.
        \end{align}
        在薛定谔绘景中, 任意纯态 $\lvert\alpha^{(i)}\rangle$ 按照
        \begin{align}
            \lvert\alpha,t_0;t\rangle=\mathcal{U}(t,t_0)\lvert\alpha^{(i)},t_0\rangle
        \end{align}
        的形式演化, 其中 $\mathcal{U}(t,t_0)$ 为时刻 $t_0$ 时刻至 $t$ 时刻的演化算符.
        故 $t$ 时刻密度算符演化为
        \begin{align}
            \notag\rho(t)=&\sum_iw_i\mathcal{U}(t,t_0)\lvert\alpha^{(i)},t_0\rangle\langle\alpha^{(i)},t_0\rvert\mathcal{U}^{\dagger}(t,t_0)\\
            \notag=&\mathcal{U}(t,t_0)\left[\sum_iw_i\lvert\alpha^{(i)},t_0\rangle\langle\alpha^{(i)},t_0\rvert\right]\mathcal{U}^{\dagger}(t,t_0)\\
            =&\mathcal{U}(t,t_0)\rho(t_0)\mathcal{U}^{\dagger}(t,t_0).
        \end{align}
        \item[(b)] $t=0$ 时刻该纯态系统的密度算符 $\rho(t=0)$ 满足
        \begin{align}
            \tr[\rho^2(t=0)]=1.
        \end{align}
        任意 $t$ 时刻该系统的密度算符仍然满足
        \begin{align}
            \notag\tr[\rho^2(t)]=&\tr[U(t,0)\rho(t=0)U^{\dagger}(t,0)U(t,0)\rho(t=0)U^{\dagger}(t,0)]\\
            \notag=&\tr[U(t,0)\rho^2(t=0)U^{\dagger}(t,0)]\\
            \notag=&\tr[U^{\dagger}(t,0)U(t,0)\rho^2(t=0)]\\
            \notag=&\tr[\rho^2(t=0)]\\
            =&1,
        \end{align}
        即该系统仍然为一纯态系统.
        因此只要时间演化由薛定谔方程控制, 则它不可能演化成一个混合系统.
    \end{itemize}
\end{pf}
\end{document}