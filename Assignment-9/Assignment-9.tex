\documentclass{assignment}
\ProjectInfos{高等量子力学}{PHYS5001P}{2022-2023 学年第一学期}{第九次作业}{截止时间: 2022 年 11 月 21 日 (周一)}{陈稼霖}[https://github.com/Chen-Jialin]{SA21038052}

\begin{document}
\begin{prob}[课本习题 4.8]
    \begin{itemize}
        \item[(a)] 假定哈密顿量在时间反演下不变, 证明对一个无自旋非简并系统在任意给定时刻的波函数总可以选择为实的.
        \item[(b)] 对于 $t=0$ 时刻的一个平面波, 其波函数由一个复函数 $e^{j\bm{p}\cdot\bm{x}/\hbar}$ 给出. 为什么这不破坏时间反演不变性.
    \end{itemize}
\end{prob}
\begin{pf}
    \begin{itemize}
        \item[(a)] 假设能量本征态 $\lvert n\rangle$, 对应的能量本征值为 $E_n$. 由于哈密顿量在时间反演下不变, 即哈密顿量与时间反演算符对易
        \begin{align}
            [H,\Theta]=H\Theta-\Theta H=0,
        \end{align}
        从而有
        \begin{align}
            H\Theta\lvert n\rangle=\Theta H\lvert n\rangle=E_n\Theta\lvert n\rangle,
        \end{align}
        即 $\Theta\lvert n\rangle$ 与 $\lvert n\rangle$ 具有相同的本征值, 又因为该系统非简并, 故 $\Theta\lvert n\rangle$ 与 $\lvert n\rangle$ 为同一个态 (至多相差一个相因子).
        不妨设 $\Theta\lvert n\rangle=e^{i\phi}\lvert n\rangle$.
        $\lvert n\rangle$ 和 $\Theta\lvert n\rangle$ 的波函数分别为 $\langle\bm{x}'\vert n\rangle$ 和 $\langle\bm{x}'\rvert\Theta\lvert n\rangle=e^{i\phi}\langle\bm{x}'\vert n\rangle$.
        由于 $\lvert\bm{x}'\rangle$ 在时间反演下不变, 即 $\lvert\bm{x}'\rangle=\Theta\lvert\bm{x}'\rangle$, 以及时间反演算符的反幺正性, 有
        \begin{gather}
            e^{i\phi}\langle\bm{x}'\vert n\rangle=\langle\bm{x}'\rvert\Theta\lvert n\rangle=\langle\bm{x}'\vert n\rangle^*,\\
            \Longrightarrow(e^{i\phi/2}\langle\bm{x}'\vert n\rangle)=(e^{i\phi/2}\langle\bm{x}'\rvert n\rangle)^*
        \end{gather}
        故总是可以将 $\langle\bm{x}'\vert n\rangle$ 分解成带有一个不含时也与 $\bm{x}'$ 无关的相因子与一个实函数的乘积的形式, 而一个整体相位无任何实质影响, 因此总是可以将本征态波函数选择为实的.

        该系统在任意给定时刻的波函数总是可表为这些本征态波函数的线性叠加, 从而也可被选择为实的.
        \item[(b)] 上一小题的结论的逆否命题是说: 若一个无自旋非简并系统在某时刻的波函数无法被选择为正的, 则时间反演不变性破坏.

        该平面波的波函数有一个依赖于 $\bm{x}$ 的相位, 从而无法给选择为实的.
        但注意到该波函数与 $e^{-j\bm{p}\cdot\bm{x}'/\hbar}$ 等波函数具有简并的能量, 这违反了非简并系统这一前提条件, 因此无法根据上一小题的结论推出时间反演不变性破坏的这件事.
    \end{itemize}
\end{pf}

\begin{prob}[课本习题 4.10]
    \begin{itemize}
        \item[(a)] 用 (4.5.53) 式证明 $\Theta\lvert j,m\rangle$ 等于 $\lvert j,-m\rangle$, 至多一个包含有因子 $(-1)^m$ 的相因子. 这就是说, 证明 $\Theta\lvert j,m\rangle=e^{i\delta}(-1)^m\lvert j,-m\rangle$, 其中的 $\delta$ 不依赖于 $m$.
        \item[(b)] 利用同样的相位约定求相应于 $\mathscr{D}(R)\lvert j,m\rangle$ 的时间反演态, 先用无穷小形式 $\mathscr{D}(\hat{\bm{n}},\mathrm{d}\phi)$ 处理, 然后推广到有限转动.
        \item[(c)] 从这些结果出发说明, 不依赖于 $\delta$, 有
        \[
            \mathscr{D}_{m'm}^{(j)*}(R)=(-1)^{m-m'}\mathscr{D}_{-m',-m}^{(j)}(R).
        \]
        \item[(d)] 可以得出如下结论: 可以自由地选取 $\delta=0$, 以及 $\Theta\lvert j,m\rangle=(-1)^m\lvert j,-m\rangle=i^{2m}\lvert j,-m\rangle$.
    \end{itemize}
\end{prob}
\begin{pf}
    \begin{itemize}
        \item[(a)] 课本式 (4.5.53)
        \begin{align}
            \Theta\bm{J}\Theta^{-1}=-\bm{J}
        \end{align}
        说明 $\bm{J}$ 具有时间反演奇对称性, 故其包括 $J_z$ 在内的各分量算符以及算符 $J_{\pm}$ 均与时间反演算符 $\Theta$ 反对易.
        由此有
        \begin{align}
            J_z\Theta\lvert j,m\rangle=-\Theta J_z\lvert j,m\rangle=-m\hbar\Theta\lvert j,m\rangle,
        \end{align}
        即 $\Theta\lvert j,m\rangle$ 和 $\lvert j,-m\rangle$ 至多相差一个相因子.
        又有
        \begin{align}
            J_{\pm}\Theta\lvert j,m\rangle=-\Theta J_{\pm}\lvert j,m\rangle=-\sqrt{(j\mp m)(j\pm m+1)}\hbar\Theta\lvert j,m\pm 1\rangle,
        \end{align}
        故 $\Theta\lvert j,m\rangle$ 总是与相邻的 $\Theta\lvert j,m\pm 1\rangle$ 相差一个负号.

        综上, $\Theta\lvert j,m\rangle$ 与 $\lvert j,-m\rangle$ 至多相差一个包含有因子 $(-1)^m$ 的相因子, 即
        \begin{align}
            \Theta\lvert j,m\rangle=e^{i\delta}(-1)^m\lvert j,-m\rangle.
        \end{align}
        \item[(b)] 无穷小形式的转动算符
        \begin{align}
            \mathscr{D}(\hat{\bm{n}},\mathrm{d}\phi)=1-\frac{i\bm{J}\cdot\hat{\bm{n}}\mathrm{d}\phi}{\hbar},
        \end{align}
        由于角动量 $\hat{J}$ 在时间反演下是奇的, 且复数单位 $i$ 在反线性的时间反演下反号, 故无穷小形式的转动算符 $\mathscr{D}(\hat{\bm{n}},\mathrm{d}\phi)$ 与时间反演算符 $\Theta$ 对易:
        \begin{align}
            \Theta\mathscr{D}(\hat{\bm{n}},\mathrm{d}\phi)\Theta^{-1}=\Theta\left[1-\frac{i\bm{J}\cdot\hat{\bm{n}}\mathrm{d}\phi}{\hbar}\right]\Theta^{-1}=1+i\mathscr{D}(\hat{\bm{n}},\mathrm{d}\phi)\frac{\bm{J}\cdot\hat{\bm{n}}\mathrm{d}\phi}{\hbar}\mathscr{D}(\hat{\bm{n}},\mathrm{d}\phi)=1-i\frac{\bm{J}\cdot\hat{\bm{n}}\mathrm{d}\phi}{\hbar}=\mathscr{D}(\hat{\bm{n}},\mathrm{d}\phi).
        \end{align}
        利用上式有
        \begin{align}
            \Theta\mathscr{D}(\hat{\bm{n}},\mathrm{d}\phi)\lvert j,m\rangle=\mathscr{D}(\hat{\bm{n}},\mathrm{d}\phi)\Theta\lvert j,m\rangle=e^{i\delta}(-1)^m\mathscr{D}(\hat{\bm{n}},\mathrm{d}\phi)\lvert j,-m\rangle
        \end{align}

        推广到有限转动:
        \begin{align}
            \notag\Theta\mathscr{D}(\hat{\bm{n}},\phi)\Theta^{-1}=&\Theta\lim_{n\rightarrow\infty}\left[1-i\frac{\bm{J}\cdot\hat{\bm{n}}}{\hbar}\frac{\phi}{n}\right]^n\Theta^{-1}=\lim_{n\rightarrow\infty}\Theta\left[1-i\frac{\bm{J}\cdot\hat{\bm{n}}}{\hbar}\frac{\phi}{n}\right]^n\Theta^{-1}\\
            \notag=&\lim_{n\rightarrow\infty}\overbrace{\Theta\left[1-i\frac{\bm{J}\cdot\hat{\bm{n}}}{\hbar}\frac{\phi}{n}\right]\Theta^{-1}\Theta\left[1-i\frac{\bm{J}\cdot\hat{\bm{n}}}{\hbar}\frac{\phi}{n}\right]\Theta^{-1}\cdots\Theta\left[1-i\frac{\bm{J}\cdot\hat{\bm{n}}}{\hbar}\frac{\phi}{n}\right]\Theta^{-1}}^{\text{共 }n\text{ 个相乘}}\\
            \notag=&\lim_{n\rightarrow\infty}\left[1-i\frac{\bm{J}\cdot\hat{\bm{n}}}{\hbar}\frac{\phi}{n}\right]^n=\exp\left(\frac{-i\bm{J}\cdot\hat{\bm{n}}\mathrm{d}\phi}{\hbar}\right)\\
            =&\mathscr{D}(\hat{\bm{n}},\phi).
        \end{align}
        从而有
        \begin{align}
            \Theta\mathscr{D}(\hat{\bm{n}},\phi)\lvert j,m\rangle=\mathscr{D}(\hat{\bm{n}},\phi)\Theta\lvert j,m\rangle=e^{i\delta}(-1)^m\mathscr{D}(\hat{\bm{n}},\phi)\lvert j,-m\rangle.
        \end{align}
        \item[(c)] 利用线性算符的性质:
        \begin{align}
            \langle\beta\rvert A\lvert\alpha\rangle=\pm\langle\tilde{\beta}\rvert A\lvert\tilde{\alpha}\rangle^*,
        \end{align}
        (其中 $\pm$ 分别对应线性算符 $A$ 在时间反演下的奇偶对称性)
        有
        \begin{align}
            \notag\mathscr{D}_{m'm}^{(j)*}(R)=&\langle j,m'\rvert\mathscr{D}(R)\lvert j,m\rangle^*=\langle\widetilde{j,m}\rvert\mathscr{D}(R)\lvert\widetilde{j,m'}\rangle=e^{-i\delta}(-1)^{m'}\langle j,-m'\rvert\mathscr{D}(R)e^{i\delta}(-1)^m\lvert j,-m\rangle\\
            \notag=&(-1)^{m'+m}\mathscr{D}_{m'm}(R)\\
            =&(-1)^{m-m'}\mathscr{D}_{m'm}(R).
        \end{align}
        \item[(d)] 由 (a) 可知 $\Theta\lvert j,m\rangle=e^{i\delta}(-1)^m\lvert j,-m\rangle$, 而由 (c) 可知, 相因子 $e^{i\delta}$ 的选取不影响算符矩阵元, 因此不妨选取 $\delta=0$, 从而
        \begin{align}
            \Theta\lvert j,m\rangle=(-1)^m\lvert j,-m\rangle=i^{2m}\lvert j,-m\rangle.
        \end{align}
    \end{itemize}
\end{pf}

\begin{prob}[课本习题 5.1]
    一个 (一维) 简谐振子受到一个微扰
    \[
        \lambda H_1=bx,
    \]
    其中 $b$ 是一个实常数.
    \begin{itemize}
        \item[(a)] 计算基态的能移到最低的非零级.
        \item[(b)] 严格求解这个问题, 并与 $(a)$ 中得到的结果比较. 可以不加证明地假定
        \[
            \langle u_{n'}\rvert x\lvert u_n\rangle=\sqrt{\frac{\hbar}{2m\omega}}(\sqrt{n+1}\delta_{n',n+1}+\sqrt{n}\delta_{n',n-1}).
        \]
    \end{itemize}
\end{prob}
\begin{sol}
    \begin{itemize}
        \item[(a)] 基态的能移为
        \begin{align}
            \Delta_0=\lambda(H_1)_{00}+\lambda^2\sum_{k\neq 0}\frac{\abs{(H_1)_{0k}}^2}{E_0^{(0)}-E_k^{(0)}}+\mathcal{O}(\lambda^3)
        \end{align}
        其中
        \begin{align}
            \lambda(H_1)_{00}=&b\langle 0^{(0)}\rvert x\lvert 0^{(0)}\rangle=b\langle 0^{(0)}\rvert\sqrt{\frac{\hbar}{2m\omega}}(a+a^{\dagger})\lvert 0^{(0)}\rangle=0,\\
            \notag\lambda^2\sum_{k\neq 0}\frac{\abs{(H_1)_{0k}}^2}{E_0^{(0)}-E_k^{(0)}}=&b^2\sum_{k\neq 0}\frac{\abs{\langle 0^{(0)}\rvert x\lvert k^{(0)}\rangle}^2}{\frac{\hbar\omega}{2}-\hbar\omega\left(k+\frac{1}{2}\right)}\\
            \notag=&-b^2\sum_{k\neq 0}\frac{\abs{\langle 0^{(0)}\rvert\sqrt{\frac{\hbar}{2m\omega}}(a+a^{\dagger})\lvert k^{(0)}\rangle}^2}{\hbar\omega k}\\
            \notag=&-\frac{b^2}{2m\omega^2},
        \end{align}
        故基态的能移为
        \begin{align}
            \Delta_0=-\frac{b^2}{2m\omega^2}+\cdots
        \end{align}
        \item[(b)] 无微扰下, 该一维简谐振子的哈密顿量为
        \begin{align}
            H=\frac{p^2}{2m}+\frac{m\omega^2x^2}{2},
        \end{align}
        对应的基态能量为
        \begin{align}
            E_0^{(0)}=\frac{\hbar\omega}{2}.
        \end{align}
        在该微扰下, 该一维简谐振子的哈密顿量为
        \begin{align}
            \notag H=&\frac{p^2}{2m}+\frac{m\omega^2x^2}{2}+bx=\frac{p^2}{2m}+\frac{m\omega^2}{2}\left(x+\frac{b}{m\omega^2}\right)-\frac{b^2}{2m\omega^2}\\
            \notag&(\text{作变量代换 }x'=x+\frac{b^2}{m\omega^2})\\
            =&\frac{p^2}{2m}+\frac{m\omega^2x'^2}{2}-\frac{b^2}{2m\omega^2},
        \end{align}
        故其基态能量为
        \begin{align}
            E_0=\frac{\hbar\omega}{2}-\frac{b^2}{2m\omega^2},
        \end{align}
        此时基态的能移为
        \begin{align}
            \Delta_0=E_0-E_0^{(0)}=-\frac{b^2}{2m\omega^2},
        \end{align}
        该结果与前一小题中用微扰理论所得一致.
    \end{itemize}
\end{sol}

\begin{prob}[课本习题 5.2]
    在非简并的时间无关的微扰论中, 在微扰能量本征态 ($\lvert k\rangle$) 中找到相应的无微扰本征态 ($\lvert k^{(0)}\rangle$) 的概率是什么? 求解这个问题至 $g^2$ 级.
\end{prob}
\begin{sol}
    归一化的微扰本征态为
    \begin{align}
        \lvert k\rangle=Z_k^{1/2}\left[\lvert k^{(0)}\rangle+\lambda\sum_{n\neq k}\lvert n^{(0)}\rangle\frac{V_{nk}}{E_k^{(0)}-E_n^{(0)}}\right],
    \end{align}
    其中
    \begin{align}
        \notag Z_k^{-1}=&(\langle k^{(0)}\rvert+\lambda\langle k^{(1)}\rvert+\lambda^2\langle k^{(2)}\rvert+\cdots)(\lvert k^{(0)}\rangle+\lambda\lvert k^{(1)}\rangle+\lambda^2\lvert k^{(2)}\rangle+\cdots)\\
        \notag=&1+\lambda^2\langle k^{(1)}\vert k^{(1)}\rangle+\mathcal{O}(\lambda^3)\\
        =&1+\lambda^2\sum_{n\neq k}\frac{\abs{V_{nk}}^2}{(E_k^{(0)}-E_n^(0))^2}+\mathcal{O}(\lambda^3).
    \end{align}
    在该微扰能量本征态中找到相应的无微扰本征态 ($\lvert k^{(0)}\rangle$) 的概率是
    \begin{align}
        P(\lvert k^{(0)}\rangle)=\abs{\langle k^{(0)}\vert k\rangle}^2=\abs{Z_0}=\left[1+\lambda^2\sum_{n\neq k}\frac{\abs{V_{nk}}^2}{(E_k^{(0)}-E_n^{(0)})^2}+\mathcal{O}(\lambda^3)\right]^{-1}=1-\lambda^2\sum_{n\neq k}\frac{\abs{V_{nk}}^2}{(E_k^{(0)}-E_n^{(0)})^2}+\mathcal{O}(\lambda^3).
    \end{align}
\end{sol}

\begin{prob}[课本习题 5.5]
    对 (5.1.50) 式给出的一维谐振子和一个额外的微扰 $V=\frac{1}{2}\varepsilon m\omega^2x^2$ 建立 (5.1.54) 式. 证明其他所有的矩阵元 $V_{k0}$ 为零.
\end{prob}
\begin{pf}
    
\end{pf}
\end{document}